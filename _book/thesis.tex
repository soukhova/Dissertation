%%%%%%%%%%%%%%%%%%%%%%%%%%%%%%%%%%%%%%%%%
% McMaster Masters/Doctoral Thesis
% LaTeX Template
% Version 2.2 (11/23/15)
%
% This template has been downloaded from:
% http://www.LaTeXTemplates.com
% Then subsequently from http://www.overleaf.com
%
% Version 2.0 major modifications by:
% Vel (vel@latextemplates.com)
%
% Original authors:
% Steven Gunn  (http://users.ecs.soton.ac.uk/srg/softwaretools/document/templates/)
% Sunil Patel (http://www.sunilpatel.co.uk/thesis-template/)
%
% Modified to McMaster format by Benjamin Furman (contact: https://www.xenben/com; Most up
% to date template at https://github.com/benjaminfurman/McMaster_Thesis_Template,
% occasionally updated on Overleaf template page)
%
% Modified for macdown by Antonio Paez; most up to date version at https://github.com/paezha/macdown
%
% License:
% CC BY-NC-SA 3.0 (http://creativecommons.org/licenses/by-nc-sa/3.0/)
%
%%%%%%%%%%%%%%%%%%%%%%%%%%%%%%%%%%%%%%%%%

%----------------------------------------------------------------------------------------
% DOCUMENT CONFIGURATIONS
%----------------------------------------------------------------------------------------

\documentclass[
11pt, % The default document font size, options: 10pt, 11pt, 12pt
oneside, % Two side (alternating margins) for binding by default, uncomment to switch to one side
english, % other languages available
singlespacing, % Single line spacing, alternatives: onehalfspacing or doublespacing
%draft, % Uncomment to enable draft mode (no pictures, no links, overfull hboxes indicated)
%nolistspacing, % If the document is onehalfspacing or doublespacing, uncomment this to set spacing in lists to single
%liststotoc, % Uncomment to add the list of figures/tables/etc to the table of contents
%toctotoc, % Uncomment to add the main table of contents to the table of contents
]{macthesis} % The class file specifying the document structure

%----------------------------------------------------------------------------------------
% Import packages here
%----------------------------------------------------------------------------------------
\usepackage[utf8]{inputenc} % Required for inputting international characters
\usepackage[T1]{fontenc} % Output font encoding for international characters
\usepackage{lastpage} % count pages
\usepackage{lmodern} % could change font type by calling a different package
\usepackage{lscape} % for landscaping pages
% New commands for landscape orientation
\newcommand{\blandscape}{\begin{landscape}}
\newcommand{\elandscape}{\end{landscape}}
%
\usepackage{siunitx} % for scientific units (micro-liter, etc)
\setcounter{tocdepth}{2} % so that only section and sub sections appear in Table of Contents. Remove or set depth to 3 to include sub-sub-sections

%----------------------------------------------------------------------------------------
% Define a blank page
%----------------------------------------------------------------------------------------
\def\blankpage{%
      \clearpage%
      \thispagestyle{empty}%
      \addtocounter{page}{-1}%
      \null%
      \clearpage}

%----------------------------------------------------------------------------------------
% Define a tight list
%----------------------------------------------------------------------------------------
\def\tightlist{}

%----------------------------------------------------------------------------------------
%	Highlight Code Chunks
%----------------------------------------------------------------------------------------

%----------------------------------------------------------------------------------------
% Handling Citations
%----------------------------------------------------------------------------------------

% definitions for citeproc citations
\NewDocumentCommand\citeproctext{}{}
\NewDocumentCommand\citeproc{mm}{%
\begingroup\def\citeproctext{#2}\cite{#1}\endgroup}
\makeatletter
% allow citations to break across lines
\let\@cite@ofmt\@firstofone
% avoid brackets around text for \cite:
\def\@biblabel#1{}
\def\@cite#1#2{{#1\if@tempswa , #2\fi}}
\makeatother
\newlength{\cslhangindent}
\setlength{\cslhangindent}{1.5em}
\newlength{\csllabelwidth}
\setlength{\csllabelwidth}{3em}
\newenvironment{CSLReferences}[2] % #1 hanging-indent, #2 entry-spacing
{\begin{list}{}{%
	\setlength{\itemindent}{0pt}
	\setlength{\leftmargin}{0pt}
	\setlength{\parsep}{0pt}
	% turn on hanging indent if param 1 is 1
	\ifodd #1
	\setlength{\leftmargin}{\cslhangindent}
	\setlength{\itemindent}{-1\cslhangindent}
	\fi
	% set entry spacing
	\setlength{\itemsep}{#2\baselineskip}}}
{\end{list}}
\usepackage{calc}
\newcommand{\CSLBlock}[1]{\hfill\break\parbox[t]{\linewidth}{\strut\ignorespaces#1\strut}}
\newcommand{\CSLLeftMargin}[1]{\parbox[t]{\csllabelwidth}{\strut#1\strut}}
\newcommand{\CSLRightInline}[1]{\parbox[t]{\linewidth - \csllabelwidth}{\strut#1\strut}}
\newcommand{\CSLIndent}[1]{\hspace{\cslhangindent}#1}


%----------------------------------------------------------------------------------------
% Collect all your header information from the chapters here, things like acronyms, custom commands, necessary packages, etc.
%----------------------------------------------------------------------------------------
\usepackage{parskip} %this will put spaces between paragraphs
\setlength{\parindent}{15pt} % this will create and indent on all but the first paragraph of each section.
% should maybe change to glossaries package
\usepackage{acro}
\DeclareAcronym{est}{
	short = EST,
	long  = expressed sequence tags
}

\DeclareAcronym{Xl}{
	short = \textit{X.~laevis},
	long  = \textit{Xenopus~laevis}
}
\DeclareAcronym{Xg}{
	short = \textit{X.~gilli},
	long  = \textit{Xenopus~gilli}
}

\usepackage{etoolbox}
\preto\chapter{\acresetall} % resets acronyms for each chapter

\usepackage{xspace} %helps spacing with custom commands.
\newcommand{\oddname}{{\sc SoME goOfY LonG ThiNg With an AwkWarD NAme}\xspace}


\usepackage{pgfplotstable} % a much better way to handle tables
\pgfplotsset{compat=1.12}

% \usepackage{float} % if you need to demand figure/table placement, then this will allow you to use [H], which demands a figure placement. Beware, making LaTeX do things it doesn't want may lead to oddities.


%%%%
% LINK COLORS
% You can control the link colors at the end of the McMasterThesis.cls file. There is also a true/false option there to turn off all link colors.
%%%%


%----------------------------------------------------------------------------------------
%	THESIS INFORMATION
%----------------------------------------------------------------------------------------

\title{A family of accessibility measures: bringing practical interpretation to access inequities}
%\thesistitle{Thesis Title} % Your thesis title, print it elsewhere with \ttitle
\author{Anastasia Soukhov}
%\author{John \textsc{Smith}} % Your name, print it elsewhere with \authorname
\bdegree{B.Eng.}
\mdegree{M.A.Sc.}
%Previous degrees % print it elsewhere with \bdeg and \mdeg
\date{June 2025}
% The month and year that you submit your FINAL draft TO THE LIBRARY (May or December)
\university{McMaster University}
%\university{\href{http://www.mcmaster.ca/}{McMaster University}} % Your university's name and URL, print it elsewhere with \univname
%\division{}
\faculty{Faculty of Science} % Your faculty's name and URL, print it elsewhere with \facname
\department{School of Earth, Environment and Society} % Your department's name and URL, print it elsewhere with \deptname
\subject{Geography} % Your subject area, print it elsewhere with \subjectname
%\group{\href{http://researchgroup.university.com}{Research Group Name}} % Your research group's name and URL, print it elsewhere with \groupname
\supervisor{Antonio Paez}
%\supervisor{Dr. Jane \textsc{Smith}} % Your supervisor's name, print it elsewhere with \supname
\examiner{} % Your examiner's name, print it elsewhere with \examname
\degree{Doctor of Philosophy}
%\degree{Doctor of Philosophy} % Your degree name, print it elsewhere with \degreename
\addresses{} % Your address, print it elsewhere with \addressname
\keywords{} % Keywords for your thesis, print it elsewhere with \keywordnames


% this sets up hyperlinks
\hypersetup{pdftitle=\ttitle} % Set the PDF's title to your title
\hypersetup{pdfauthor=\authorname} % Set the PDF's author to your name
\hypersetup{pdfkeywords=\keywordnames} % Set the PDF's keywords to your keywords

\begin{document}
\sloppy

\frontmatter % Use roman page numbering style (i, ii, iii, iv...) for the pre-content pages

\pagestyle{plain} % Default to the plain heading style until the thesis style is called for the body content

%----------------------------------------------------------------------------------------
%	Half Title (lay title)
%----------------------------------------------------------------------------------------
%\begin{halftitle} % could not get this environment working
%\vspace*{\fill}
\vspace{6cm}
\begin{center}
\ttitle
\end{center}
%\vspace*{\fill}
\pagenumbering{gobble} % leave this here, McMaster doesn't want this page numbered
%\end{halftitle}
\clearpage

%----------------------------------------------------------------------------------------
%	TITLE PAGE
%----------------------------------------------------------------------------------------
\pagenumbering{gobble}
\begin{center}

\vfill
\textsc{\Large \ttitle} \\

\vfill
{By \authorname\, \bdeg \, \mdeg }


 \vfill
{\large \textit{A Thesis Submitted to the School of Graduate Studies in the Partial Fulfillment of the Requirements for the Degree \degreename}}\\

\vfill
{\large \univname\, \copyright\, Copyright by \authorname\, \today}\\[4cm] % replace \today with the submission date

\end{center}
\blankpage
\clearpage

%----------------------------------------------------------------------------------------
%	QUOTATION PAGE
%----------------------------------------------------------------------------------------

\vspace*{0.2\textheight}

\noindent{\itshape SOME QUOTE}\bigbreak

\hfill\textemdash Some author of a quote

\blankpage
\clearpage

%%%%%%%%%%%%%%%%%%%%%%%%%%%
%%%%%%%%%%%%%%%%%%%%%%%%%%%
% optional page stuff
%----------------------------------------------------------------------------------------
% can do physical constraints and symbols pages, see the original thesis example on overleaf if you want to include them at https://www.overleaf.com/latex/templates/template-for-a-masters-slash-doctoral-thesis/mkzrzktcbzfl#.VlPeicorpE4
%----------------------------------------------------------------------------------------

%----------------------------------------------------------------------------------------
%	DEDICATION
%----------------------------------------------------------------------------------------

    You can have a dedication here if you wish.

\blankpage
\clearpage


%----------------------------------------------------------------------------------------
%	Descriptive note numbered ii
%----------------------------------------------------------------------------------------
% Need to add below info
\newpage
\pagenumbering{roman} % leave to turn numbering back on
\setcounter{page}{2} % leave here to make this page numbered ii, a Grad School requirement

\noindent % stops indent on next line
\univname \\
\degreename\, (\the\year) \\
Hamilton, Ontario (\deptname) \\[1.5cm]
TITLE: \ttitle \\
AUTHOR: \authorname\,  %list previous degrees
(\univname)  \\
SUPERVISOR: \supname\, \\
NUMBER OF PAGES: \pageref{lastoffront}, \pageref{LastPage}  % put in iv and number

\clearpage

%----------------------------------------------------------------------------------------
%	Lay abstract number iii
%----------------------------------------------------------------------------------------
% not actually included in most theses, though requested by the GSA
% uncomment below lines if you want to include one
\section*{Lay Abstract}
  (150 words or less).

  The aim of transportation systems is to connect people and opportunities (i.e., jobs, services). However, traditionally transportation planning has focused on mobility (distances travelled), instead of accessibility (how many opportunities can be reached). Experts have been calling for a shift from mobility-based methods to access-based ones, but this change hasn't fully happened yet for a variety of reasons. One challenge is methodological: the lack of clear units for measuring accessibility. This thesis aims to help address this gap by outlining how accessibility methods relate to dominant mobility-based techniques, and how the concept of `constraints' can be borrowed from these techniques to re/introduce units for accessibility measures.
\blankpage
\clearpage


%----------------------------------------------------------------------------------------
%	ABSTRACT PAGE number iv
%----------------------------------------------------------------------------------------

\section*{\Huge Abstract}
\addchaptertocentry{\abstractname}
% Type your abstract here.
Transportation systems plays a fundamental role in cities by facilitating access between people and various social and economic opportunities. Access, otherwise known as Accessibility, can be defined as the potential to spatially interact with opportunities. However, for decades transportation planning has relied on mobility-based estimates and indicators, oriented based on realized movement (e.g., kilometres travelled, emissions released) as opposed to potential movement (e.g., the number of opportunities that can be reached). In recent years, there has been calls to move from mobility-based methods to access-based ones, but a few barriers remain. One signficant issue is methodological, namely, the lack in clarity in the interpretion of conventional accessiblity measures' scores. An emerging approach is to link these scores to outcomes, but this thesis proposes a preceeding step: clarifying the units of accessibility.

In this line, the aim of this thesis is fourfold: 1) review how accessibility literature largely diverged from the spatial interaction literature, but how the addition of a proportionally constaint that both returns the units to the measure and balances them to reflect known constraints in the system may be of use. 2) formally introduce the total constraint, equivalent to the conventional accessibility measure in magnitude but now results are in units of opportunities. 3) introduce the single constraint which considers population competition for opportunities and could be understood as the 2SFCA before normalizing by capita. 4) demonstrate these constrained accessibility measures use on an empirical example of accessibility to parks in Toronto, and how constrained accessibility (i.e.., in units of opportunities) can be used to communicate for the purpose of policy.
\blankpage
\clearpage

%----------------------------------------------------------------------------------------
%	ACKNOWLEDGEMENTS
%----------------------------------------------------------------------------------------

  \begin{acknowledgements}
  \addchaptertocentry{\acknowledgementname} % Add the acknowledgments to the table of contents
    I want to thank a few people. Antonio Paez. Moataz Mohamed. Chris Higgins. Core team. Rafael Perriera. Accessibility nerds at the City of Toronto: dedicated to developing equity thinking in a systematic and rigourous way, Lorina, Michael, Bryce, Herman, and many others.

    Along the way, the mobilizing justice team: Nacho, Matthew Palm, Steven Farber, Joao, Robert. UGs at UpfT and McMaster. Colleagues at McMaster - especially Lea Ravensenberg, that led me to supervise Nicholas Mooney, my first supervisee as a PhD, and an incredible opporuntity to work with Angel and Isla. To colleagues at UPM. Javier, Julio, Carlos. Lab mates Alberto, Manu, Amor, Raul, Manuel and others at TRANSyT. To Colleages at UoFT especially Madealine. And my many friends along the way at the many international conferences, NARSC, AAG, TRB, WCTRS, and especially NECTAR\ldots{}
  \end{acknowledgements}
\blankpage
\clearpage

%----------------------------------------------------------------------------------------
%	LIST OF CONTENTS/FIGURES/TABLES PAGES
%----------------------------------------------------------------------------------------

\tableofcontents % Prints the main table of contents

\listoffigures % Prints the list of figures

\listoftables % Prints the list of tables

%----------------------------------------------------------------------------------------
%	ABBREVIATIONS
%----------------------------------------------------------------------------------------
% many theses don't use this section, as it will be declared at first use and again each chapter. Uncomment these four lines to activate if you want
%\clearpage
%\section*{\Huge Acronyms}
%\addchaptertocentry{Acronyms}
%\printacronyms[name] % name without an option stops the header

%----------------------------------------------------------------------------------------
%	DECLARATION PAGE
%----------------------------------------------------------------------------------------

\begin{declaration}
\addchaptertocentry{\authorshipname}

\noindent I, \authorname, declare that this thesis titled, \emph{\ttitle} and the work presented in it are my own. I confirm that:

I did most of the research.

Also the writing.

Sometimes I cried.

But mostly I had fun.

\end{declaration}


%----------------------------------------------------------------------------------------
% The following bit is just here to make sure we end up on a new page and get the total number of roman numeral
\label{lastoffront}
\clearpage
% make sure this command is on the last of your frontmatter pages, i.e. only this command, a \clearpage then \mainmatter
% should be fine without modification
%----------------------------------------------------------------------------------------

%----------------------------------------------------------------------------------------
%	THESIS MAIN BODY
%----------------------------------------------------------------------------------------

\mainmatter % here the regular arabic numbering starts
\pagestyle{thesis}
\chapter*{Preface - Introduction}\label{preface---introduction}
\addcontentsline{toc}{chapter}{Preface - Introduction}

\section{This dissertation's evolution}\label{this-dissertations-evolution}

This work has come to be non-chronologically. In the summer of 2020, a largely `virtual' summer for many of us, I began working with Dr.~Antonio Paez on a reading course as a Masters of Applied Science student under the supervision of Dr.~Moataz Mohammed. At that point, I was nearing the end of my Master's program, writing a MASc. thesis aimed at providing guidance on right-sizing passenger transportation sustainability policies based on vehicle technologies well-to-wheel life cycle operating conditions e.g., (Soukhov, Foda, \& Mohamed, 2022; Soukhov \& Mohamed, 2022). Coming from transportation engineering, I had to take on a new research process in engaging with this new research. Antonio, who had recently developed a deep appreciation for open and reproducible science and the R universe, combined with a passion for historical analysis and use of data for advocacy, had worked with students before me to gather information on public schools in Hamilton. A public school in his neighborhood, along with several others across the city, was closing, and he was keen to explore the impacts these closures would have.

It was within this context that I began the reading course. I found myself intuitively working through the available data, building upon my beginner R skills, estimating additional data to supplement the analysis, and crafting research questions that felt sufficiently satisfying; questions that blended policy impacts, spatial analysis, and active transportation implications of a change in school-seats. Spatial accessibility, or the potential to spatially interact with school-seats, as an indicator that could answer these questions. Accessibility, conventionally defined as the sum of opportunities \(O_j\) weighted by the travel impedance function \(f(c_{ij})\) of some travel cost \(c_{ij}\) for a set of zones of origin \(i\) \(i...I\) and destination \(j\) \(j...J\) in Equation \ref{eq:access-unconstrained}, takes the following general form:

\begin{equation}
\label{eq:access-unconstrained}
S_i = \sum_i O_j f(c_{ij}) 
\end{equation} 

Driven by my engineering training, I wanted as precise a solution with an interpretable meaning at the highest spatial resolution possible. For instance: how had the number of potentially reachable school-seats change for households? Looking back now, that's a high expectation from a place-based accessibility metric, but it was my engineering background that was informing this desire and fixation on units. I knew the number of school-seats in Hamilton for each study year, the number of students, and I knew that, intuitively, the school-seat accessibility for each household should decrease within proximity to schools. However, conventional methods produced results that were difficult to interpret. Moreover, competitive measures, such as those often applied using Floating Catchment Areas (FCA), had their own inflationary issues (i.e., discussed in Antonio Paez, Higgins, \& Vivona, 2019).

This challenge led towards developing a more intuitive approach, one that involved establishing proportional allocation factors. These factors allocate opportunities proportionally based on a zone's population relative to the region's total population, as well as the zone's travel impedance to other zones relative to the travel impedance to all zones from that zone. The balancing factors ensure that each parcel is assigned a portion of the total school-seats in the region. This intuitive approached ensured the school-seats from each school is proportionally allocated to each zone based on the zone's relative population and travel impedance \emph{and} each value summed to the total number of opportunities in the region. This method of competitive accessibility, which preserves its units, was introduced as ``spatial availability'' (Soukhov, Paez, Higgins, \& Mohamed, 2023).

Chris Higgins, an early collaborated, helped us realize ``spatial availability'' is \emph{singly-constrained} from the perspective of opportunities, akin to the singly-constrained spatial interaction model in Wilson (1971) We adopted this conceptualisation, without scrutinizing the Wilson's framework (at the time), and published ``spatial availability'', a singly-constrained competitive accessibility measure in \emph{PLOS ONE} (Soukhov et al., 2023). An open data paper demonstrating my growing expertise in R and reproducible methods in \emph{Environment and Planning B: Urban Analytics and City Science} (Soukhov \& Páez, 2023) then followed, followed by a multimodal extension of the spatial availability measure the year after (Soukhov, Tarriño-Ortiz, Soria-Lara, \& Páez, 2024). Along with years of thinking about the equity and justice implications of transportation systems summarized in the (forthcoming) review of the literature Soukhov, Aitken, Palm, Farber, \& Paez (2025), the first work -that started back in my MASc during that reading course- the article assessing the change in active (and motorized) school-seat accessibility after 10 years of school closures was published in \emph{Networks and Spatial Economics} (Soukhov, Higgins, Páez, \& Mohamed, 2025).

However, in the past few months, I returned to Chris' original comment: how similar is the spatial availability balancing factor to the single constraint in the spatial interaction model? It turns out, they can be simplified to be identical. This means that the most popular competitive accessibility measure ---the 2SFCA--- can also be seen as proportional to the singly-constrained expression in Wilson (NOTE: in Soukhov et al. (2023), the mathematical equivalence between spatial availability per capita and 2SFCA is established). And, requiring only information about the total number of opportunities in the region and non-competitive in its definition, a `total constraint', fitting within Wilson's framework, can also be established. The total constraint ensures the zonal accessibility values are proportional to the total in the region, share the same scaled magnitude as conventional accessibility in Equation \ref{eq:access-unconstrained}, but maintains units of the accessibility concept , i.e.~the \emph{number of accessible opportunities}.

Thematically, this thesis begins with this forthcoming (and submitted) work of Soukhov, Pereira, Higgins, \& Paez (2025), which details the precedent, logic, and intuition of weaving `constraints' from the spatial interaction model into measures of potential for spatial interaction: both the singly-constrained version (as previously published) and, for the first time, introducing the total constraint are included.

In summary, this this dissertation is based on the following publications (in chronological order):

\begin{itemize}
\tightlist
\item
  Soukhov, A., Pereira, R. H. M., Higgins, C. H. \& Paez, A. (2025). A family of accessibility measures derived from spatial interaction principles. (Forthcoming - submitted to \emph{XXX}).
\item
  Soukhov, A., Higgins, C. D., Páez, A., \& Mohamed, M. (2025). Ten Years of School Closures and Consolidations in Hamilton, Canada and the Impact on Multimodal Accessibility. Networks and Spatial Economics. \url{https://doi.org/10.1007/s11067-025-09677-z}
\item
  Soukhov, A., Aitken, I. T., Palm, M., Farber, S., \& Paez, A. (2025). Searching for fairness standards in the transportation literature. (Forthcoming - submitted to \emph{Transportation}). An extended version, as a report, can be found here: \url{https://mobilizingjustice.ca/wp-content/uploads/2024/01/just-transportation-1.pdf}
\item
  Soukhov, A., Tarriño-Ortiz, J., Soria-Lara, J. A., \& Páez, A. (2024). Multimodal spatial availability: A singly-constrained measure of accessibility considering multiple modes. PLOS ONE, 19(2), e0299077. \url{https://doi.org/10.1371/journal.pone.0299077}
\item
  Soukhov, A., \& Páez, A. (2023). TTS2016R: A data set to study population and employment patterns from the 2016 Transportation Tomorrow Survey in the Greater Golden Horseshoe area, Ontario, Canada. Environment and Planning B: Urban Analytics and City Science, 23998083221146781. \url{https://doi.org/10.1177/23998083221146781}
\item
  Soukhov, A., Paez, A., Higgins, C. D., \& Mohamed, M. (2023). Introducing spatial availability, a singly-constrained measure of competitive accessibility. PLOS ONE, 1--30. \url{https://} doi.org/10.1371/journal.pone.0278468
\end{itemize}

\section{The importance of accessibility; and conceptual issues interpreting `unconstrained' access}\label{the-importance-of-accessibility-and-conceptual-issues-interpreting-unconstrained-access}

Transportation is a special kind of land-use that compresses space and time, enabling connections, and hence potential interaction, between populations in different zones. But, as posed in S. Handy (2020), ``is accessibility an idea whose time has finally come?'' This concept has been discussed in the literature for decades as the ulterior goal of transportation systems. Accessibility doesn't express mobility directly but instead captures the potential for meaningful mobility --- the reachability of opportunities. Despite the utility of accessibility measures in both literature and practice to identify areas of high and low opportunity access, the interpretability of foundational accessibility measures, such as the Hansen-type accessibility measure (Hansen, 1959) (as in Equation \ref{eq:access-unconstrained} ), can be challenging (Karst T. Geurs \& van Wee, 2004a; E. J. Miller, 2018; Santana Palacios \& El-geneidy, 2022).

By definition, accessibility is the product of the sum of opportunities that can be reached through some travel impedance, meaning each zonal value is sensitive to both inputs in a way that isn't constrained by any known system. For example, consider comparing the accessibility values for a region between two years where the number of opportunities remained constant but the travel behaviour dramatically changed. In this case, both years would have travel impedance functions of different forms. As a result, the zone values would have different units, one in opportunities-weighted-travel-impedance-value-from-form-1 and another in units of opportunities-weighted-travel-impedance-value-from-form-2. These values would exist within different ranges, and while they could be normalized and compared, this process introduces bias. Consider also, when comparing values between regions, the number of opportunities varies with the size of the region (e.g., larger regions generally have more opportunities than smaller ones). For this reason, raw accessibility scores aren't necessarily comparable without some form of normalization, which again introduces bias. As with many spatial issues, these problems are compounded by the Modifiable Areal Unit Problem (MAUP) and Modifiable Temporal Unit Problem (MTUP), where the zonal and temporal units may differ between comparison periods, further obscuring the results.

Aside from temporal and regional comparisons, gleaning interpretation of raw `opportunities-weighted-travel-impedance-value' scores within the same temporal and spatial region is challenging. To move closer to an intuitive interpretation, a balancing factor that is sensitive to the known information about the transportation system can be introduced. This would allow the zonal value of `how many opportunities can be accessed' to be more plainly contemplated - as the units of travel impedance is set aside. In this way, linking accessibility values to something else meaningful (e.g., outcomes) can become more straightforward and interpretable.

\section{Aims}\label{aims}

The primary contribution of this dissertation is demonstrating how the units of accessibility -the number of opportunities that can potentially be reached- can be preserved within the zonal values of the accessibility measure. This contribution offers a practical way to enhance accessibility as a planning tool, making it more interpretable and potentially more useful for decision-making.

This dissertation offers four key contributions. First, it defines accessibility as the \emph{potential} for spatial interaction by explicitly linking it the spatial interaction literature, and providing a review of how the two streams of literature diverged. Next, it explores how spatial interaction modelling literature maintained `units' in outputs by introducing system and zonal constraints through balancing factors, and shows how accessibility can be reformulated using these same constraints.

Second, the total constraint is introduced, which is conceptually equivalent in magnitude to the Hansen-type accessibility measure (Equation \ref{eq:access-unconstrained}) but is expressed in units of `accessible opportunities'. This approach can help ensure that accessibility measures remain more consistent and comparable across different areas and time periods, addressing the issue of unit discrepancies that often arise in accessibility analyses.

Third, the single constraint is introduced, which accounts for competition among populations for access to opportunities. This constraint can be understood as a competitive measure, and offering an additional layer of constraints (e.g., ensuring opportunities at a destination are proportionally distributed to reachable origins based on relative population and travel impedance, as well as maintaining the total constraint). It can also be seen as a more general version of the Two Step Floating Catchment Area (2SFCA) method (Luo \& Wang, 2003a), before it is divided by zonal population. In this way, singly-constrained accessibility can offer a more accurate understanding of localized competition for opportunities.

Fourth, the multi-modal extension of both totally- and singly- constrained accessibility is specified. Specifically, these extensions demonstrate how these measures can be adapted to consider different groups with different travel impedance functions (i.e., be it multiple modes of transport or populations with different travel likelihoods) accessing the same opportunities. This multi-modal extension can be used to interpret the accessibility gap between groups, illustrating how many more opportunities can be reached by one group compared to another.

\section{A practical case study: Toronto and publically owned and opertated green space}\label{a-practical-case-study-toronto-and-publically-owned-and-opertated-green-space}

The magnitude of opportunities are reachable, and the distribution of this value spatially or among people, is often the preoccupation of those who use accessibility measures. An advantage of the \emph{constrained} accessibility measures, is in their interpretation. They can be used in place of a static metrics of opportunity per zone or opportunity per capita in a zone metric (e.g., the number of jobs located in a census block, or the number of hospital beds per person in a neighbourhood). Instead, constrained accessibility measures yield a value of the number of \emph{accessible} opportunities per zone or per capita (in a zone), cleanly folding in the assumptions about how the population may interact with that opportunity (transportation-land-use interaction), but still allowing the value itself to be understood like a level of service metric.

To demonstrate the practical advantage constrained accessibility's interpretation, a case study reporting the accessibility to green space that is owned and operated by the City of Toronto will be detailed. Green space is important to well being\ldots{} XYZ. Furthermore, the City is undergoing benchmarking of access to different opportunity types. From this perspective, greenspace is an important public service to benchmark using accessibility, as its use is related to its ease of reach {[}{]}.

Green space is an interesting opportunity, as there are a variety of types with different qualities that may impact how important population-competition considerations are. For instance, smaller neighbourhood parks typically only attract people residing in proximity to these parks, and the furniture may get congested - meaning competition may matter, but only locally. However, data reflection the number of people that visit these parks is more likely to be unavailable. On the other hand, large lgeacy parks with natural attractions may attract people from further distances, so congestion of the park may be more of a pertinent consideration.

With the aim of showcasing constrained accessibility's practical interpretation, the calculation and comparision between totally constrained and singly constrained accessibility will be outlined in this work. Constrained accessibility for each zonal origin will be calculated using the total constraint assuming one travel impedance function and a multi-modal extension, showing how constrained accessibility, expressed in units of accessible opportunities, can be aggregated across different spatial and temporal scales. This aggregation allows for a clearer communication of the policy implications of transportation and land-use decisions, making accessibility measures more communicable for urban planning and decision-making.

\section{Overview of methods}\label{overview-of-methods}

Firstly, this entire thesis adopts principles of open and reproducible methods; using R and RStudio workflows, everything from the manuscript, code, and the final manuscript is transparent and freely available for all to replicate {[}X{]}.

\subsection{Accessibility metrics}\label{accessibility-metrics}

The dissertation begins with the demonstration of the unconstrained accessibility (Equation \ref{eq:access-unconstrained}), then totally constrained accessibility, and singly-constrained accessibility. Framing of the constrained and unconstrained accessibility methods in the spatial interaction literature.

\subsection{Data sources and travel time estimates}\label{data-sources-and-travel-time-estimates}

\begin{enumerate}
\def\labelenumi{\arabic{enumi})}
\item
  Parks - from City of Toronto portal.
\item
  Canadian census for Toronto CMA. Population weighted centroids. Census data at the DA. Neighbourhoods.
\item
  Empirical travel flows from the 2022 TTS for leisure trips for different modes (as included in the updated version of the TTS2016R package (text from \emph{``TTS2016R: A data set to study population and employment patterns from the 2016 Transportation Tomorrow Survey in the Greater Golden Horseshoe area, Ontario, Canada''} is included)
\item
  Using modified OSM Toronto network, routed r5r travel times for multiple modes.
\end{enumerate}

\section{Chapters outline}\label{chapters-outline}

This dissertation is divided into six chapters. The first provides the general framework of this thesis as written in forthcoming \emph{``Family of accessibility measures derived from spatial interaction principles''} paper submitted to \emph{PLOS ONE}. This chapter outlines the evolution of the spatial interaction literature, how accessibility literature diverged, and introduces the total and single constraints using numeric examples.

The second chapter calculates totally-constrained accessibility for an empirical example of green space in Toronto. A discussion of the balancing factor is included.

The third chapter calculates singly-constrained accessibility for the same empirical example of green space in Toronto. This chapter pulls the introduction and discussion of the \emph{``Introducing spatial availability, a singly-constrained measure of competitive accessibility''} paper. A discussion of the balancing factor is included.

The fourth chapter calculates a multi-modal extension of the totally constrained and singly-constrained accessibility calculate for the same empirical example of green space in Toronto. However two modes are considered: X and Y. This chapter pulls the introduction and discussion of the \emph{``Multimodal spatial availability: A singly-constrained measure of accessibility considering multiple modes''} paper.

The fifth chapter includes a comprehensive comparison of unconstrained (Hansen-type), 2SFCA (shen-type) and all the constrained versions of accessibility for the empirical example. A discussion on how they rank under different spatial aggregations is included (text used from \emph{``Ten Years of School Closures and Consolidations in Hamilton, Canada and the Impact on Multimodal Accessibility''}).

The sixth chapter concludes the work. the open and reproducibility of methods learned is discussed (text from \emph{``TTS2016R: A data set to study population and employment patterns from the 2016 Transportation Tomorrow Survey in the Greater Golden Horseshoe area, Ontario, Canada''} is included), how constrained measures can be interpreted, how this interpretation can aid policy makers, and how policy makers can ultimately use this to plan for inequities (some text pulled from \emph{``Searching for fairness standards in the transportation literature''}). Future research directions (?)

\chapter{CHP 1 - A common history between accessibility and spatial interaction principles}\label{chp-1---a-common-history-between-accessibility-and-spatial-interaction-principles}

\section{Chapter overview}\label{chapter-overview}

This chapter consists of text from the first half of the \emph{``Family of accessibility measures derived from spatial interaction principles''} authored by Anastasia Soukhov, Rafael H. M. Pereira, Christopher D. Higgins, and Antonio Páez and submitted to \emph{PLOS ONE} in May 2025. This portion of the submitted manuscript reviews the historical background of spatial interaction modelling in the context of accessibility, the use of proportionality constants--linked to empirical information about the system--to maintain units, and the divergence of accessibility's practice in applying this approach. Overall: this chapter provides the historical context for the introduction of the family of measures, providing supporting motivation for the formulation of the measures and their use in various applications throughout this dissertation.

\section{Introduction}\label{introduction}

In the early nineteenth century, industrializing cities grew rapidly: so did traffic congestion and the development of transportation planning practice focused primarily on mobility. In this newly founded practice, access to destinations was treated as a by-product of movement. Following the rise of the automobile and significant investments in transportation infrastructure after World War II, this mobility-oriented approach has led to some problematic outcomes. Specifically, the car became seen as the ultimate mobility tool, helping to foster the development of low-density, single-use residential neighborhoods and entrenching an automobility mono-culture within transportation systems (Lavery, Páez, \& Kanaroglou, 2013; H. J. Miller, 2011). Decades of planning for this automobility mono-culture, still often characterized by road and highway expansion, have been marked by increased travel costs and environmental burdens, with limited impact on enhancing the ease with which people can reach destinations (Steven Farber \& Páez, 2011; S. Handy, 2002; Antonio Páez, Mercado, Farber, Morency, \& Roorda, 2010). In response, transportation researchers have increasingly advocated for the adoption of accessibility as a planning criterion, in contrast to traditional mobility-oriented transportation planning approaches which translate into indicators that benchmark movement (e.g., vehicle kilometres traveled, intersection through traffic, etc.) which are not necessarily linked to improved accessibility (El-Geneidy \& Levinson, 2022; S. Handy, 2020; A. Paez, Moniruzzaman, Bourbonnais, \& Morency, 2013; Silva, Bertolini, Te Brömmelstroet, Milakis, \& Papa, 2017).

Accessibility may be defined as the ``potential of opportunities for {[}spatial{]} interaction'' (Hansen, 1959), in contrast to mobility which is simply, movement in itself or `spatial interaction'. Mobility has been the basis of traditional transportation planning approaches (Ortúzar \& Willumsen, 2011). While accessibility brings a more holistic understanding of combined transportation and land use systems, incorporating both the explicit consideration of destination as opportunity and the concept of \emph{potential} (S. L. Handy \& Niemeier, 1997).

The growing interest in accessibility has been accompanied by a boom in scholarly research using different methods and focusing on different research contexts; it has grown to include studies of access to employment (Grengs, 2010a; e.g., Karst \& Van Eck, 2003; Merlin \& Hu, 2017; Antonio Páez, Farber, Mercado, Roorda, \& Morency, 2013; Tao, Zhou, Lin, Chao, \& Li, 2020), health care (Boisjoly, Moreno-Monroy, \& El-Geneidy, 2017; Delamater, 2013; e.g., Luo \& Wang, 2003a; Antonio Páez et al., 2010; Pereira, Braga, et al., 2021; Wan, Zou, \& Sternberg, 2012), green spaces (Liang, Yan, \& Yan, 2024; Reyes, Paez, \& Morency, 2014; Rojas, Paez, Barbosa, \& Carrasco, 2016), schools (Marques, Wolf, \& Feitosa, 2021; Romanillos \& Garcia-Palomares, 2018; e.g., Williams \& Wang, 2014), social contacts (S. Farber, Neutens, Miller, \& Li, 2013; S. Farber, Páez, \& Morency, 2012; e.g., Neutens, Witlox, Van de Weghe, \& De Maeyer, 2007), and regional economic analysis (Gutierrez, Condeco-Melhorado, Lopez, \& Monzon, 2011; Lopez, Gutierrez, \& Gomez, 2008; Ribeiro, Antunes, \& Páez, 2010; e.g., R. Vickerman, Spiekermann, \& Wegener, 1999) among many other domains of application. In other words, accessibility analysis is used today to broadly understand the potential to reach (or spatially interact with) opportunities that are important to people (Ferreira \& Papa, 2020). However, despite its growth in popularity in scholarly works, challenges remain with respect to the more widespread adoption of accessibility in planning practice. For instance, the diversity of accessibility definitions has been flagged by van Wee (2016), S. Handy (2020), and Kapatsila, Palacios, Grisé, \& El-Geneidy (2023). Further, difficulties in the interpretability and communicability of outputs has also been noticed by many authores, including Karst T. Geurs \& van Wee (2004b), van Wee (2016), and Ferreira \& Papa (2020).

The adoption of accessibility in planning practice is not necessarily made easier when potential adopters have to contend with a plethora of definitions, each seemingly more sophisticated but less intuitive than the last (Kapatsila et al., 2023). At a high level, Karst T. Geurs \& van Wee (2004b) identify four families of accessibility measures: infrastructure-, place-, person-, and utility-based. Of the place-based family, which is the focus of this paper, the menu has grown to include gravity-based accessibility (e.g., Hansen, 1959; Pirie, 1979), cumulative opportunities (Pirie, 1979; e.g., Wachs \& Kumagai, 1973; Ye, Zhu, Yang, \& Fu, 2018), modified gravity (e.g., Schuurman, Berube, \& Crooks, 2010), 2-Step Floating Catchment Areas (e.g., Luo \& Wang, 2003a), Enhanced 2-Step Floating Catchment Areas (e.g., Luo \& Qi, 2009), 3-Stage Floating Catchment Areas (e.g., Wan et al., 2012), Modified 2-Step Floating Catchment Areas (e.g., Delamater, 2013), inverse 2-Step Floating Catchment Areas (e.g., F. Wang, 2021), and n-steps Floating Catchment Areas (Liang et al., 2024). How is a practitioner to choose among this myriad options? What differences in accessibility scores should matter, and how should they be communicated? {[}see van Wee (2016); p.~14{]}.

Here, we seek to address the breath of place-based accessibility's definitions and lack of interpretability by demonstrating that mobility-oriented models--such as the commonly used gravity-based accessibility (e.g., Hansen, 1959) and the spatial interaction model (e.g., Wilson, 1971)-- are rooted in the same spatial interaction modeling foundation. In fact, we propose that accessibility can be specified using spatial interaction principles as a \emph{family of measures}, akin to the family of spatial interaction models introduced in Wilson (1971) that can be defined using balancing factors that constrains values based on known information about the land-use.

This work aims to offer three contributions: (1) it introduces a family of accessibility measures within the principles of spatial interaction; (2) it formally defines three \emph{constrained} accessibility measures by reintroducing Wilson-analogous balancing factors. These measures are the total constrained accessibility measure, the singly constrained accessibility measure, and the doubly constrained measure, which are explicitly connected to popular measures such as the Hansen-type accessibility (Hansen, 1959), the popular competition approach of the 2-Step Floating Catchment Area (2SFCA) (Luo \& Wang, 2003a; Q. Shen, 1998a), and the concept of market potential (C. D. Harris, 1954; R. W. Vickerman, 1974). These Wilson-analogous balancing factors introduced are also defined in order of increasing restrictiveness to shed light on the role of \emph{potential} in accessibility and access. (3) This work also demonstrates that the introduction of balancing factors makes accessibility measures easier to interpret and to communicate by restoring the measurement units to the resulting raw accessibility values. Each zonal and zonal flow value from a constrained accessibility measure is always in interpretable units, namely the number of `opportunities for spatial interaction' or `population for spatial interaction'. This is in contrast to conventional accessibility measures, particularly gravity based measures, that yield values in units of `opportunities weighted by some representation of travel friction'.

To achieve these stated objectives, this paper contends that accessibility research must reconnect with its spatial interaction origins. Particularly, we argue that an important aspect of spatial interaction modelling--namely, constraining the results to match empirical observations--was never effectively reincorporated into accessibility analysis. Empirical constraints were embraced by early spatial interaction literature following the work of Wilson (1971), but this stream of literature tended to flow separately from research inspired by Hansen (1959)' accessibility. The application of Wilson (1971)`s empirical constraints supported the development of various spatial interaction models that remain relevant in research and practice today (Ortúzar \& Willumsen, 2011). However, the same cannot be said of the contemporary accessibility literature, where empirical constraints were not explicitly adopted. We argue that the absence of empirical constraints (and their attendant proportionality constants) has contributed to some of accessibility analysis' interpretability issues; for instance, the fuzziness of insights beyond simple proportional statements like `higher-than' or `lower-than' (E. J. Miller, 2018). Moreover, without constraints we lose track of what accessibility measures, that is, the number of opportunities that can be spatially interacted with. Without a clear sense of measurement units, comparability between accessibility measures, across cities, and transport modes, may also become compromised.

Fortunately, accessibility and spatial interaction modelling literature share common headwaters, and the latter has given careful attention to measurement units and their interpretability. It is by looking to the past that we believe accessibility analysis can newly wade into the future. We continue this work in the following section by tracing the development of accessibility from its origins in spatial interaction: from the Newtonian gravitational expression in Ravenstein (1889) through to the seminal accessibility work of Hansen (1959). We then present evidence for a narrative highlighting the marked divergence between accessibility and spatial interaction modelling research after the work of Wilson (1971). Next, we hark back to Wilson (1971)'s spatial interaction models, and use them to derive a family of accessibility measures based on analogous constraints. We illustrate members of this family of constrained accessibility measures with a simple numerical example. We then conclude by discussing the uses of these measure and their interpretation.

\section{Newtonian's roots of human spatial interaction research}\label{newtonians-roots-of-human-spatial-interaction-research}

The patterns of people's movement in space have been a subject of scientific inquiry for at least a century and a half. From as far back as Henry C. Carey's \emph{Principles of Social Science} (Carey, 1858), a concern with the scientific study of human spatial interaction can be observed. It was in this work where Carey stated that \emph{``man [is] the molecule of society [and their interaction is subject to] the direct ratio of the mass and the inverse one of distance''}. (McKean, 1883, pp. 37--38). This statement shows how investigations into human spatial interaction have often been explicitly coloured by the features of Newton's Law of Universal Gravitation, first posited in 1687's \emph{Principia Mathematica} and expressed as in Equation \ref{eq:phys-grav-prop}.

\begin{equation}
\label{eq:phys-grav-prop}
F_{ij} \propto \frac{M_i M_j} {D_{ij}^{2}}
\end{equation} 

To be certain, the expression above is one of proportionality, and is also the most famous in all of science. It states that the force of attraction \(F\) between a pair of bodies \(i\) and \(j\) is directly \emph{proportional} to the product of their masses \(M_i\) and \(M_j\), and inversely \emph{proportional} to the square of the distance between them \(D_{ij}\). Direct proportionality means that as the product of the masses increases, so does the force. Likewise, inverse proportionality means that as the distance increases, the force decreases. Equation \ref{eq:phys-grav-prop}, however, does not quantify the magnitude of the force. To do so, an empirical constant, a.k.a. the gravitational constant, is required to convert the proportionality into an equality, ensuring that values of the force \(F\) in Equation \ref{eq:phys-grav-prop} match the observed force of attraction between masses. In other words, Equation \ref{eq:phys-grav-prop} needs to be \emph{constrained} using empirical data. Ultimately, the equation for the force is as seen in Equation \ref{eq:phys-grav}, where \(G\) is an empirically calibrated proportionality constant:

\begin{equation}
\label{eq:phys-grav}
F_{ij} = G \frac{M_i M_j} {D_{ij}^{2}}
\end{equation} 

Newton's initial estimate of \(G\) was based on a speculation that the mean density of earth was between five or six times that of water, an assumption that received support after Hutton's experiments of 1778 (Hutton, 1778, p. 783). Still, it took over a century from the publication of \emph{Principia} to refine the estimate of the proportionality constant to within 1\% accuracy, with Cavendish's 1798 experiment (Cavendish, 1798).

\section{Early research on human spatial interaction: from Ravenstein (1889) to Stewart (1948)}\label{early-research-on-human-spatial-interaction-from-ravenstein-1889-to-stewart-1948}

Since the 1880s to the 1940s, a number of researchers theoretically and empirically attempted to characterise human spatial interaction as some force of attraction \(F\) that is directly proportion to the masses \(M_i\) and \(M_j\) and inversely proportional by their separation distance. This concept was captured with different expressions, but all tie back to the same Newtonian gravity analogy, although not all of them included a proportionality constant in their formulation.

Following Carey's \emph{Principles} of 1858, research into human spatial interaction continued in different contexts. In the late 1880s, Ravenstein proposed some ``Laws of Migration'' based on his empirical analysis of migration flows in various countries (Ravenstein, 1885, 1889). In these works, Ravenstein posited 1) a directly proportional relationship between migration flows and the size of destinations (i.e., centres of commerce and industry), and 2) an inversely proportional relationship between the size of flows and the separation between origins and destinations. As with Carey, these propositions echo Newton's gravitational laws. Over time, other researchers discovered similar relationships. For example, Reilly (1929) formulated a law of retail gravitation, expressed in terms of equal attraction to competing retail destinations that could be understood as `potential trade territories'. Later, Zipf proposed a \(\frac{P_1P_2}{D}\) hypothesis for the case of information (Zipf, 1946a), intercity personal movement (Zipf, 1946b), and goods movement by railways (Zipf, 1946c). The \(\frac{P_1P_2}{D}\) hypothesis stated that the magnitude of flows was proportional to the product of the populations of settlements, and inversely proportional to the distance between them.

A common feature of these early investigations of human spatial interaction is that a proportionality constant similar to \(G\) in Equation \ref{eq:phys-grav} was never considered. Of the researchers cited above, only Reilly and Zipf expressed their hypotheses in mathematical terms. Reilly's hypothesis was presented in the following form:

\begin{equation}
\label{eq:reilly}
B_a = \frac{(P_a\cdot P_T)^N}{D_{aT}^n}
\end{equation} 

\noindent where \(B_a\) is the amount of business drawn to \(a\) from \(T\), \(P_a\) and \(P_T\) are the populations of the two settlements, and \(D_{aT}\) is the distance between them. Quantity \(N\) was chosen by Reilly in a somewhat \emph{ad hoc} fashion as 1, and he used empirical observations of shoppers to choose a value of \(n = 2\).

Zipf, on the other hand, wrote his hypothesis in mathematical form as:

\begin{equation}
\label{eq:zipf}
C^2 = \frac{P_1\cdot P_2}{D_{12}}
\end{equation} 

\noindent where \(C\) is the volume of goods exchanged between \(1\) and \(2\), \(P_1\) and \(P_2\) are the populations of the two settlements, and \(D_{12}\) is the distance between them.

After Carey, it is in Stewart's work on the principles of demographic gravitation that we find the strongest connection yet to Newton's law (Stewart, 1948). This may relate to academic backgrounds; where Stewart was a physicist while Ravenstein, Reilly, and Zipf were social scientists. Besides awareness of preceding research (he cites both Reilly and Zipf as predecessors in the analysis of human spatial interaction), Stewart appears to have been the first author to express his theorized relationships for human spatial interaction with a proportionality constant \(G\), as follows:

\begin{equation}
\label{eq:stewart-force}
F = G\frac{(\mu_1N_1)(\mu_2N_2)}{d_{12}^2} = G\frac{M_1\cdot M_2}{d_{12}^2} 
\end{equation} 

\noindent Where:

\begin{itemize}
\tightlist
\item
  \(F\) is the \emph{demographic force}
\item
  \(N_1\) and \(N_2\) are the numbers of people of in groups 1 and 2
\item
  \(\mu_1\) and \(\mu_2\) are so-called \emph{molecular weights}, the attractive weight of groups 1 and 2
\item
  \(M_1 = \mu_1N_1\) and \(M_2 = \mu_2N_2\) are the demographic masses at 1 and 2
\item
  \(d_{12}^2\) is the distance between \(1\) and \(2\)
\item
  And finally \(G\), a constant that Stewart ``left for future determination'' (1948, p. 34)
\end{itemize}

In addition to demographic force, Stewart defined a measure of the ``population potential'' of group \(2\) with respect to group \(1\). In other words, the potential number of people from location \(2\) that could visit location \(1\), as follows:

\begin{equation}
\label{eq:stewart-population-potential}
V_1 = G\frac{M_2}{d_{12}}
\end{equation} 

For a system with more than two population bodies, Stewart formulated the population potential at \(i\) as follows (after arbitrarily assuming that \(G=1\)):

\begin{equation}
\label{eq:stewart-population-potential-integral}
V_i = \int\frac{D}{r} ds
\end{equation} 

\noindent where \(D\) is the population density over an infinitesimal area \(ds\) and \(r\) is the distance to \(i\). In Equation \ref{eq:stewart-population-potential-integral}, \(D\cdot ds\) gives an infinitesimal count of the population, say \(dm\), and so, after discretizing space, Equation \ref{eq:stewart-population-potential-integral} can be rewritten as:

\begin{equation}
\label{eq:stewart-population-potential-sum}
V_i = \sum_j M_jd_{ij}^{-1}
\end{equation} 

Alerted readers will notice that Equation \ref{eq:stewart-population-potential-sum}, with some re-organization of terms, is formally equivalent to our modern definition of accessibility popularized by Hansen (1959) in the late 1950s.

Stewart's formulation of demographic force, developed in the context of what he called ``social physics'' (Stewart, 1947), was problematic. It had issues with inconsistent mathematical notation. More seriously though, Stewart's work was permeated by a view of humans as particles following physical laws, but tinted by unscientific and racist ideas. For instance, he assumed that the molecular weight \(\mu\) of the average American was one, but ``presumably\ldots much less than one\ldots.for an Australian aborigine'' \[p. 35\]. Stewart's ideas about ``social physics'' soon fell out of favour among social scientists, but not before influencing the nascent field of accessibility research, as detailed next.

\section{Hansen's gravity-based accessibility to today}\label{grav-to-today}

From Stewart (1948), we arrive to 1959 and Walter G. Hansen, whose work proved to be exceptionally influential in the accessibility literature (Hansen, 1959). In this seminal paper, Hansen defined accessibility as ``the potential of opportunities for interaction\ldots{} a generalization of the population-over-distance relationship or \emph{population potential} concept developed by Stewart (1948)'' (p.~73). As well as being a student of city and regional planning at the Massachusetts Institute of Technology, Hansen was also an engineer with the Bureau of Roads, and preoccupied with the power of transportation to shape land uses in a very practical sense. Hansen (1959) focused on Stewart (1948)'s \emph{population potential} (expressed in Equation \ref{eq:stewart-population-potential-sum}, and not on the other formulaic contributions and objectionable aspects of ``social physics''. Hansen (1959) recast Stewart's population potential to reflect accessibility, a model of human behaviour useful to capture regularities in mobility patterns. Hansen (1959) replaced \(M_j\) in Equation \ref{eq:stewart-population-potential-sum} with \emph{opportunities} to derive an \emph{opportunity potential}, or more accurately, a \emph{potential of opportunities for interaction} as follows:

\begin{equation}
\label{eq:accessibility}
S_{i} = \sum_j \frac{O_j }{d_{ij}^\beta}
\end{equation} 

A contemporary rewriting of Equation \ref{eq:accessibility} accounts for a variety of impedance functions beyond the inverse power \(d^{-\beta}\):

\begin{equation}
\label{eq:accessibility-general}
S_{i} = \sum_j O_j \cdot f(d_{ij})
\end{equation} 

\(S_{i}\) in Equation \ref{eq:accessibility} is a measure of the accessibility from site \(i\). This is a function of \(O_j\) (the mass of opportunities at \(j\)), \(d_{ij}\) (the cost, e.g., distance or travel time, incurred to reach \(j\) from \(i\)), and \(\beta\) (a parameter that modulates the friction of cost). Today, Hansen is frequently cited as the father of modern accessibility analysis (e.g., Reggiani \& Martín, 2011), and Hansen-type accessibility is commonly referred to as the gravity-based accessibility measure.

However, Hansen's use of Stewart's \emph{population potential} measure included one crucial omission that afflicts the literature to this day. The omission is that between Stewart (1948) and Hansen (1959), the proportionality constant \(G\) in Equation \ref{eq:stewart-population-potential} vanished. This constant was not explicitly addressed in Hansen (1959) and accessibility research continues to evolve without it. Since Hansen (1959), accessibility analysis has been widely used in numerous disciplines but, to our knowledge, the proportionality constant has remained forgotten, with no notable developments to explicitly acknowledge or determine it.

The omission of this constant generates a fundamental problem for the measurement unit of accessibility estimates, which undermines the interpretation, communication and comparability of accessibility analysis. Those reading Hansen (1959) must recall that Stewart (1948) had set the proportionality constant \(G\) to 1, with a note that ``\(G\) \[was\] left for future determination: a suitable choice of other units can reduce it to unity'' {[}p.~34{]}. In practice, the persistent omission of the constant in accessibility analyses means that \(G\) continues to be implicitly set to \(1\), even when the fundamental relationship in accessibility is proportionality (e.g., \(S_{i} \propto \sum_j g(O_j)f(d_{ij})\)) and not equality (for instance, see the formula for accessibility at the top of Figure 1 in Wu \& Levinson, 2020). The direct consequence is that without a proportionality constant, the units of \(S_i\) remain unclear: the unit of ``potential of opportunity for interaction'' is left free to change as \(\beta\) is calibrated. For example, if \(c_{ij}\) is distance in meters, it will be number of opportunities per \(m^{\beta}\) when \(f(c_{ij}) = d^{-\beta}\) but number of opportunities per \(e^{-\beta\cdot m}\) when \(f(c_{ij}) = e^{-\beta\cdot m}\). This undermines the comparability of accessibility metrics with different decay functions, and renders their results difficult to understand and communicate. The Hansen-style accessibility estimates found in the literature, therefore, are better thought of as ordinal measures of potential that can only be interpreted in terms of higher and lower accessibility, but which has not palpable meaning (E. J. Miller, 2018).

\section{Wilson's family of spatial interaction models}\label{wilsons-family-of-spatial-interaction-models}

On the other side of the Atlantic, Alan G. Wilson was developing related, yet parallel work. In his groundbreaking study (Wilson, 1971), Wilson defined a general spatial interaction model as follows. While accessibility was characterised as an associated concept of `potential', the primary focus was on modelling observed spatial interaction:

\begin{equation}
\label{eq:phys-gravity-model}
T_{ij} = k W_i^{(1)} W_j^{(2)} f(c_{ij})
\end{equation} 

The model in Equation \ref{eq:phys-gravity-model} posits a quantity \(T_{ij}\) that represents a value in a matrix of flows of size \(n \times m\), that is, between \(i = 1,\cdots, n\) origins and \(j = 1,\cdots, m\) destinations. The quantities \(W_i^{(1)}\) and \(W_j^{(2)}\) are proxies for the masses at \(i=1,\cdots,n\) origins and \(j=1,\cdots,m\) destinations. The super-indices \((1)\) and \((2)\) are meant to indicate that these masses could be different things, i.e., \(W_i^{(1)}\) could be populations, and \(W_j^{(2)}\) hectares of park space. Finally, \(f(c_{ij})\) is some function of travel cost \(c_{ij}\) which reflects travel impedance. In this way, \(T_{ij}\) explicitly measures \emph{interaction} in the unit of trips, and the role of \(k\) is to ensure that the system-wide sum of \(T_{ij}\) represents the total flows in the data. In other words, \(k\) is a scale parameter that makes the overall amount of flows identical to the magnitude of the phenomenon being modeled. In other words, it balances the units, in a conceptually similar sense as the gravitational constant in Newton's Law of Universal Gravitation.

Traditionally, the development of the spatial interaction model put an emphasis on the interpretability of the results (Kirby, 1970; Wilson, 1967, 1971). But instead of relying on the heuristic of Newtonian gravity (e.g., some interaction between a mass at \(i\) and a mass at \(j\) separated by some distance), Wilson's approach was to maximise the entropy of the system. Entropy maximisation in this case achieves stable results as a statistical average that represents the population. The approach works by assuming undifferentiated individual interactions, and assessing their probabilities of making a particular journey. The result of Equation \ref{eq:phys-gravity-model} then is a statistical average (Senior, 1979; Wilson, 1971).

To ensure that \(T_{ij}\) in Equation \ref{eq:phys-gravity-model} is in the unit of trips (our unit of origin-destination spatial interaction), additional knowledge about the system is required. At the very least, this framework assumes that the total number of trips in the system \(T\) is known, and therefore:

\begin{equation}
\label{eq:constraint0-gravitymodel}
\sum_i\sum_j T_{ij} = T
\end{equation} 

Additional information can be introduced. For example, when information is available about the total number of trips produced by each origin, \(W_i^{(1)}\) is represented as \(O_i\) and the following constraint can be used:

\begin{equation}
\label{eq:constraint1-gravitymodel}
\sum_j T_{ij} = O_i
\end{equation} 

Alternatively, if there is information available about the total number of trips attracted by each destination, \(W_j^{(2)}\) is represented as \(D_j\) and the following constraint can be used:

\begin{equation}
\label{eq:constraint2-gravitymodel}
\sum_i T_{ij} = D_j
\end{equation} 

It is also possible to have information about both \(O_i\) and \(D_j\), in which case both constraints could be imposed on the model at once.

Using information about the system that satisfy these constraints fully, partially or not at all, a family of spatial interaction models can be derived based on Equation \ref{eq:phys-gravity-model}. \(K\) is specified depending on the applied constraint(s). In the framework introduced in Wilson (1971), three constrained versions are specified: the first being a case where to results only match the total volume of interaction, the second being a singly constrained case, and the third a doubly constrained case.

In the first, Equation \ref{eq:constraint1-gravitymodel} and Equation \ref{eq:constraint2-gravitymodel} do not hold. In practical terms, this means that the total number of trips predicted by the model must be equal to sum of all flows from origins i to destinations j. The balancing constant \(K\) in this case is (see Cliff, Martin, \& Ord, 1974; A. S. Fotheringham, 1984):

\begin{equation}
\label{eq:total-flow-balancing-factor}
K=\frac{T}{\sum_i\sum_j T_{ij}}
\end{equation} 

In the second case, only one of Equation \ref{eq:constraint1-gravitymodel} or Equation \ref{eq:constraint2-gravitymodel} hold. The resulting models are, in Wilson's terms, singly constrained. When only Equation \ref{eq:constraint1-gravitymodel} holds, entropy maximisation leads to the following production-constrained model:

\begin{equation}
\label{eq:production-constrained-gravitymodel}
T_{ij} = A_i O_i W_j^{(2)} f(c_{ij})
\end{equation} 

Notice how, in this model, the proxy for the mass at the origin \(W_i^{(1)}\) is replaced with \(O_i\), representing what we know about the system, the spatial interaction outbound flow, i.e., outbound trips produced at \(i\). Also, there is no longer a single system-wide proportionality constant, but rather a set of proportionality constants (i.e., balancing factors) specific to origins. For this model, the balancing factors ensure that Equation \ref{eq:constraint1-gravitymodel} is satisfied, meaning that the sum of predicted flows from one origin going to all destinations must equal the known mass at that origin \(O_i\) i.e., the total number of outbound trips. Satisfying this constraint also implicitly fulfills the total constraint (Equation \ref{eq:constraint0-gravitymodel}), since the sum of \(O_i\) values across all origins equals the total number of trips. This model is useful when trips ends are unknown but the number of trips originating from each location is known and the total of these trips represents all trips in the system. The balancing factors for the production-constrained model are solved for each origin \(A_i\), and according to Wilson are:

\begin{equation}
\label{eq:production-constrained-balancing-factor}
A_i = \frac{1}{\sum_j W_j^{(2)} f(c_{ij})}
\end{equation} 

The attraction-constrained model is similar to the production-constrained model as it is also singly constrained but from the perspective of the mass at the destination. From the attraction-constrained model, the proxy for the mass at the destination \(W_j^{(2)}\) is now replaced with \(D_j\), representing the spatial interaction inbound flow, i.e., trips attracted at the destination and takes the following form:

\begin{equation}
\label{eq:attraction-constrained-gravitymodel}
T_{ij} = B_j D_j W_i^{(1)} f(c_{ij})
\end{equation} 

In this model (Equation \ref{eq:attraction-constrained-gravitymodel}), the balancing factors ensure that Equation \ref{eq:constraint2-gravitymodel} is satisfied (hence the total constraint Equation \ref{eq:constraint0-gravitymodel} is as well), meaning that the sum of predicted flows going to one destination from all origins must equal the known mass of that destination \(D_j\) i.e., the total number of inbound trips to \(j\). This should hold for all destinations. As before, destination-specific proportionality constants (i.e., balancing factors) \(B_j\) were derived by Wilson as:

\begin{equation}
\label{eq:attraction-constrained-balancing-factor}
B_j = \frac{1}{\sum_i W_i^{(1)} f(c_{ij})}
\end{equation} 

The third case in the family of spatial interaction models is the production-attraction constrained model. In this case, both Equation \ref{eq:constraint1-gravitymodel} and Equation \ref{eq:constraint2-gravitymodel} hold simultaneously. These constraints ensure that the sum of predicted flows from one origin to all destination, and the predicted flows going to one destination from all origins must equal the known mass of the origin \(O_i\) and of the destination \(D_j\). This should hold for all origins and destinations. The resulting model is, in Wilson's terms, doubly constrained, and takes the following form:

\begin{equation}
\label{eq:doubly-constrained-gravitymodel}
T_{ij} = A_i B_j O_i D_j f(c_{ij})
\end{equation} 

In this model, both proxies for the masses are replaced with the known masses, that is, the trips produced by the origin and the trips attracted by the destination. There are now two sets of mutually dependent proportionality constants:

\begin{equation}
\label{eq:doubly-constrained-balancing-factors}
\begin{array}{l}
A_i = \frac{1}{\sum_j B_j D_j f(c_{ij})}\\
B_j = \frac{1}{\sum_i A_i O_i f(c_{ij})}
\end{array}
\end{equation} 

Derivation of these models is demonstrated in detail elsewhere (e.g., Ortúzar \& Willumsen, 2011; Wilson, 1967). It is worth noting, however, that although Wilson's approach is built on a different conceptual foundation than the old reference to Newtonian gravity, the work succeeded at identifying the steps from proportionality to equality to yield variations of proportionality constants, including the one that eluded Stewart (1948) and that has been overlooked in almost all subsequent accessibility research. Why was this key element of spatial interaction models potentially ignored in accessibility research? In the next section we aim to address this question.

\section{Accessibility and spatial interaction modelling: two divergent research streams}\label{accessibility-and-spatial-interaction-modelling-two-divergent-research-streams}

The work of Hansen (1959) and Wilson (1971) responded to important developments, in particular a need ``to meet the dictates and needs of public policy for strategic land use and transportation planning'' (Michael Batty, 1994). These dictates and needs were far from trivial. In the United States alone, the Federal-Aid Highway Act of 1956 authorized the creation of the U.S. Interstate Highway System, with a budget that ultimately exceeded one hundred billion dollars (equivalent to over \$600 billion in 2023) (MDOT, 2007; Weiner, 2016). Spatial interaction modelling was incorporated into institutional modelling practices meant to ``predict and provide'', i.e., predict travel demand and supply transportation infrastructure (Kovatch, Zames, et al., 1971; Weiner, 2016). Accessibility, at the time, did not quite have that power, as it did not quantify trips, but rather something somewhat more elusive: it predicts the less tangible ``potential'' for spatial interaction with opportunities. In this way, where spatial interaction modelling became a key element of transportation planning practice, accessibility remained a somewhat more academic pursuit, and the two streams of literature only rarely connected.

\begin{figure}
\includegraphics[width=0.7\linewidth]{data/figures/chp1-docs_per_year_plot} \caption{Historical pattern of publication: documents per year.}\label{fig:chp1-fig-docs-per-year}
\end{figure}

To illustrate this point, we conducted a bibliographic analysis of the literature that cites Hansen (1959), Wilson (1971), or both. We retrieved all relevant documents using the Web of Science ``Cited References'' functionality, and the digital object identifiers of Hansen (1959) and Wilson (1971). As a result of this search, we identified 1,788 documents that cite Hansen (1959), only 258 documents that cite Wilson (1971), and 76 that cite both. The earliest document in this corpus dates to 1976 and the most recent is from 2025. The number of documents per year appears in Figure \ref{fig:chp1-fig-docs-per-year}, where we see the frequency of documents over a span of almost fifty years. In particular, we notice the remarkable growth in the number of papers that cite Hansen (1959) compared to those that cite Wilson (1971) since the year 2000.

As noted, literature that cites both Wilson (1971) and Hansen (1959) are sparse (only 3.6\% of the corpus, visualised in pink in Figure \ref{fig:chp1-fig-docs-per-year}). After reading the works, we can also discern that they too are divergent, with one stream focused on developing accessibility (i.e., \emph{potential for} spatial interaction) and another on spatial interaction. The focuses of these divergent streams contribute this paper's broader hypothesis that the concepts of accessibility and spatial interaction have remained largely disconnected, and at times, improperly conflated.

On one hand, the stream of literature focused on spatial interaction models inspired by Wilson (1971) and which cite both Hansen (1959) and Wilson (1971), tends to contribute to understanding how accessibility is interpreted and incorporated in spatial interaction models. These works treat separate spatial interaction and accessibility as a separate but related phenomenon. Specifically, some early works interpret the spatial interaction model's balancing factors (Equation \ref{eq:production-constrained-balancing-factor} or Equation \ref{eq:doubly-constrained-balancing-factors} as the inverse of Hansen (1959)`s model (A. Stewart Fotheringham, 1981; A. S. Fotheringham, 1985; B. Harris \& Wilson, 1978; G. Leonardi, 1978), recognizing it as a ``common sense'' approach (Morris, Dumble, \& Wigan, 1979, p. 99) to including accessibility in the spatial interaction model, though further exploration of its relationship is warranted (M. Batty \& March, 1976). Some authors have explored this relationship, for instance as in A. S. Fotheringham (1985) who demonstrates how the spatial interaction model may insufficiently explain spatial patterns, and suggests that explicitly defining destinations' accessibility as a variable within the model may remedy the issue (e.g., the \emph{competition destination} model). Other works used both Hansen (1959) and Wilson (1971)'s framework in conjunction, such as in defining location-allocation problems in operations research (Beaumont, 1981; G. Leonardi, 1978), estimating trip flows (or some other spatial interaction flows) alongside accessibility (e.g., Clarke, Eyre, \& Guy, 2002; Grengs, 2004; Türk, 2019), or considering accessibility within spatial interaction models, in line with A. S. Fotheringham (1985)'s demonstration (e.g., Beckers et al., 2022). Other works departed from Hansen (1959)'s definition and aligned with spatial interaction in different ways, such as using micro-economic consumer behaviour concepts to express potential for spatial interaction (Giorgio Leonardi \& Tadei, 1984; Morris et al., 1979).

On the other hand, there is another subset of literature that cite both Hansen (1959) and Wilson (1971) that is accessibility-focused. We categorise their citation of Wilson (1971) for three general reasons. Firstly, a group of these works cite Wilson (1971) as attribution for using context-dependent travel cost functions (Ashiru, Polak, \& Noland, 2003; Caschili, De Montis, \& Trogu, 2015; Chia \& Lee, 2020; Grengs, 2015; S. L. Handy \& Niemeier, 1997; Kharel, Sharifiasl, \& Pan, 2024; Kwan, 1998; Margarida Condeço Melhorado, Demirel, Kompil, Navajas, \& Christidis, 2016; Pan, 2013; Pan, Jin, \& Liu, 2020; Rau \& Vega, 2012; Roblot, Boisjoly, Francesco, \& Martin, 2021; Sharifiasl, Kharel, \& Pan, 2023; Q. Shen, 1998a; e.g., J. W. Weibull, 1980). These works do not necessarily comment on spatial interaction explicitly. Secondly, another group of this accessibility-focused ``both'' citing literature \emph{does} associate spatial interaction as defined in Wilson (1971) with accessibility's potential for spatial interaction more explicitly (Giuliano, Gordon, Pan, \& Park, 2010; Grengs, 2010a, 2012; Grengs, Levine, Shen, \& Shen, 2010; He, Li, Yu, Liu, \& Huang, 2017; Levine, Grengs, Shen, \& Shen, 2012; Levinson \& Huang, 2012; X. Liu \& Zhou, 2015; e.g., H. J. Miller, 1999; Naqavi, Sundberg, Västberg, Karlström, \& Hugosson, 2023; Ng, Roper, Lee, \& Pettit, 2022; Suel et al., 2024; Tong, Zhou, \& Miller, 2015; Wu \& Levinson, 2020). We agree accessibility and spatial interaction are related topics: accessibility is an expression of its \emph{potential} and the Wilson (1971) paper briefly touches on the concept. However, in some of this literature, Hansen (1959) and Wilson (1971) are co-cited as both being `gravity models' (Chia \& Lee, 2020; Dai, Wan, \& Gai, 2017; e.g., S. Liu \& Zhu, 2004; Y. Shen, 2019), perhaps revealing the murkiness of the distinction between spatial interaction and the \emph{potential for} spatial interaction in the literature. Thirdly, there is a group of accessibility-focused works that interpret Hansen (1959)'s model as the singly- or doubly- constrained spatial interaction model's inverse balancing factor (e.g., R. W. Vickerman, 1974). This group cities the earlier spatial interaction works that make this connection and is especially prominent in the investigation of competitive accessibility topics (Albacete, Olaru, Paül, \& Biermann, 2017; Allen \& Farber, 2020; Alonso, Beamonte, Gargallo, \& Salvador, 2014; Chen \& Silva, 2013; Curtis \& Scheurer, 2010; El-Geneidy \& Levinson, 2011; Karst T. Geurs, van Wee, \& Rietveld, 2006; e.g., Karst \& Van Eck, 2003; Levinson \& Wu, 2020; Manaugh \& El-Geneidy, 2012; Marwal \& Silva, 2022; Mayaud, Tran, Pereira, \& Nuttall, 2019; Sahebgharani, Mohammadi, \& Haghshenas, 2019; Su \& Goulias, 2023; Willigers, Floor, \& van Wee, 2007). As outlined in preceding sections, we argue interpreting the singly- or doubly- constrained spatial interaction model's balancing factor as accessibility yields output values that are similarly plagued by interpretability issues.

Lastly, as an extension of the third reason within this group, only the works of Soukhov et al. (2023) and Soukhov et al. (2024) use Wilson (1971)'s balancing factors as a method for maintaining constraints on opportunities within the context of competitive accessibility. These works introduce the balancing factors as a mechanisms to ensure that opportunities at each destination are proportionally allocated to each zone (based on the proportion of population seeking opportunities and the relative travel impedance). This is to ensure that each zonal accessibility value is the sum of this proportional allocation from each destination, and that all zonal values ultimately sum to the total number of opportunities in the region. However, these balancing factors were deduced intuitively. These works do not explicitly state that the mathematical formulation of the equations are effectively equivalent to Wilson's singly constrained model (derived from entropy maximization); this equivalence was only discovered in hindsight. These two works also do not discuss other constrained cases, which will be addressed in the next chapter.

\section{Chapter conclusion}\label{chapter-conclusion}

In this chapter, the Newtonian roots of human spatial interaction modelling was traced from Ravenstein (1885), to Hansen (1959)`s application of Stewart (1947)'s 'population potential' (Equation \ref{eq:stewart-population-potential}) and then to Wilson (1971)'s family of spatial interaction models. We detailed how the balancing factors from Wilson's family of spatial interaction models are formulated to reflect known system constraints (i.e., the total constraint, and production and attraction constraints in Equations \ref{eq:constraint0-gravitymodel}, \ref{eq:constraint1-gravitymodel} and \ref{eq:constraint2-gravitymodel}.

Literature that cite both seminal works is then reviewed (Hansen (1959) and Wilson (1971)). It is found that these works are divergent: some are spatial interaction focused, citing Hansen (1959) as the father of the accessibility concept and uses accessibility as a variable within spatial interaction modelling. Another set of work is accessibility focused, and what we may now know as the modern accessibility literature beginning in the early 2000s. These accessibility focused works cite Wilson (1971) for various general reasons, but none (other than Soukhov et al. (2023) and Soukhov et al. (2024)) cite the logic of the balancing factors. Beginning with this logic in the following chapter, balancing factors are introduced akin to those used in Wilson's family of spatial interaction models. They are formulated to incorporate system-wide or zonal constraints (i.e., knowns) to accessibility. In this way, the following chapter introduces a `family of accessibility measures'. Each member and variant of the family is mathematically formulated and solved using a simple toy example.

\chapter{CHP 2 - A family of accessibility measures}\label{chp-2---a-family-of-accessibility-measures}

\section{Chapter overview}\label{chapter-overview-1}

This chapter consists of text from two journal articles: first, from the second half of the \emph{``Family of accessibility measures derived from spatial interaction principles''} authored by Anastasia Soukhov, Rafael H. M. Pereira, Christopher D. Higgins, and Antonio Páez (submitted to \emph{PLOS ONE} in May 2025) that details the formulation of the family of accessibility measures and solved simple numeric examples, and second, from the manuscript detailing the multimodal extension of spatial availability (Soukhov et al., 2024).

The objective of this chapter is to detail the formulation of the family of accessibility measures which are based on spatial interaction principles. These formulations are used in the next chapters of this dissertation.
This chapter first details four cases of the family: unconstrained, total constrained, singly constrained and doubly constrained accessibility measures. Each case has two variants, either accessible `opportunities' (i.e., the way Hansen (1959) is understood) or accessible `populations' (i.e., market potential, the way Reilly (1929) can be understood). An empirical example will be solved using the total constrained and singly constrained cases, so for these cases their multimodal extensions are defined. Lastly, a demonstration of a solved simple toy example is included following the formulation of each case and all the cases are summarised in the conclusion.

This chapter offers two general contributions:

\begin{itemize}
\tightlist
\item
  The formulaic specification of four cases of the family of accessibility measures, along with different variants and multimodal extensions: specifically, the `unconstrained' measure (i.e., Hansen-type measure), the `total constrained' measure (i.e., a constrained version of the Hansen-type measure), the `singly constrained' measure (i.e., related to the popular two step floating catchment approach (2SFCA)), and the `doubly constrained' measure representing realized interactions or `access', effectively equal to the doubly constrained spatial interaction model in formulation.
\item
  Summary of the interpretability advantages of the family using a worked simple toy example, as these constrained accessibility measures yield values in units of the number of potential ``opportunities for spatial interaction'' or ``population for spatial interaction'' for each zone and zonal flow.
\end{itemize}

\section{Outline of the family of accessibility measures}\label{outline-of-the-family-of-accessibility-measures}

As argued in the preceding chapter, the streams of research on accessibility and spatial interaction modelling have evolved as largely separate streams with little contact since Hansen (1959) and Wilson (1971). This may explain why the constraints and associated balancing factors of spatial interaction models did not cross over to accessibility analysis. This is intriguing since Wilson made an effort to connect developments in spatial interaction modelling to accessibility, noting for instance, the denominator of the proportionality constants specific to origins (Equation \ref{eq:production-constrained-balancing-factor}) is the inverse of balancing factor \(A_i\) (Wilson, 1971, p. 10):

\begin{equation}
\label{eq:Ai-as-accessibility}
S_i = \frac{1}{A_i} = \sum_j W_j^{(2)} f(c_{ij})
\end{equation} 

Understanding \(S_i\) as the inverse of Wilson's \(A_i\) does not uncover any new meaning for \(S_i\) itself. Indeed, mathematically it is true, \(A_i\)'s role in Wilson's general model \(T_{ij} = k W_i^{(1)} W_j^{(2)} f(c_{ij})\) is that of a balancing factor \(k\) i.e., keeping units balanced and proportionality based on constraints. Understanding \(S_i\) itself as a balancing factor is not wholly helpful as Hansen and (Stewart before him) defined accessibility as a partial sum of the demographic force \(F\) or the system-wide population potential of opportunities for spatial interaction (i.e., \(V_i = \sum_j \frac{M_j}{d_{ij}}\) after losing \(G\)).

Hence, we propose stepping back to introduce a revised definition of accessibility: the \emph{constrained} potential for spatial interaction. Once we bring back Wilson's proportionality constant \(k\) into the picture, we can define the \emph{potential for spatial interaction} between two locations \(i\) and \(j\) is as follows:

\begin{equation}
\label{eq:access-01}
V_{ij} = k W_j^{(2)} f(c_{ij})
\end{equation} 

\noindent where \(V_{ij}\) is the potential for interaction from \(i\) to \(j\). The accessibility from origin \(i\) can then be summarised as as a partial sum of the potential at \(i\):

\begin{equation}
\label{eq:accesssibility-01}
V_{i} = k \sum_j W_j^{(2)} f(c_{ij})
\end{equation} 

Similar to Equation \ref{eq:phys-gravity-model}, \(W_j^{(2)}\) above is the mass at the destination and the sub-indices are for \(i = 1,\cdots, n\) origins and \(j = 1,\cdots, m\) destinations.

The market potential variant can also be generally defined, which is effectively the transpose of \(i\) and \(j\) in Equation \ref{eq:access-01} and Equation \ref{eq:accesssibility-01} as follows:

\begin{equation}
\label{eq:market-01}
M_{ji} = k W_i^{(1)} f(c_{ij})
\end{equation} 

\begin{equation}
\label{eq:market-potential-01}
M_{j} = k \sum_j W_i^{(1)} f(c_{ji})
\end{equation} 

Similar to Equation \ref{eq:phys-gravity-model}, \(W_j^{(2)}\) above is the mass at the destination and the sub-indices are for \(i = 1,\cdots, n\) origins and \(j = 1,\cdots, m\) destinations.

To detail the anatomy of \(V_{ij}\) and \(M_{ji}\) along with the partial sums of \(V_i\) and \(M_j\), Figure \ref{fig:chp2-fig-analytical-device-conc-accessibility} illustrates the accessibility analytical framework we propose using a simple 3 zone system. Each measure's most disaggregate value is \(X_{ij}\), potential for spatial interaction from \(i\) to \(j\). The \(X\) is a stand in for the \(ij\) values of all the cases, their variants and multimodal extensions that will be described (i.e., \(V_{ij}^0, V_{ij}^{m0}, M_{ji}^0, M_{ji}^{m0}, V_{ij}^T, V_{ij}^{mT}, M_{ji}^T, M_{ji}^{mT}, V_{ij}^S, V_{ij}^{mS}, M_{ji}^S, M_{ji}^{mS}\) and \(V_{ij}^D, M_{ji}^D\)). The single marginals represent the origin-side and destination-side weights of the zones. The total marginal represent the sum of a single marginal.

\begin{figure}
\includegraphics[width=0.7\linewidth]{data/figures/chp2-access-analytical-device} \caption{The family of accessibility measures analytical framework: labelling and associating ij flows, zonal weights, the single marginals, and the total marginal.}\label{fig:chp2-fig-analytical-device-conc-accessibility}
\end{figure}

As an additional overview, we define four distinct members of the constrained accessibility measure family, all delineated based on their constant \(k\), which takes the form of either balancing factors \(K^T\), \(B_j\), \(A_i\) (and their multimodal \(m\) versions) depending on the indicator:

\begin{enumerate}
\def\labelenumi{\arabic{enumi}.}
\item
  \textbf{The unconstrained case} with variants \(V_i^0\) and \(M_j^0\) and multimodal extension \(V_i^{m0}\) and \(M_j^{m0}\). The first variant \(V_i^0\) is equivalent to Hansen (1959)`s formulation and \(M_j^0\) is equivalent to Reilly (1929)'s market potential formulation. Both these variants neglect including a balancing factor, so units of zonal values are in units of 'opportunities-by-travel-impedance-value' and `population-by-travel-impedance-value'. In this case, no constraints are introduced to ensure that values of marginals are preserved.
\item
  \textbf{The total constrained case} with variant \(V_i^T\) and \(M_j^T\) along with multimodal extensions \(V_i^{mT}\) and \(M_j^{mT}\). The first variant \(V_i^T\)--the \emph{total constrained accessible opportunities measure}--resembles Hansen (1959)`s formulation but with an additional regional balancing factor \(K^T\) term, defined as the ratio of the total number of opportunities in the region to the total sum of unconstrained accessibility values in the region. In this way, \(K^T\) ensures that each zonal accessibility value is a proportion of the total opportunities in the region (i.e., the total marginal in Figure \ref{fig:chp2-fig-analytical-device-conc-accessibility}), requires no information about the population seeking opportunities, and is in units of 'opportunities' accessible. The second variant \(M_j^T\) is the transpose of \(i\) to \(j\) of the first variant, effectively a constrained version of market potential and is referred to as \emph{total constrained accessible population measure}. In this variant, each zonal accessibility value is a proportion of the total population in the region (maintained by \(\hat K^T\)), requires no information about the opportunities that are sought, and each zonal value is in units of `population' accessible.
\end{enumerate}

\begin{itemize}
\tightlist
\item
  When conceptualising \emph{potential} for spatial interaction--whether with opportunities or population--the total constrained case reflects the lowest level of restriction and hence maximum potential. The balancing factors \(K^T\) and \(\hat K^T\) only ensure that \(V_{ij}^T\) and \(M_{ji}^T\) values end up matching the \emph{total sum} of one of the single marginals, not the individual marginals themselves. For example, \(K^T\) does not guarantee that all the opportunities accessibility at \(V_{i=1}^T\) reflect a proportional allocation of the destination mass (i.e., \emph{number} of opportunities) at \(j=1, 2, 3\). In some cases, an allocation of opportunities from a destination \(j\) to \(i=1\) could \emph{exceed} the number of opportunities at that \(j\) (i.e., meaning that destination \(j\) is very attractive and reachable for an \(i\), relative other flows in the region). However, what \(K^T\) \emph{does} ensure is that each \(V_{ij}^T\) does not exceed the overall number of opportunities in the region--the total marginal. In other words, \(V_{i=1}^T\) expresses the number of opportunities that origin \(i=1\) could potentially interact with, as drawn from the entire regional opportunity mass, rather than constrained by individual destination totals. In this way, the \emph{total constrained} case reflects the maximum amount of \emph{potential} while still maintaining interpretable units.
\end{itemize}

\begin{enumerate}
\def\labelenumi{\arabic{enumi}.}
\setcounter{enumi}{2}
\tightlist
\item
  \textbf{The singly constrained case}, with two variants \(V_i^S\) (opportunity-constrained) and \(M_j^S\) (population-constrained) along with their multimodal extensions \(V_i^{mS}\) and \(M_j^{mS}\). The first variant is mathematically equivalent to the spatial availability measure (Soukhov et al., 2023), and this variant's per capita form is equivalent to the popular 2SFCA measure (Luo \& Wang, 2003a; Q. Shen, 1998b). Both variants can also be defined using either Hansen (1959)'s or the market potential formulation but with balancing factor \(B_j\) for \(V_i^S\) or \(A_i\) for \(M_j^S\). These balancing factors ensure that the total marginal (green box in Figure \ref{fig:chp2-fig-analytical-device-conc-accessibility}) is maintained \emph{as well as} the values at one of the single marginals (destination-mass marginal for opportunity-constrained and origin-mass marginal for population constrained in Figure \ref{fig:chp2-fig-analytical-device-conc-accessibility}). In this way, the singly constrained case reflects a medium amount of potential, restricted by either values fitting the single destination-mass marginal or the single origin-mass marginal.
\end{enumerate}

\begin{itemize}
\tightlist
\item
  The first variant \(V_i^S\) includes a set of destination-side balancing factors \(B_j\) which ensure opportunities (or destination masses) are allocated to each origin \(i\) based on the population (i.e, origin mass) of the origin \(i\) and travel impedance from that \(j\) to all \(i\)s. Hence, \(B_j\) ensures that each \(V_i^S\) is the product sum of a \emph{proportional share} of opportunities allocated from each destination \(j\) implicitly also being a share of the total opportunities in the region. Similar to the total constrained case, each \(V_i^S\) and \(V_ij^S\) accessibility value is expressed in units of `opportunities' accessible. But unlike the total constrained case, the singly constrained case explicitly incorporates population by allocating a balanced share of a destination's total opportunities to populations at reachable origins.
\item
  The second variant \(M_j^S\) includes a set of opportunity-side balancing factors \(A_i\) which ensure population (or origin masses) are allocated to each destination \(j\) based on the destination mass of the destination \(j\) and all possible travel impedance from that \(i\) to all \(j\)s. \(A_i\) ensures similar, but transposed, so constraints are satisfied. Specifically, each \(M_j^S\) represents both a share of the total regional population and the sum of balanced proportions of population from all origins, allocated to each destination based on the opportunities sought and travel impedance. Each zonal value is expressed in units of `population' accessible.
\end{itemize}

\begin{enumerate}
\def\labelenumi{\arabic{enumi}.}
\setcounter{enumi}{3}
\tightlist
\item
  \textbf{The doubly constrained case}. It is constrained simultaneously by population (origin masses) and opportunities (destination masses), so each \(V_{ij}^D\) (equivalent to \(M_{ji}^D\)) is expressed in units of `population-opportunity capacity' that is accessible between \(i\) to \(j\). This case requires the number of opportunities and population to match, e.g., the analyst must know the matching spatial interaction capacity of the population (demand) and opportunities (supply). By using this one-to-one matching data and the double constraints (i.e., \(A_i\) and \(B_j\) at once, ensuring both the double constraint maintains both marginals in Figure \ref{fig:chp2-fig-analytical-device-conc-accessibility}), this case restricts the \emph{potential} to spatially interact completely. In other words, the doubly constrained accessibility values reflect the number of predicted interactions between the opportunities and population, effectively, this case is equivalent formulaically as Wilson's doubly constrained spatial interaction model (i.e., \emph{attraction-production constrained}). It can be understood as predicting a value of `access', and not accessibility.
\end{enumerate}

And as a summary: each member of the family of accessibility measure is named, explained in plain language, alongside their balancing factor(s), proportional allocation factor(s), and mathematical equation and value interpretations in Table \ref{tab:summary-family-access-measures-table}.

{\tiny
\begin{longtable}{|p{1.2cm}|p{2cm}|p{3.8cm}|p{3.7cm}|p{2cm}|}
\caption{Summary of family of accessibility measure members, their definitions and interpretations} \label{tab:summary-family-access-measures-table} \\
\hline
\textbf{ } & \textbf{Unconstrained ($V_i^0$, $M_j^0$)} & \textbf{Total Constrained ($V_i^T$, $V_i^{mT}$, $M_j^T$, $M_j^{mT}$)} & \textbf{Singly Constrained ($V_i^S$, $V_i^{mS}$, $M_j^S$, $M_j^{mS}$)} & \textbf{Doubly Constrained ($V_{ij}^D$, $M_{ij}^D$)} \\
\hline
\endfirsthead

\textbf{Measure Equation} &
$V_i^0 = \sum_j W^{(2)}_j f(c_{ij})$;

$M_j^0 = \sum_i W^{(1)}_i f(c_{ij})$ 

&
$V^T_i = \sum_j  \kappa_{ij}^T W^{(2)}_j$;

$M_j^T =\sum_i \hat{\kappa}_{ji}^T  W^{(1)}_i$

Multimodal:

$V^{mT}_i = \sum_j \kappa_{ij}^{mT} D_j$;

$M_{mj}^T = \sum_i \hat{\kappa}_{ji}^{mT} O_i$ 

&
$V^S_i = \sum_j \kappa^S_{ij} D_j$;

$M_j^S = \sum_i \hat \kappa^S_{ji}  O_i$

Multimodal:

$V^{mS}_i = \sum_j \kappa_{ij}^{mS} W^{m(2)}_j$;

$M^{mS}_j = \sum_i \hat{\kappa}_{ji}^{mS} W^{m(1)}_i$ 

&
$V_{ij}^D = A_i B_j O_i D_j f(c_{ij})$ \\
\hline

\textbf{Constraint Explanation and Balancing Factor} &
No constraints; marginals not equal to any regional or zonal knowns.

Multimodal extension, where other modes $m$ are a function of $V_i^0$ and $M_j^0$ is not specified -- however, a specific $V_i^0$ and $M_j^0$ value for a mode can be expressed, and is represented as $V_i^{m0}$ and $M_j^{m0}$ 
&
Balancing factors $K^T$ and $\hat{K}^T$ ensures the sum of $ij$ values matches the total marginal, where:

$K^T = \frac{D}{\sum_i V_i^0}$;

$\hat{K}^T = \frac{O}{\sum_j M_j^0}$ 

Multimodal balancing factors ensuring the $ij$ matrix for each $m$ is a proportion of the marginal, where: 

$K^{mT} = \frac{D}{\sum_m\sum_i V_i^{m0}}$;

$\hat{K}^{mT} = \frac{O}{\sum_m\sum_j M_j^{m0}}$ 
&
Single balancing factor $B_j$ (for $V_i^S$) that ensures the destination-mass marginal is constrained, and $A_i$ (for $M_j^S$) ensures the origin-mass marginal is constrained:

$B_j = \frac{1}{\sum_i W_i^{(1)} f(c_{ij})}$;

$A_i = \frac{1}{\sum_j W_j^{(2)} f(c_{ij})}$ 

Multimodal extension:

$B_j^m = \frac{1}{\sum_m\sum_i W_i^{m(1)} f^m(c_{ij}^m)}$;

$A_i^m = \frac{1}{\sum_m\sum_j W_j^{(2)} f^m(c_{ij})^m}$ 
&
Values reflect both single marginals simultaneously, maintained via $A_i$ and $B_j$. 

Multimodal extension, where other modes $m$ are a function of $V_i^0$ and $M_j^0$ is not specified in this work -- however, balancing factors that are multimodal could be solved for, but is out of this work's scope. \\
\hline

\textbf{Propor-tional Allocation Factor} &
None &
Allocates the total marginal as opportunities proportionally based on $\kappa_{ij}^T = K^T f(c_{ij})$ and as population based on  $\hat{\kappa}_{ji}^T = \hat K^T f(c_{ij})$.

Multimodal extension:

$\kappa_{ij}^{mT} = B_j^m W_i^{m(1)} f^m(c_{ij}^m)$;
$\hat{\kappa}_{ji}^{mT} = A_i^m W_j^{m(2)} f^m(c_{ij}^m)$ 

&
Allocates the single opportunities marginal proportionally based on $\kappa^S_{ij} = B_j W_i^{(1)} f(c_{ij}) $ in the case of $V_i^S$ and the single population marginal proportionally based on $\hat \kappa^S_{ji} = A_i W_j^{(2)} f(c_{ij})$;

Multimodal extension:

$\kappa_{ij}^{mS} = B_j^m W_i^{m(1)} f^m(c_{ij}^m)$;
$\hat{\kappa}_{ji}^{mS} = A_i^m W_j^{m(2)} f^m(c_{ij}^m)$ 

&
— \\
\hline

\textbf{Interpret-ation} &
Values in various units depending on the impedance and destination-mass (e.g., "opportunities x decay") for $V_i^0$ and impedance and origin-mass (e.g., "population x decay"); no total or marginal constraint &
Values reflect a share of total regional opportunities ($V^T_i$) or population ($M^T_j$). &
$V^S_i$ values reflect a share of the opportunities at each destination based on origin population 'demand' and impedance; $M^S_j$ values reflect a share of the population at each origin based on destination opportunities 'supply' and impedance. &
The spatial interactions between population and opportunities (i.e., access). \\
\hline

\end{longtable}
}


\subsection{{[}NEW - from multimodal SA paper{]} A brief review of multimodal accessibility measures}\label{new---from-multimodal-sa-paper-a-brief-review-of-multimodal-accessibility-measures}

Unconstrained accessibility, namely does not consider the opportunities allocated (or population) as finite. To cite an example, Tahmasbi et al.~(2019) (Tahmasbi, Mansourianfar, Haghshenas, \& Kim, 2019) uses Hansen-type accessibility to assess the potential interaction with retail locations by three modes: walking, public transit, and car (i.e., \(m = w, p, c\)). \(S_i^m\) is the sum of retail locations \(j\) that can potentially be reached under the travel impedance as calculated for each \(i\) and \(m\). In other words, for each origin \(i\) three accessibility scores are calculated. Tahmasbi et al. (2019) shows that car travel affords the highest \(S_i^{m}\) values in the majority of \(i\) i.e., travelers who use a car can potentially reach more retail opportunities than populations using other modes. However, higher \(S_i^{m}\) values for car do not affect the values of \(S_i^{m}\) for other modes: in effect, each mode is analysed as if the others did not exist. Since the measure is not constrained, each opportunity is typically counted multiple times within and between modes, and as a result the sum of accessibility is not necessarily a meaningful quantity. The accessibility scores for the modes are often values that are difficult to interpret beyond making statements about relative size.

As another example, Lunke (2022) reports accessibility scores for car in the order of tens of thousands of employment opportunities in the Oslo region. The corresponding scores for transit are lower, but still often in the thousands or tens of thousands. As reported, the ratio of the transit to the car score can be lower than 0.2 (meaning transit gives access to less than 20\% of the opportunities than car). But despite the discussion about ``sufficient accessibility'', it is unclear what the unconstrained scores mean: is having access to 10,000 jobs by transit insufficient? After all, 10,000 employment opportunities are still plenty of opportunities. These ratios can be found elsewhere in the literature e.g., Figs. 7, 8, and 9 in Páez et al.~(2010) (A. Páez, Mercado, Farber, Morency, \& Roorda, 2010), Fig. 5 in Páez et al.~(2010) (Antonio Páez et al., 2010), Páez et al.~(2013) Figs. 6, 7, and 8 in (Antonio Páez et al., 2013). They are useful as relative assessment of when some members of the public are better or worse off than others, but they are silent on how bad is ``worse off''.

Besides ratios of accessibility, another way seen in the literature to improve interpretability of scores is to standardise them within a {[}0-1{]} range. This adjustment is only helpful insofar as it facilitates relative comparisons, but interpretation of the scores remains challenging because the values are specific to a region and convey no meaning about the magnitude of the scores. In this approach, zones always have values between 0 and 1, but how remarkable is a zone with a low score for pedestrians and a high value for car? And if remarkable, what does the difference in these standardized values mean for planners? By how much should transport systems and land-use configurations be changed to improve conditions? And in what way can these scores be used to track differences over time? Or between regions? These questions lack straightforward answers since certain values will always be relatively `low' or `high', but do not track to a quantity that can be intuitively understood. Presentation or discussion of Hansen-type accessibility that has been standardised in this way is not uncommon in the literature (Campbell, Rising, Klopp, \& Mbilo, 2019; Maharjan, Tilahun, \& Ermagun, 2022).

If we understand opportunities to be finite and/or subject to some levels of congestion, it is possible for an accessibility measure to take on a crisper meaning. Accessibility research has a history of considering opportunity competition, especially regarding school-seats, hospital capacity, and employment opportunities (Cui et al., 2020; Grengs, 2010b; Higgs, Zahnow, Corcoran, Langford, \& Fry, 2017; Joseph \& Bantock, 1982; Kawabata \& Shen, 2006; Kelobonye, Zhou, McCarney, \& Xia, 2020; Kwok \& Yeh, 2004; Li \& Wang, 2024; Mao \& Nekorchuk, 2013; Merlin \& Hu, 2017; Morris et al., 1979; Q. Shen, 1998a; Soukhov et al., 2023; Van Wee, Hagoort, \& Annema, 2001; Jörgen W. Weibull, 1976). If one person reaches an opportunity - it is taken: the supply of an opportunity and the demand for that opportunity are the nodes in accessibility analysis. These types of opportunities are unambiguously exclusive.

Amenities are a good example of this. For instance, standards for providing green spaces are often stated in the form of \emph{exclusive access}, in units of amenity per capita. For example, a 2013 planning document for the Ile-de-France region suggested a public green space municipal standard of 10 \(m^2\) \emph{per inhabitant} (Liotta, Kervinio, Levrel, \& Tardieu, 2020). Green spaces are not evenly distributed, meaning those who have access to them depends on where they are and how easy they are to reach. This formulation of amenity provision is not unusual. As another example, Natural England recommends a national ``accessible natural greenspace standard'' such that the minimum supply of space is 1 \(ha\) of statutory local nature reserves per thousand population (Natural England, 2010). Similarly, the World Health Organization (OECD, 2013) recommends that cities provide a minimum of 9 \(m^2\) of green area per inhabitant. For our purposes, standards of this type translate into ``how much of this resource is available to one individual that has not been claimed by anyone else?''. Green spaces often have large capacities, but they still have a capacity and it is not the same for a person to have access to 5 \(m^2\) of \emph{uncongested} green space as 15 \(m^2\). This difference is in fact a matter of justice (Lara-Valencia \& García-Pérez, 2015; Liotta et al., 2020). Constraining accessibility is a useful way to evaluate the congested availability of any type of opportunity. As development of sound standards is emphasized in the planning literature, in particular in regards to fairness in transportation (Martens \& Golub, 2021), spatial availability analysis can be used to develop and assess standards.

The relevance of the considerations above is put in sharper relief when we think about the use of multiple modes (or heterogeneous populations). If we return to Oslo for a moment (Lunke, 2022), we notice that the places that have high accessibility by transit are also the places that have \emph{very high} accessibility by car (in their Figure 2). Those two populations are going for the same opportunities, and those travelling by transit have fewer to choose from the start. More generally, people in a zone who are advantaged with relatively low cost of travel will have the ability to potentially reach more opportunities than other people. Due to this advantage, through the perspective of finite opportunities, there are fewer opportunities left for everyone else, especially for those who use modes that are slower or otherwise more expensive.

`Competitive' accessibility was the rationale for developing floating catchment area methods (FCA), popularized in Luo \& Wang (2003b) who reformulated the work of Q. Shen (1998a) into two steps (although similar, and earlier, developments are found in (Joseph \& Bantock, 1982; Jörgen W. Weibull, 1976)). Shen-type accessibility is formulated as: \(a_i^m = \sum_j \frac{O_jf^m(c_{ij}^m)}{\sum_m D_j^m}\) where \(D_j^m\) is the potential demand for opportunities equal to travel impedance weighted population \(\sum_i P_i^m f^m(c_{ij}^m)\) and the remaining variables are repeated in the Hansen-type measure. Shen-type modal accessibility (\(a_i^m\)) can be understood as a ratio of the travel impedance-weighted supply of opportunities for \(m\)-mode in \(i\) over the travel impedance-weighted demand for opportunities. In this way, it considers competition. That said, the measure remains unconstrained, meaning both population \emph{and} opportunities are multiply counted (Antonio Paez et al., 2019). In other words, interpretation of the Shen-type accessibility scores between modes is fraught, as it is for Hansen-type measures.

To illustrate, Tao et al. (2020) calculates \(a_i^m\) to jobs for different income-group populations in Shenzhen, China for those using transit and car. Their results indicate that zones with low-income populations have lower \(a_i^m\) than zones with higher-income populations. Further, they show that \(a_i^{\text{transit}}\) is lower than \(a_i^{\text{car}}\) in many zones, arguing that this may further place those zones with lower-income populations at a disadvantage. \(a_i\) and/or \(a_i^m\) are used to compare relative spatial differences in overall competitive accessibility and multimodal competitive accessibility, but because opportunities were doubly counted (entering the sums of both modes), this makes for uneasy interpretations of the differences in \(a_i^{m}\) between modes. Questions that this approach leaves unaddressed include: what is the impact of competition on the difference in \(a_i^m\) values? How does the impact vary spatially? And what is the interpretation of this difference?

The family of accessibility measures improves on the discussed Hansen-type and Shen-type accessibility approaches by constraining the sum of opportunities, that is, by treating them as finite. This is done by means of proportional allocation factors that follow well established principles of spatial interaction and the gravity model (Wilson, 1971). These principles consider: the cost of travel from different zones for a certain traveling group \(m\) compared to all groups (e.g., some sub-populations face relatively higher or lower costs), and if singly-constrained, the mass effect of \(m\) compared to all other \(m\)s e.g., the mass at different origins or destinations).

\section{Setup of the simple numeric example}\label{setup-of-the-simple-numeric-example}

Consider a simple region as shown in Figure \ref{fig:chp2-fig-analytical-device-conc-accessibility}, with zone IDs 1, 2 and 3. Each zone is both an origin \(i\) and a destination \(j\). The following three pieces of information are defined: zonal population and opportunities, zonal cost matrix, and travel impedance functions for three types of travel behaviour. In this example, the population are people, the opportunities are physicians, and the region can be described by three possible travel behaviours.

\begin{table}[!t]
\caption{Simple system with three zones (ID 1, 2 and 3). Population is in 10,000 persons and opportunities in number of physicians} \label{tab:chp2_small_system_land_use_tab}
\fontsize{7.5pt}{9.0pt}\selectfont
\begin{tabular*}{\linewidth}{@{\extracolsep{\fill}}rcc}
\toprule
ID (i or j) & Population\textsuperscript{\textit{1}} & Opportunities\textsuperscript{\textit{2}} \\ 
\midrule\addlinespace[2.5pt]
1 & 4 & 160 \\ 
2 & 10 & 150 \\ 
3 & 6 & 180 \\ 
\bottomrule
\end{tabular*}
\begin{minipage}{\linewidth}
\textsuperscript{\textit{1}}Population is \emph{Wi\textsuperscript{(1)}} when used as a proxy for the mass at the origin, and \emph{Oi} when used as a constraint.\\
\textsuperscript{\textit{2}}Opportunities is \emph{Wj\textsuperscript{(2)}} when used as a proxy for the mass at the destination, and \emph{Dj} when used as a constraint.\\
\end{minipage}
\end{table}



\begin{table}[!t]
\caption{Cost matrix for system with three zones (travel time in minutes)} \label{tab:chp2_small_system_land_use_cost_tab}
\fontsize{7.5pt}{9.0pt}\selectfont
\begin{tabular*}{\linewidth}{@{\extracolsep{\fill}}cccc}
\toprule
 & \multicolumn{3}{c}{Destination ID} \\ 
\cmidrule(lr){2-4}
Origin ID & 1 & 2 & 3 \\ 
\midrule\addlinespace[2.5pt]
1 & 10 & 30 & 15 \\ 
2 & 30 & 10 & 25 \\ 
3 & 15 & 25 & 10 \\ 
\bottomrule
\end{tabular*}
\end{table}



Firstly, Table \ref{tab:chp2_small_system_land_use_tab} summarises the population (in units of 10,000s of people) and the opportunities (the number of physicians) per zone. {[}\^{}1{]}. Considering Table \ref{tab:chp2_small_system_land_use_tab}'s values, the Provider-to-Population-Ratio (PPR) in this system is 24.5. For reference, the number of physicians per 10,000 in Canada in 2022 was 24.97 (WHO, 2025). Secondly, to pair the zonal population and opportunity information, the assumed cost of movement (in minutes of travel time) between origins and destinations is as shown in Table \ref {tab:chp2_small_system_land_use_cost_tab}.

From both Table \ref{tab:chp2_small_system_land_use_tab} and Table \ref {tab:chp2_small_system_land_use_cost_tab}, Zone 3 and 1 can be interpreted as of an urban core with major healthcare institutions and a healthcare cluster on the edge of the city respectively, each with a moderate residing population (with zone 3 having both a higher population and physician count). Zone 2, can be interpreted as a more distant bedroom community, with a relatively high population and fewer physicians. In sum, Zones 1 and 3 are more proximate to each other than to Zone 2, and together match Zone 2's population while offering more than twice the physician availability.

And lastly, in Equation \ref{eq:travel-behaviour-scenarios} we distinguish accessibility measure values for the following three impedance functions that represent the potential for spatial interaction travel behaviour of the population to opportunities. Accessibility will be calculated three times for each case, one assuming the most decay (\(f_1(c_{ij})\)), another assuming medium decay (\(f_2(c_{ij})\)), and a third assuming the least decaying travel behaviour (\(f_3(c_{ij})\)) for the entire region. A helpful analogy may be tying travel behaviour to the used mode's mobility potential, i.e., the most decaying travel behaviour (\(f_1(c_{ij})\)) would assume all travel in the region being done by foot, while calculating accessibility assuming the least decay (\(f_3(c_{ij})\)) would assume unfettered automobility. Or alternatively, these functions could represent travel behaviour on snowstorm-affected day (\(f_1(c_{ij})\)) for the entire region versus a clear, ideal travel day (\(f_3(c_{ij})\)). As an example of a discussion on how travel behaviour has been considered in accessibility measures cost of travel see A. Paez, Scott, \& Morency (2012).

\begin{equation}
\label{eq:travel-behaviour-scenarios}
\begin{array}{l}
f_1(c_{ij}) = \frac{1}{c_{ij}^3}\\
f_2(c_{ij}) = \frac{1}{c_{ij}^2}\\
f_3(c_{ij}) = \frac{1}{c_{ij}^{0.1}}
\end{array}
\end{equation} 

Any set of concepts representing population, opportunities, and their associated travel behaviour, whether representing the entire region uniformly (as we will demonstrate) or representing specific subgroups, can be substituted into our simple example, depending on the research question. The purpose of the following simple example is to demonstrate the calculation of each member of the accessibility measure family, interpret the values, and compare them both within and across travel behaviour groups and members of the family of accessibility measures.

\section{Unconstrained accessibility}\label{unconstrained-accessibility}

Setting the balancing factor \(k\) to 1 or omitting it completely in Equation \ref{eq:access-01} results in the unconstrained accessibility case:

\begin{equation}
\label{eq:unconstrained-access}
V^0_{ij} = 1 *W_j^{(2)} f(c_{ij})
\end{equation} 

In this case, the partial sum of spatial interaction is simply identical to Hansen's accessibility \(S_i\) (Hansen, 1959), the current standard practice in accessibility measurement:

\begin{equation}
\label{eq:unconstrained-accessibility}
V^0_i = \sum_j V^0_{ij} = \sum_j W^{(2)}_jf(c_{ij}) = S_i
\end{equation} 

The sum of the unconstrained accessibility values for each origin \(V^0_{i}\) generally does not equal the total number of opportunities \(O\) (e.g., \(\sum_i V^0_{i} \not= O\)), since arbitrarily setting \(k\) to 1 (or neglecting \(k\) all together) strips the values of any meaningful unit-based interpretation, the units are in `summed opportunities by some travel impedance value'. Moreover, comparisons of these \(V^0_{i}\) values across different contexts such as different impedance functions \(f(c_{ij})\) or varying number of zones exacerbates this issue as the units between \(V^0_{i}\) values change i.e., comparisons between a value of `summed opportunities by a travel impedance value' to a value of `summed opportunities by another travel impedance value' are not directly interpretable. From this perspective, the raw unconstrained accessibility scores are not intuitively comparable across different contexts and decay functions. They should more appropriately be used as an ordinal variable to make comparisons of size (i.e., greater than, less than, equal to), not to calculate ratios or intervals (i.e., the magnitude of differences).

The multimodal version of unconstrained accessibility can be represented as \(V^{m0}_{i}\) (Equation \ref{eq:unconstrained-multimodal-accessibility}) for each mode \(m\) assuming mode-specific impedance functions \(f^m(c^m_{ij})\). However, variables of other modes (i.e., summation of all \(m\), etc.) are not included in the equation, so \(V^{m0}_{i}\) is effectively equivalent to \(V^0_{i}\)--except \(m\) makes explicit that there are multiple groups at each \(i\) and those groups have their own \(V^0_i\) values based on \(f^m(c^m_{ij})\). There is no consistent approach of representing \(V^0_{i}\) as a function of multiple modes, other than averaging \(i\) values or other post hoc adjustments that generally do not preserve known properties of the system. From this perspective, as \emph{other} \(m\) values are not incorporated into the measure, we assume \(V^0_{i}\) cannot be made multimodal, and only \(V^0_{i}\), as calculated for different \(m\)s is used in this work.

\begin{equation}
\label{eq:unconstrained-multimodal-accessibility}
V^{m0}_{i} =  \sum_j W_j^{(2)} f^m(c^m_{ij})
\end{equation} 

Returning to our numeric example, the calculated unconstrained accessibility \(V^0_{i}\) for each origin, a sum of all the travel impedance weighted opportunities at each destination (\(\sum_i V^0_{i}\)), in displayed in Table \ref{tab:chp2_simple_example_unconc_access_tab}.

\begin{table}[!t]
\caption{Simple system: unconstrained accessibility} \label{tab:chp2_simple_example_unconc_access_tab}
\fontsize{7.5pt}{9.0pt}\selectfont
\begin{tabular*}{\linewidth}{@{\extracolsep{\fill}}>{\raggedright\arraybackslash}p{\dimexpr 45.00pt -2\tabcolsep-1.5\arrayrulewidth}|>{\centering\arraybackslash}p{\dimexpr 135.00pt -2\tabcolsep-1.5\arrayrulewidth}>{\centering\arraybackslash}p{\dimexpr 135.00pt -2\tabcolsep-1.5\arrayrulewidth}>{\centering\arraybackslash}p{\dimexpr 135.00pt -2\tabcolsep-1.5\arrayrulewidth}}
\toprule
 & \multicolumn{3}{>{\centering\arraybackslash}m{\dimexpr 405.00pt -2\tabcolsep-1.5\arrayrulewidth}}{V\textsubscript{i}\textsuperscript{0}} \\ 
\cmidrule(lr){2-4}
 & f\textsubscript{1} (c\textsubscript{ij}) = 1/c\textsubscript{ij}\textsuperscript{3} & f\textsubscript{2} (c\textsubscript{ij}) = 1/c\textsubscript{ij}\textsuperscript{2} & f\textsubscript{3} (c\textsubscript{ij}) = 1/c\textsubscript{ij}\textsuperscript{0.1} \\ 
\cmidrule(lr){2-2} \cmidrule(lr){3-3} \cmidrule(lr){4-4}
Origin & units: \emph{physicians-minute\^{}-3} & units: \emph{physicians-minute\^{}-2} & units: \emph{physicians-minute\^{}-0.1} \\ 
\midrule\addlinespace[2.5pt]
1 & 0.219 & 2.567 & 371.143 \\ 
2 & 0.167 & 1.966 & 363.479 \\ 
3 & 0.237 & 2.751 & 373.738 \\ 
\midrule 
\midrule 
Sum & 0.6233422 & 7.283556 & 1108.361 \\ 
\bottomrule
\end{tabular*}
\end{table}



As the different impedance functions represent different travel behaviours, comparing the raw unconstrained accessibility values across groups is meaningless beyond notions of higher or lower. For instance, at zone 1 the difference between the least decay (\(f_3(c_{ij})\)) and most decay (\(f_1(c_{ij})\)) groups is 370.92, but in what units? These two values are a product of different impedance functions, making the comparison uninterpretable in absolute terms. Likewise, we could compare values within the same travel behaviour scenario across different zones, as they are in the same units, however the issue of unit interpretability will also be apparent. Considering the most decaying scenario \(f_1(c_{ij})\) and zone 1 (the zone with a healthcare cluster at the edge of the city): zone 1 captures 0.0181185 fewer physicians-minute\(^{-1}\) than zone 3 (urban core). Again, the fundamental uninterpretability of what is a \emph{physicians-minute\(^{-1}\)} or \emph{opportunity-weighted-travel-impedance} unit remains.

While one could attempt to adjust the units post-calculation (e.g., scaling, population normalization) or select impedance functions to facilitate comparison across scenarios (potentially at the expense of accurately reflecting travel behavior), such adjustments may introduce bias. Hence, the raw unconstrained accessibility values themselves are challenging to compare due to their units. To enable more meaningful comparison, the following sections will detail the introduction of constraining constants to ensure consistent units across scenarios and demonstrate results on the same numeric example.

\section{Total constrained accessibility}\label{total-constrained-accessibility}

In the total constrained accessibility case, a total balancing factor proportionally adjusts unconstrained zonal accessibility values \(V^0_i\) based on the regional sum of \(V^0_i\) and the total population or opportunities in the region. Alternatively, we reformulate this case using a proportional allocation constant, which allocates opportunities (or population) proportionally based on the the travel impedance and the total population or opportunities in the region. In both formulations, all zonal values become a proportion of a known system total, be it the regional opportunities or regional population depending on the variant.

We define two variants for this case: (a) \(V_i^T\) where accessibility is constrained by the total number of opportunities (total constrained accessible opportunity) and which is interpreted as Hansen's accessibility with a constraining constant, and (b) \(M_j^T\), where \(i\) and \(j\) of the first variant is transposed, yielding a measure constrained by the total number of population and to be interpreted as constrained `market potential'. Following these definitions, these variants are extended into their multimodal expressions, considering multiple travel behaviour groups or modes \(m\).

\subsection{Total constrained accessible opportunities: Hansen's accessibility with a total constraint}\label{total-constrained-accessible-opportunities-hansens-accessibility-with-a-total-constraint}

Unlike in Equation \ref{eq:access-01}, the proportionality constant \(k\) is retained. For the total constrained case, it is represented as \(K^T\):

\begin{equation}
\label{eq:total-constrained-access}
V^T_{ij} = K^T \cdot W_j^{(2)} \cdot f(c_{ij})
\end{equation} 

In this way, the total constrained accessibility measure now becomes Hansen's accessibility with a balancing factor \(K^T\):

\begin{equation}
\label{eq:total-constrained-accessibility}
V^T_i = \sum_j V^T_{ij} = K^T \sum_j W^{(2)}_jf(c_{ij}) = K^T \cdot V^0_i
\end{equation} 

Imagine that the only system known is the total number of opportunities \(D\) in the region. Accordingly, the constant we impose in this case ensures the regional sum of total constrained accessibility is equal to the total number of opportunities as follows:

\begin{equation}
\label{eq:total-constraint-accessibility}
\sum_i V^T_{i} = \sum_i\sum_j V^T_{ij} = D
\end{equation} 

This constraint is analogous to the total constraint of Equation \ref{eq:constraint0-gravitymodel}, congruent with Wilson's framework. Given the total number of opportunities in the region, we can then substitute Equation \ref{eq:total-constrained-accessibility} in Equation \ref{eq:total-constraint-accessibility} to solve for \(K^T\):

\begin{equation}
\label{eq:total-opportunity-balancing-factor}
K^T =\frac{D}{\sum_i\sum_j V^0_{ij}} = \frac{D}{\sum_i\sum_j W^{(2)}_jf(c_{ij})}
\end{equation} 

Which is also congruent with Wilson's framework as it comparable to the total flow spatial interaction model (e.g., Equation 2.11 in Cliff et al. (1974)). Hence, rearranging the equation to have opportunities and the proportional constant distinctly represented, our total constrained accessibility model is:

\[
V^T_i = K^T\sum_j W^{(2)}_jf(c_{ij}) = \sum_j W^{(2)}_j\frac{D\cdot f(c_{ij})}{\sum_i\sum_j W^{(2)}_jf(c_{ij})}
\]

Further, we can see that, since \(D\) and \(W^{(2)}_j\) are both in units of opportunities, the proportional allocation factor for the total constrained opportunity case \(\kappa_{ij}^T\) is dimensionless:

\[
\kappa_{ij}^T = \frac{D\cdot f(c_{ij})}{\sum_i\sum_j W^{(2)}_jf(c_{ij})}
\]

\noindent and therefore \(V^T_i\) is now in the units of \(W^{(2)}_j\), that is, the mass at the destination (\(V^T_i = \kappa_{ij}^T\sum_j W^{(2)}_j\)). The role of \(\kappa_{ij}^T\) in this reformulation of accessibility is to transform between units and to adjust the number of opportunities accessible from \(i\) so they represent a proportion of the total number of opportunities in the region. \(\kappa_{ij}^T\) then assigns opportunities in proportion to the impedance between \(i\) and \(j\). This is why we refer to \(\kappa_{ij}^T\) as a proportional allocation factor. On the other hand, the proportionality constant \(K^T\) balances the units of \(V^0_i\), the Hansen-type accessibility values, and is an alternative expression of the total constrained accessibility measure.

Referring back to our simple numeric example, balancing factor \(K^T\) for the most decay travel behaviour scenario \(f_1(c_{ij}) = 1/c_{ij}^3\) would then be:
\[
\begin{array}{l}
K^T = \frac{D}{\sum_{i}\sum_{j} W_j^{(2)} f(c_{ij})}\\
K^T = \frac{D}{\frac{W_1^{(2)}}{c_{11}^3}+\frac{W_1^{(2)}}{c_{21}^3} + \frac{W_1^{(2)}}{c_{31}^3} + \cdots + \frac{W_3^{(2)}}{c_{31}^3} + \frac{W_3^{(2)}}{c_{32}^3} + \frac{W_3^{(2)}}{c_{33}^3}
}\\
K^T = \frac{490}{0.6233422} \\
K^T = 786.085
\end{array}
\]

\(K^T\) for the lower decay scenarios \(f_2(c_{ij}) = 1/c_{ij}^2\) and \(f_3(c_{ij}) = 1/c_{ij}^0.1\) are 67.2748352 and 0.4420944 respectively.

Using the calculated balancing factors for all zones and multiplying them by the unconstrained accessibility value \(V^0_i\), the total opportunity constrained accessibility values for all zones and different travel behaviour scenarios is presented in Table \ref{tab:chp2_simple_example_total_opp_access_tab}. \(\kappa_{ij}^T\) for each zone is not reported, but can be understood to be the unitless proportion of opportunities (of the total opportunities) allocated to each zone based on travel impedance.

\begin{table}[!t]
\caption{Simple system: total constrained accessible opportunities} \label{tab:chp2_simple_example_total_opp_access_tab}
\fontsize{7.5pt}{9.0pt}\selectfont
\begin{tabular*}{\linewidth}{@{\extracolsep{\fill}}>{\raggedright\arraybackslash}p{\dimexpr 45.00pt -2\tabcolsep-1.5\arrayrulewidth}|>{\centering\arraybackslash}p{\dimexpr 135.00pt -2\tabcolsep-1.5\arrayrulewidth}>{\centering\arraybackslash}p{\dimexpr 135.00pt -2\tabcolsep-1.5\arrayrulewidth}>{\centering\arraybackslash}p{\dimexpr 135.00pt -2\tabcolsep-1.5\arrayrulewidth}}
\toprule
 & \multicolumn{3}{>{\centering\arraybackslash}m{\dimexpr 405.00pt -2\tabcolsep-1.5\arrayrulewidth}}{V\textsubscript{i}\textsuperscript{T}} \\ 
\cmidrule(lr){2-4}
 & f\textsubscript{1} (c\textsubscript{ij}) = 1/c\textsubscript{ij}\textsuperscript{3} & f\textsubscript{2} (c\textsubscript{ij}) = 1/c\textsubscript{ij}\textsuperscript{2} & f\textsubscript{3} (c\textsubscript{ij}) = 1/c\textsubscript{ij}\textsuperscript{0.1} \\ 
\cmidrule(lr){2-2} \cmidrule(lr){3-3} \cmidrule(lr){4-4}
Origin & units: \emph{physicians} & units: \emph{physicians} & units: \emph{physicians} \\ 
\midrule\addlinespace[2.5pt]
1 & 172.065 & 172.672 & 164.080 \\ 
2 & 131.627 & 132.247 & 160.692 \\ 
3 & 186.308 & 185.081 & 165.228 \\ 
\midrule 
\midrule 
Sum & 490 & 490 & 490 \\ 
\bottomrule
\end{tabular*}
\end{table}



In contrast to unconstrained accessibility \(V^0_i\), imposing a constraint -in this case a total opportunity constraint for this variant- allows for the comparison of differences and ratios between regions and across different travel behaviour scenarios as well. Each value is effectively in units of physicians, with the impedance units already accounted for by \(\kappa_{ij}^T\).

Considering the highest decay scenario (\(f_1(c_{ij})\)), zone 1 (a healthcare cluster at the edge of the city) captures an intermediate amount of physicians (172.0652825) like in the unconstrained accessibility case. However, unlike in the unconstrained case, we can say that this value is out of the 490 physicians in the region, which allows us also to deduce that zone 1 captures 1.3072213 and 0.9235529 times more than zone 2 and 3. Values for the lesser decay (\(f_2(c_{ij})\)) and lowest decay (\(f_3(c_{ij})\)) scenarios are calculated separately, with decay scenario values also summing to equal 490 physicians accessible in the region.

One can also directly compare values at a specific zone due to the consistent units. For instance, zone 1 remains intermediate in capturing accessible physicians relative to zones 2 and 3 across scenarios, similar to the unconstrained case. However, the difference between travel behaviour scenarios differ in direction. Specifically, Zone 1 captures 0.6067946 more and 7.9850478 fewer opportunities than the lesser decay scenarios \(f_2(c_{ij})\) and \(f_3(c_{ij})\) respectively. Why? \(\kappa_{ij}^T\) ensures proportional allocation for each travel behaviour scenario. Meaning, while the unconstrained accessibility increases, \(\kappa_{ij}^T\) adjusts the values to remain proportional to the total number of opportunities (490 physicians accessible in the region). As the decay behaviour decreases, more opportunities are accessible for all zones. In the medium decay scenario \(f_2(c_{ij})\), zone 1 sees a slight increase in values (relative to the highest decay scenario) as the zone can accessible more opportunities relative to increases seen in other zones. However, in the lowest decay scenario, zone 1 sees a decrease, as it is outpaced by increases in other zones - namely zone 2 (recall: zone 2 has the lowest number of opportunities, hence the increases in opportunity gains is much higher in a low decay scenario).

Using the total opportunity constrained formulation of accessibility offers a solution to the unit interpretability issue of Hansen (1959)'s accessibility measure. Intuitively, the use of the constraint illustrates how the differences and ratios of values between zones and decay groups can be compared. This is true for other constrained cases of the family of accessibility measures.

\subsection{Total constrained accessible population: Reilly's potential trade territories with a total constraint}\label{total-constrained-accessible-population-reillys-potential-trade-territories-with-a-total-constraint}

Another variant of the total constrained accessibility measure is the \emph{total constrained accessible population measure}, which represents the transpose of \(i\) to \(j\) of the total constrained accessible opportunities measure. This variant, expressed in Equation \ref{eq:total-constrained-market}, represents an expression of the concept of market potential (i.e., potential users) as proposed in C. D. Harris (1954) and R. W. Vickerman (1974), and which Reilly earlier referred to as `potential trade territories' (Reilly, 1929). This unconstrained form of market potential \(M_j^0\) (Equation \ref{eq:unconstrained-market}), effectively the \(i\) to \(j\) transpose of \(V^0_{ij}\), has been used in recent research to express the potentially accessible population (i.e., users) as a result of regional transportation infrastructure investment projects (Condeço-Melhorado \& Christidis, 2018; e.g., Gutiérrez, 2001; Holl, 2007). Put another way, market potential can also be thought of as a form of \emph{passive accessibility}, indicating the number of people that can reach each destination. However, like \(V_{ij}^0\), issues of unit interpretability arise in \(M_j^0\)'s unconstrained form. To address this, the constrained variant, the total constrained accessible population measure \(M^T_{j}\), is introduced. To formulate this variant, the total balancing factor \(K^T\) is applied to the mass of the \emph{population} at \(i\) (\(W_i^{(1)}\)) instead of the opportunities at \(j\).

\begin{equation}
\label{eq:total-constrained-market}
M^T_{ji} = \hat K^T \cdot W_i^{(1)} f(c_{ij}) = \hat K^T \cdot M_{ji}^0
\end{equation} 

\noindent with \(M_j^0\) being the \(i\) \(j\) transpose of \(V^0_{ij}\):

\begin{equation}
\label{eq:unconstrained-market}
M_j^0 = W_i^{(1)} f(c_{ij})
\end{equation} 

The total constrained accessible population measure (market potential) then becomes:
\[
M^T_j = \sum_i M^T_{ji} = \hat K^T \sum_i W^{(1)}_if(c_{ij})
\]

Where we impose the total system known as a constraint, i.e., that the total market potential equals the total population \(O\) in the region:

\begin{equation}
\label{eq:total-constraint-market}
\sum_j M^T_{j} = \sum_i\sum_j M^T_{ji} = O
\end{equation} 

Substituting Equation \ref{eq:total-constrained-market} in Equation \ref{eq:total-constraint-market}, and solving for \(\hat K^T\), we obtain:

\begin{equation}
\label{eq:total-population-balancing-factor}
\hat K^T=\frac{O}{\sum_i\sum_i M^T_{ji}} = \frac{O}{\sum_i\sum_j W^{(1)}_i f(c_{ij})} 
\end{equation} 

The constrained market potential then takes the following form:
\[
M^T_j = \hat K^T\sum_i W^{(1)}_if(c_{ij}) = \sum_i W^{(2)}_i \frac{O \cdot f(c_{ij})}{\sum_i\sum_j W^{(2)}_if(c_{ij})}
\]

Where the following \(\hat \kappa_{ji}^T\) proportional allocation factor is dimensionless:
\[
\hat \kappa_{ji}^T = \frac{\sum_i O \cdot f(c_{ij})}{\sum_i\sum_j W^{(2)}_if(c_{ij})}
\]

Returning back to the numerical example, the proportionality constant \(\hat K^T\) is solved for each travel behaviour scenario, and the market potential of each zone \(M^T_j\) is expressed as units of population (e.g., the number of people accessible from each origin at that destination) in Table \ref{tab:chp2_simple_example_total_pop_market_tab}.

\begin{table}[!t]
\caption{Simple system: total constrained accessible population.} \label{tab:chp2_simple_example_total_pop_market_tab}
\fontsize{7.5pt}{9.0pt}\selectfont
\begin{tabular*}{\linewidth}{@{\extracolsep{\fill}}>{\raggedright\arraybackslash}p{0.1\linewidth}|>{\centering\arraybackslash}p{0.225\linewidth}>{\centering\arraybackslash}p{0.225\linewidth}>{\centering\arraybackslash}p{0.225\linewidth}>{\centering\arraybackslash}p{0.225\linewidth}}
\toprule
 & \multicolumn{3}{>{\centering\arraybackslash}m{\dimexpr 337.50pt -2\tabcolsep-1.5\arrayrulewidth}}{M\textsubscript{i}\textsuperscript{S}} \\ 
\cmidrule(lr){2-4}
 & f\textsubscript{1} (c\textsubscript{ij}) = 1/c\textsubscript{ij}\textsuperscript{3} & f\textsubscript{2} (c\textsubscript{ij}) = 1/c\textsubscript{ij}\textsuperscript{2} & f\textsubscript{3} (c\textsubscript{ij}) = 1/c\textsubscript{ij}\textsuperscript{0.1} \\ 
\cmidrule(lr){2-2} \cmidrule(lr){3-3} \cmidrule(lr){4-4}
Destination & units: \emph{population in 10,000s} & units: \emph{population in 10,000s} & units: \emph{population in 10,000s} \\ 
\midrule\addlinespace[2.5pt]
1 & 5.018 & 5.447 & 6.598 \\ 
2 & 8.596 & 7.986 & 6.717 \\ 
3 & 6.386 & 6.567 & 6.684 \\ 
\midrule 
\midrule 
Sum & 20 & 20 & 20 \\ 
\bottomrule
\end{tabular*}
\end{table}



Readers may note the difference in trends in accessible population (Table \ref{tab:chp2_simple_example_total_pop_market_tab}, immediately above) and the preceeding accessible physicians (Table \ref{tab:chp2_simple_example_total_opp_access_tab}). In the accessible population Table \ref{tab:chp2_simple_example_total_pop_market_tab}, zone 1, 2, 3 represent destinations and the accessibility values reflect the number of accessible people from the vantage of physicians. Zone 1, in its role as a destination, is no longer intermediately-ranked relative to other zones; it now attracts the fewest number of people across all three travel behaviour scenarios. However, similar to the total constrained opportunity case, as travel decay reduces, the availability of population begins to converge (though Zone 1 continues as the lowest-ranked) for similar reasons. As decay reduces, the population's travel impedance to all zones become more similar, making the relative location of the zones less important and all people in the region more equally accessible. Overall: like the total constrained accessible opportunities case, this variant allows for the interpretation of both ordinal and interval comparisons of the raw values themselves.

\subsection{{[}NEW{]} Multimodal extension of total constrained accessible opportunities and population}\label{new-multimodal-extension-of-total-constrained-accessible-opportunities-and-population}

Having detailed both variants \(V_i^T\) and \(M_j^T\), their multimodal extensions can now be introduced. This extension is relevant if one is interested in accounting for different travel behaviour groups \(m\) within the same system, i.e., where the resulting total constrained accessibility value is a function of all the \(m\)s in the system. The total constrained multimodal accessible opportunities for each group \(m\) would be:

\begin{equation}
\label{eq:total-constrained-multimodal-access}
V^{mT}_{ij} = K^{mT} \cdot W_j^{(2)} \cdot f^m(c^m_{ij})
\end{equation} 

Where:
- \(V^{mT}_{ij}\) is the number of opportunities that can be accessed at origin zone \(i\) from destination zone \(j\) by mode \(m\),
- \(f^m(c^m_{ij})\) is the cost of travel \(c^m_{ij}\) by mode \(m\) from \(i\) to \(j\),
- the destination zone attraction mass \(W_j^{(2)}\); and
- \(K^{mT}\) is the modal balancing factor \(\frac{D}{\sum_m\sum_i\sum_j W^{(2)}_jf^m(c^m_{ij})}\) that serves to allocate the opportunities in the region and ensures units remain balanced.

Summarising equation \ref{eq:total-constrained-multimodal-access} as a measure of modal-group-specific total constrained accessibility \(V^{mT}_i\) by summing all \(V^{mT}_{ij}\) for a specific \(i\) and \(m\) (i.e., \(V^{mT}_i = \sum_j V^{mT}_{ij}\)). \(V^{mT}_i\) can also be summed by mode to equal \(V^{T}_i\) (i.e., \(\sum_m V^{mT}_i = V^{T}_i\)) and summed across the region to equal \(D\) (i.e., \(\sum_m\sum_i\sum_j V^{mT}_{ij} = \sum_m\sum_i V^{mT}_{i} = D\)).

And like in the unimodal formulation, \(\kappa_{ij}^{mT}\), a unitless expression of the proportion of \(W^{(2)}_j\) that are allocated to each zone \(i\) for each \(m\), can also be defined for the origin zone \(i\), allowing for the expression of total constrained multimodal accessible opportunities at a zone \(i\) \(V^{mT}_i\) to be equal to \(\kappa_{ij}^{mT}\sum_j W^{(2)}_j\):

\[
\kappa_{ij}^{mT} = \frac{\sum_j D\cdot f(c_{ij})}{\sum_m\sum_i\sum_j W^{(2)}_jf(c_{ij})}
\]

Regarding multimodal market potential, i.e., the \(V_i^{mT}\) transpose of \(i\) and \(j\). \(M^{mT}_{ji}\), the multimodal `market potential' or the population accessible by mode can also be defined using the total constrained formulation, with similar parameters as previously defined.

\begin{equation}
\label{eq:total-constrained-multimodal-market}
M^{mT}_{ji} = \hat K^{mT} \cdot W_i^{(1)} f^m(c^m_{ji})
\end{equation} 

And like in the unimodal formulation of market potential, \(\hat \kappa_{ji}^{mT}\), a unitless expression of the proportion of \(W^{(1)}_i\) that is allocated to each zone \(j\) for each \(m\), can also be defined for the origin zone \(j\), allowing for the expression of total constrained multimodal accessible population at a zone \(j\) \(M^{mT}_j\) to be equal to \(\hat \kappa_j^{mT}\sum_i W^{(1)}_i\):
\[
\hat \kappa_{ji}^{mT} = \frac{\sum_i O \cdot f(c_{ij})}{\sum_m\sum_i\sum_j W^{(1)}_if(c_{ij})}
\]

Returning to the numeric example, we consider all three travel behaviour groups \(m\) together in a single total constrained accessible opportunities calculation. Specifically, each origin \(i\) has three type of travellers, either those traveling at the highest decay of \(f_1 (c_ij)\), or medium decay of \(f_2(c_ij)\), or lowest decay of \(f_3(c_ij)\). In this way, there is only one \(K^{mT}\) for this multimodal system, equal to \(\frac{D}{\sum_m\sum_i\sum_j W^{(2)}_jf^m(c^m_{ij})} = \frac{490}{0.6233422 + 7.283556 + 1108.361}=\) 0.4389629. Note: \(K^{mT}\) is smaller than 1, lower than the majority of higher decay unimodal \(K^T\). This suggests that the numerator (sum of the destination-side marginal, i.e., opportunities) is smaller than the sum of all the unconstrained accessibility \(V^{0}_i\) values for each mode across the system. It can be seen that the modal group travelling at \(f_1 (c_ij)\) contributes a lower amount of unconstrained accessibility (0.6233422 \(\text{physicians-minute}^{-3}\)), compared to the furthest moving lowest decay \(f_3(c_ij)\) group (\texttt{r\ unc\_acc\_ij\$V\_unc\_ij\_3\ \textbar{}\textgreater{}\ sum()} \(\text{physicians-minute}^{-0.1}\)).

Multiplying the system-wide \(K^{mT}\) value by the unconstrained accessibility flows \(V^{0}_{ij}\) for each group yields a \(V^{mT}_{ij}\) value for each \(m\). This value can be summarised for each \(m\) at each \(i\) (\(V^{mT}_{i}\)) and also summarised for each \(i\) (\(\sum_m V^{mT}_{ij}\)) as presented in Table \ref{tab:chp2_simple_example_total_m_opp_access_tab}.

\begin{table}[!t]
\caption{Simple system: total constrained multimodal accessible opportunities.} \label{tab:chp2_simple_example_total_m_opp_access_tab}
\fontsize{7.5pt}{9.0pt}\selectfont
\begin{tabular*}{\linewidth}{@{\extracolsep{\fill}}>{\raggedright\arraybackslash}p{0.1\linewidth}|>{\centering\arraybackslash}p{0.225\linewidth}>{\centering\arraybackslash}p{0.225\linewidth}>{\centering\arraybackslash}p{0.225\linewidth}>{\centering\arraybackslash}p{0.225\linewidth}}
\toprule
& \multicolumn{3}{c}{$V_i^{mT}$} & $\sum_m V_i^{mT}$ \\ 
\cmidrule(lr){2-4} \cmidrule(lr){5-5}
 & $f_1 (c_{ij}) = 1/c_{ij}^3$ & $f_2 (c_{ij}) = 1/c_{ij}^2$ & $f_3 (c_{ij}) = 1/c_{ij}^{0.1}$ & All groups \\
\cmidrule(lr){2-2} \cmidrule(lr){3-3} \cmidrule(lr){4-4} \cmidrule(lr){5-5}
Origin & units: \emph{physicans} & units: \emph{physicans} & units: \emph{physicans} & units: \emph{physicans} \\ 
\midrule\addlinespace[2.5pt]
1 & 0.096 & 1.127 & 162.918 & 164.141 \\ 
2 & 0.074 & 0.863 & 159.554 & 160.490 \\ 
3 & 0.104 & 1.208 & 164.057 & 165.369 \\ 
\midrule 
\midrule 
Sum & 0.2736241 & 3.19721 & 486.5292 & 490 \\ 
\bottomrule
\end{tabular*}
\end{table}



A few important findings can be discerned from Table \ref{tab:chp2_simple_example_total_m_opp_access_tab}; Firstly the sum of all \(V_i^{mT}\) values is equal to the known number of physicians in the system, 490. Secondly, the lowest decay group \(f_3(c_{ij})\) captures the vast majority of accessible opportunities (99\%), demonstrating the significant travel impedance advantage that \(m\)-travelling \(f_3(c_{ij})\) group has \emph{relative} to other \(m\) traveling groups competing in the same system. When all \(m\)s are considered in the same system, the advantage that one impedance offers in proportionally allocating more opportunities becomes more apparent. Thirdly, the consideration of \(m\) within the system allows for the comparison of values between origin \(i\) values but also between traveling group --just like in the unimodal version, but with a different meaning. For instance, zone 3 is allocated the most accessible opportunities out of any other zone for each travel group, just like in Table \ref{tab:chp2_simple_example_total_opp_access_tab}. However, the magnitude of values are different: whereas in the unimodal case \(c_{ij}\) drives the allocation, in the multimodal case both the \(c^m_{ij}\) and the \(f^m()\) drive the allocation. As \(c^m_{ij}\) is the same across travel groups, their rank in allocation does not shift, but their contribution on account of \(f^m()\) changes.

It is important to highlight that a multimodal comparison, i.e., going beyond a post hoc adjustment after calculation, is not possible with unconstrained accessibility, but is possible using constraints. Total constrained accessibility scores can be summed across all \(m\)s (i.e., column 5 in Table \ref{tab:chp2_simple_example_total_m_opp_access_tab}) and represented per \(i\); representing the number of accessible opportunities at that zone considering all travelling groups \(m\).

The multimodal total constrained accessible population (i.e., market potential), can also be represented. Each destination \(j\) is attracting the same three type of travelers \(m\), so \(\hat K^{mT}\) for this multimodal system equals to \(\frac{O}{\sum_m\sum_i\sum_j W^{(1)}_i f^m(c^m_{ij})} = \frac{20}{0.02450548 + 0.2856 + 45.07428}=\) 0.4406802. Note: like \(K^{mT}\), \(\hat K^{mT}\) is smaller than 1 and lower than the majority of higher decay unimodal \(\hat K^T\). This suggests that the origin-side marginal sum of 20 is smaller than the sum of all the unconstrained accessibility \(M^{0}_i\), for each \(m\), across the entire system.

\(M^{mT}_{ij}\) values are summarised for each \(m\) at each \(j\) (\(M^{mT}_{j}\)) and also summarised for each \(j\) (\(\sum_m M^{mT}_{ij}\)) as presented in Table \ref{tab:chp2_simple_example_total_m_pop_market_tab}.

\begin{table}[!t]
\fontsize{7.5pt}{9.0pt}\selectfont
\begin{tabular*}{\linewidth}{@{\extracolsep{\fill}}>{\raggedright\arraybackslash}p{\dimexpr 45.00pt -2\tabcolsep-1.5\arrayrulewidth}|>{\centering\arraybackslash}p{\dimexpr 135.00pt -2\tabcolsep-1.5\arrayrulewidth}>{\centering\arraybackslash}p{\dimexpr 135.00pt -2\tabcolsep-1.5\arrayrulewidth}>{\centering\arraybackslash}p{\dimexpr 135.00pt -2\tabcolsep-1.5\arrayrulewidth}>{\centering\arraybackslash}p{\dimexpr 135.00pt -2\tabcolsep-1.5\arrayrulewidth}}
\toprule
 & \multicolumn{3}{>{\centering\arraybackslash}m{\dimexpr 405.00pt -2\tabcolsep-1.5\arrayrulewidth}}{M\textsubscript{j}\textsuperscript{mT}} & \textbackslash{}textbackslash\{\}sum\textsubscript{m} M\textsubscript{j}\textsuperscript{\{mT\}} \\ 
\cmidrule(lr){2-4} \cmidrule(lr){5-5}
 & f\textsubscript{1} (c\textsubscript{ij}) = 1/c\textsubscript{ij}\textsuperscript{3} & f\textsubscript{2} (c\textsubscript{ij}) = 1/c\textsubscript{ij}\textsuperscript{2} & f\textsubscript{3} (c\textsubscript{ij}) = 1/c\textsubscript{ij}\textsuperscript{0.1} & for all groups \\ 
\cmidrule(lr){2-2} \cmidrule(lr){3-3} \cmidrule(lr){4-4} \cmidrule(lr){5-5}
Destination & units: \emph{population in 10,000s} & units: \emph{population in 10,000s} & units: \emph{population in 10,000s} & units: \emph{population in 10,000s} \\ 
\midrule\addlinespace[2.5pt]
1 & 0.003 & 0.034 & 6.553 & 6.590 \\ 
2 & 0.005 & 0.050 & 6.671 & 6.726 \\ 
3 & 0.003 & 0.041 & 6.639 & 6.684 \\ 
\midrule 
\midrule 
Sum & 0.01079908 & 0.1258583 & 19.86334 & 20 \\ 
\bottomrule
\end{tabular*}
\end{table}



In Table \ref{tab:chp2_simple_example_total_m_pop_market_tab}, similar to Table \ref{tab:chp2_simple_example_total_pop_market_tab}, travel group \(f_3(c_{ij})\) drives the allocation, with the majority of population being allocated from zones assuming such impedance. Population also is allocated most from zone 3, like in the previous results.

\section{Singly constrained accessibility}\label{singly-constrained-accessibility}

Similar to the total constrained accessibility measure, the singly constrained case includes a balancing factor that adjusts the unconstrained zonal accessibility values \(V^0_i\) such that a \emph{single} constraint is satisfied. Two variants are defined: (a) the \emph{singly constrained accessible opportunities} case (an alternative formula to spatial availability in Soukhov et al. (2023)) \(V_{i}^{S}\), and (b) the \emph{singly constrained accessible population} case, its transpose, or singly constrained market potential \(M_{j}^{S}\).

Unlike the total constraint (i.e., Equation \ref{eq:total-constraint-accessibility}), the single constraint--as will be defined in Equation \ref{eq:opportunity-constraint} and Equation \ref{eq:population-constraint} for the first and second variants-- incorporates additional information. In the opportunities-accessible variant, the associated balancing factor constrains the `potential' for spatial interaction by ensuring that only a proportional amount of opportunities at each destination are allocated to `demanding' origins. This allocation is informed by demand, or the relative amount of population, and associated travel impedance connecting the zones. In the population-accessible variant (i.e., the \(i\) -\textgreater{} \(j\) transpose of the first variant), population at each origin is proportionally allocated to destinations based on the share of opportunities and travel impedance.

In both variants, the singly constrained accessibility measure introduces population-based (or opportunity-based) competition at the zonal level, unlike the total constraint, which more simply allocates a fixed regional total of opportunities (or population depending on the variant). In sum, all singly constrained zonal accessibility values are both a proportion of the known regional opportunity (or population) total \emph{and} a sum of a \emph{balanced} proportion of opportunities allocated from each destinations (or population allocated from each origin). Each zonal value remains in units of opportunities accessible (or population accessible).

Following the definition of these two variants, they are extended into their multimodal expressions \(V_{i}^{mS}\) and \(M_{j}^{mS}\), an expression of the value for each travel behaviour group or mode \(m\). In this extension, multimodal travel impedance function \(f^m(c^m_{ij})\) and either origin- or destination- side marginals (\(D_j^m\) or \(O_i^m\)) are defined for each \(m\). This multimodal extension allocates the variant's constrained marginal based on the proportion of travel impedance for that \(m\) relative to all impedance in the region \emph{and} the non-constrained (i.e., other) marginal's mass relative to the total non-constrained marginal mass in the region. The later is not considered in the total constrained multimodal accessibility case. Put another way, the multimodal extension of singly constrained accessibility allocates the opportunities (or population) at the marginal based on the population's productive demand (or opportunity's attractive supply) of each \(m\) as a share of this potential relative to all other \(m\)s in the region.

\subsection{Singly constrained accessible opportunities: spatial availability}\label{singly-constrained-accessible-opportunities-spatial-availability}

To demonstrate this variant formulaically, we begin with the opportunity constraint (Equation \ref{eq:opportunity-constraint}) as our known piece of information. Namely, the sum of accessible opportunities from a destination should equal the number of opportunities \(D_j\) at that destination. As the number of opportunities at each \(j\) are known, it is represented as \(D_j\) instead of \(W_j^(2)\) as in the total constrained case. This constraint should hold for all destinations in the region. This is comparable to the single attraction-constraint (Equation \ref{eq:constraint2-gravitymodel}) from Wilson's framework.

\begin{equation}
\label{eq:opportunity-constraint}
\sum_i V^S_{ij} =  D_j
\end{equation} 

The underlying spatial interaction model is now the attraction-constrained model in Equation \ref{eq:attraction-constrained-gravitymodel}, and our accessibility measure becomes:

\begin{equation}
\label{eq:opportunity-constrained-accessibility}
V^S_{i} = \sum_j B_j D_j W_i^{(1)} f(c_{ij})
\end{equation} 

\noindent where \(W_i^{(1)}\) is a measure of the mass at origin \(i\) (i.e., the opportunity-seeking population). The corresponding balancing factor, as per Wilson, is:

\begin{equation}
\label{eq:opportunity-constrained-proportionality-constants}
B_j = \frac{1}{\sum_i W_i^{(1)} f(c_{ij})}
\end{equation} 

Introducing the balancing factor in Equation \ref{eq:opportunity-constrained-accessibility}, we obtain:

\begin{equation}
\label{eq:opportunity-constrained-accessibility-with-balancing-factor}
V^S_{i} = \sum_j D_j \frac{W_i^{(1)} f(c_{ij})}{\sum_i W_i^{(1)} f(c_{ij})}
\end{equation} 

Further, we define the following proportional allocation factor:

\begin{equation}
\label{eq:opportunity-constrained-proportional-allocation-factor}
\kappa^S_{ij} = \frac{W_i^{(1)} f(c_{ij})}{\sum_i W_i^{(1)} f(c_{ij})}
\end{equation} 

After this, it is possible to rewrite Equation \ref{eq:opportunity-constrained-accessibility-with-balancing-factor} as an origin summary expression of proportionally allocated known opportunities (i.e., \(D_j\)).

\begin{equation}
\label{eq:attraction-constrained-accessibility-with-proportional-allocation-factor}
V^S_{i} = \sum_j \kappa^S_{ij} D_j
\end{equation} 

Soukhov et al. (2023) have shown that the role of \(\kappa^S_{ij}\) is to allocate opportunities \(D_j\) proportionally to the mass at each origin \(i\) and the impedance between \(i\) and \(j\). As in the total constrained opportunity case, \(\kappa^S_{ij}\) is dimensionless and \(V_i^S\) is in the units of opportunities \(D_j\). The singly constrained accessibility measure in Equation \ref{eq:attraction-constrained-accessibility-with-proportional-allocation-factor} is called spatial availability by Soukhov et al. (2023), because it represents the number of opportunities that can be reached \emph{and} are available, in the sense that accessible opportunities have been proportionally allocated based on relative demand, travel impedance and the regional total number of opportunities, i.e., spatial competition for them has been considered. These authors also show that the following expression (accessibility per capita) is a constrained version of the popular two-stage floating catchment area measure of Q. Shen (1998b) and Luo \& Wang (2003a):

\begin{equation}
\label{eq:opportunity-constrained-accessibility-per-capita}
v^S_{i} = \frac{V^S_{i}}{W^{(1)}_i}
\end{equation} 

Returning to the simple numeric example, the opportunity-constrained case would yield the following \(B_{j}\) for \(f_1(c_{ij})\):

\[
\begin{array}{l}
B_{j} = \frac{1}{\sum_i W_i^{(1)} f(c_{ij})}\\
B_{1} =  \frac{1}{\frac{4}{10^3} + \frac{10}{30^3} + \frac{6}{15^3}} = 162.6506\\ 
B_{2} =  \frac{1}{\frac{4}{30^3} + \frac{10}{10^3} + \frac{6}{25^3}} = 94.9474\\
B_{3} =  \frac{1}{\frac{4}{10^3} + \frac{10}{25^3} + \frac{6}{10^3}} = 93.9850
\end{array}
\]

The balancing factors \(B_j\) for the \(f_2(c_{ij})\) decay group for zones 1, 2 and 3 is 12.8571429, 8.7685113 and 10.6635071, and for \(f_3(c_{ij})\) decay group is 0.0672461, 0.0660559 and 0.0663798. Using these these balancing constants, we can calculate the singly constrained opportunity accessibility as presented in Table \ref{tab:chp2_simple_example_singly_opp_access_tab}.

\begin{table}[!t]
\caption{Simple system: singly constrained accessible opportunities} \label{tab:chp2_simple_example_singly_opp_access_tab}
\fontsize{7.5pt}{9.0pt}\selectfont
\begin{tabular*}{\linewidth}{@{\extracolsep{\fill}}>{\raggedright\arraybackslash}p{\dimexpr 36.00pt -2\tabcolsep-1.5\arrayrulewidth}|>{\centering\arraybackslash}p{\dimexpr 108.00pt -2\tabcolsep-1.5\arrayrulewidth}>{\centering\arraybackslash}p{\dimexpr 108.00pt -2\tabcolsep-1.5\arrayrulewidth}>{\centering\arraybackslash}p{\dimexpr 108.00pt -2\tabcolsep-1.5\arrayrulewidth}>{\centering\arraybackslash}p{\dimexpr 108.00pt -2\tabcolsep-1.5\arrayrulewidth}}
\toprule
 &  & \multicolumn{3}{>{\centering\arraybackslash}m{\dimexpr 324.00pt -2\tabcolsep-1.5\arrayrulewidth}}{V\textsubscript{i}\textsuperscript{S}} \\ 
\cmidrule(lr){3-5}
 &  & f\textsubscript{1} (c\textsubscript{ij}) = 1/c\textsubscript{ij}\textsuperscript{3} & f\textsubscript{2} (c\textsubscript{ij}) = 1/c\textsubscript{ij}\textsuperscript{2} & f\textsubscript{3} (c\textsubscript{ij}) = 1/c\textsubscript{ij}\textsuperscript{0.1} \\ 
\cmidrule(lr){3-3} \cmidrule(lr){4-4} \cmidrule(lr){5-5}
Origin & Population (units: \emph{people in 10,000s}) & units: \emph{physicians} & units: \emph{physicians} & units: \emph{physicians} \\ 
\midrule\addlinespace[2.5pt]
1 & 4 & 133.469 & 122.255 & 98.848 \\ 
2 & 10 & 166.781 & 185.096 & 241.877 \\ 
3 & 6 & 189.750 & 182.650 & 149.275 \\ 
\midrule 
\midrule 
Sum & — & 490 & 490 & 490 \\ 
\bottomrule
\end{tabular*}
\end{table}



Imposing the single proportional allocation factor \(\kappa^S_{ij}\) allows for the comparison of differences and ratios of the accessibility values, like previously discussed in the total constrained accessible opportunities case. The proportional allocation factor ensures that resulting values are in units of \emph{physicians}, with the impedance units already accounted for in the allocation process.

However, unlike the total constrained opportunity case, \(\kappa^S_{ij}\) reflects zonal competition based on the mass of the origin, i.e., population. Again, consider the highest decay scenario \(f_1(c_{ij})\). Under this scenario, zone 1 no longer captures a medium amount of physicians as in the total constrained opportunity case: it now captures the fewest in the region i.e., 133.4687282 at zone 1, 166.7813387 at zone 2, and 189.7499331 at zone 3. Why may zone 1 (healthcare cluster at the edge of the city) capture 27\% of the physicians regionally while this same zone captures 35\% in the total opportunity constrained case?

This difference is due to the single-opportunity factor \(\kappa^S_{ij}\)'s role in \(V_i^S\). The only inputs required in the total-opportunity factor \(\kappa^T_{ij}\) is the total number of opportunities in the region \(D\) as well as the associated opportunities at each \(j\) and travel impedance. Opportunities are allocated to \(i\)s, regardless of the mass weights of origin. Whereas the single opportunity constraint \(\kappa^S_{ij}\) requires the population at \(i\) as an input. In fact, \(\kappa^S_{ij}\) is calculated as the proportion of impedance-weighted population at an \(i\) to the sum of impedance-weighted population for the entire region. Hence, since zone 1 has the lowest population in the region, is in close proximity to a more populated zone (zone 3, the `urban core'), and is not well connected (in terms of travel impedance) to other zones with opportunities, \(V_{1}^S\) values, or the number of physicians accessible, is lowest.

Readers may also notice the change in the proportion of opportunities drawn from different zones depending on the travel behaviour scenarios considered. For instance, consider Zone 2 which has the highest population. It is more evident in the \(f_3(c_{ij})\) scenario than in higher decay scenarios that this zone does not have an exceptionally large population for the region - Zone 2 only represents 50\% of the population in the three zone region. In this sense, other zones are not \emph{that} disadvantaged, and in this scenario with unfettered travel cost, Zones 1 and 3 also take opportunities from Zone 2 (i.e., Zone 1 and Zone 3 takes 17\% and 25\% more from Zone 2 between \(f_3(c_{ij})\) and \(f_1(c_{ij})\) scenarios hence \(\kappa_{2,2}^S\) decreases by 42\%). Zones 1 and 3 are allocated opportunities at relates similar to their relative population size.

Readers may also notice the change in the proportion of opportunities drawn from different zones depending on the travel behaviour scenarios considered. For instance, Zone 2 has the highest zonal population, representing 50\% of the 200,000 regional population. However, due to its high relative travel distance from the other zones, its population is less competitive in capturing opportunities in the high-decay travel scenario (\(f_1(c_{ij})\)): Under this scenario, zone 2 captures almost exclusively opportunities from its own zone. However, in \(f_3(c_{ij})\), the scenario with unfettered travel cost, Zone 2 captures by far the most number of physicians. But in this scenario, Zones 1 and 3 also take opportunities from Zone 2 (i.e., Zone 1 and Zone 3 takes 17\% and 25\% more from Zone 2 between \(f_3(c_{ij})\) and \(f_1(c_{ij})\) scenarios hence \(\kappa_{2,2}^S\) decreases by 42\%). Zones 1 and 3 are allocated opportunities at rates similar to their relative population size.

In this way, the consideration of constrained accessibility \emph{per capita} may be clarifying. Often, accessibility values are reported as raw scores without the consideration for population. But, as we introduced constraints, these constrained accessibility values can be normalized using anything that is relevant to the zone. In Table \ref{tab:chp2_simple_example_singly_opp_access_per_capita_tab}, we present per capita accessibility for the numeric example, simply in units of number of physicians accessible per population at each zone. Notably, these per capita rates are equivalent to the 2SFCA values.

\begin{table}[!t]
\caption{Simple system: singly constrained accessible opportunities per capita} \label{tab:chp2_simple_example_singly_opp_access_per_capita_tab}
\fontsize{7.5pt}{9.0pt}\selectfont
\begin{tabular*}{\linewidth}{@{\extracolsep{\fill}}>{\raggedright\arraybackslash}p{\dimexpr 36.00pt -2\tabcolsep-1.5\arrayrulewidth}|>{\centering\arraybackslash}p{\dimexpr 108.00pt -2\tabcolsep-1.5\arrayrulewidth}>{\centering\arraybackslash}p{\dimexpr 108.00pt -2\tabcolsep-1.5\arrayrulewidth}>{\centering\arraybackslash}p{\dimexpr 108.00pt -2\tabcolsep-1.5\arrayrulewidth}>{\centering\arraybackslash}p{\dimexpr 108.00pt -2\tabcolsep-1.5\arrayrulewidth}}
\toprule
 &  & \multicolumn{3}{>{\centering\arraybackslash}m{\dimexpr 324.00pt -2\tabcolsep-1.5\arrayrulewidth}}{v\textsubscript{i}\textsuperscript{S}} \\ 
\cmidrule(lr){3-5}
 &  & f\textsubscript{1} (c\textsubscript{ij}) = 1/c\textsubscript{ij}\textsuperscript{3} & f\textsubscript{2} (c\textsubscript{ij}) = 1/c\textsubscript{ij}\textsuperscript{2} & f\textsubscript{3} (c\textsubscript{ij}) = 1/c\textsubscript{ij}\textsuperscript{0.1} \\ 
\cmidrule(lr){3-3} \cmidrule(lr){4-4} \cmidrule(lr){5-5}
Origin & Population (units: \emph{people in 10,000s}) & units: \emph{physicians per capita} & units: \emph{physicians per capita} & units: \emph{physicians per capita} \\ 
\midrule\addlinespace[2.5pt]
1 & 4 & 33.367 & 30.564 & 24.712 \\ 
2 & 10 & 16.678 & 18.510 & 24.188 \\ 
3 & 6 & 31.625 & 30.442 & 24.879 \\ 
\bottomrule
\end{tabular*}
\end{table}



This simple example was constructed so that the regional average equals 24.5 physicians per 10,000 people. As distance decay decreases and becomes \emph{relatively} uniform (all zones can reach all zones), the effect of population drives the proportional allocation of opportunities. Consequently, per capita accessibility values begin to stabilise to the regional per capita average (e.g., in the lowest distance decay \(f_3(c_{ij})\), per capita values are all around 24 physicians accessible per capita).

This trend mirrors how the accessibility values in the total constrained opportunity case stabilises to the \(V_i^T\) regional average (e.g., the accessible opportunities allocated to each of the three zones approaches a third of 490 physicians, or 163.33, under the unfettered mobility scenario \(f_3(c_{ij})\)). These patterns make intuitive sense: the balancing factors act as regional and/or zonal averaging mechanisms. In distance decay travel behaviour scenarios that are more relatively uniform (i.e., low for all zones like in \(f_3(c_{ij})\)), what remains is the relative effect of the other variables in the balancing factor. In the total constrained case, this is the proportion of opportunities relative to the regional opportunities, and in the case of the single opportunity constrained case, this is the population at a zone relative to the regional population.

\subsection{{[}NEW EXAMPLE SOLVED{]} Singly constrained accessible population: market availability}\label{new-example-solved-singly-constrained-accessible-population-market-availability}

Similar to Equation \ref{eq:total-population-balancing-factor} in transposing the origins and destinations, we can define a \emph{singly constrained} measure of market potential that preserves the known population (i.e., the mass weight at the origin \(W_i^{(1)}\) is now represented by \(O_i\)). In it's per-capita expression, i.e., equivalent to 2SFCA, this constrained concept of market potential been used to express ``facility crowdedness'' as in F. H. Wang (2018).

The underlying spatial interaction model is now the production-constrained model in Equation \ref{eq:production-constrained-gravitymodel}, and our market potential measure \(M^S_j\) becomes:

\begin{equation}
\label{eq:population-constrained-accessibility}
M^S_j = \sum_i A_i O_i W_j^{(2)} f(c_{ij})
\end{equation} 

In this case, the measure is singly constrained by the population \emph{by origin} (i.e., \(O_i\)), like Equation \ref{eq:constraint2-gravitymodel} from Wilson's framework:

\begin{equation}
\label{eq:population-constraint}
\sum_j M^S_{ji} =  O_i \\
\end{equation} 

And the corresponding balancing factor, as per Wilson, is:

\begin{equation}
\label{eq:population-constrained-proportionality-constants}
A_i = \frac{1}{\sum_j W_j^{(2)} f(c_{ij})}
\end{equation} 

Following the same logic as in the preceding section on total constrained market potential, one arrives at the following expression:

\begin{equation}
\label{eq:production-constrained-accessibility-with-proportional-allocation-factor}
M^S_{j} = \sum_i \hat \kappa^S_{ji} O_i
\end{equation} 

\noindent with:

\begin{equation}
\label{eq:attraction-constrained-proportional-allocation-factor}
\kappa^S_{ji} = \frac{W_j^{(2)} f(c_{ij})}{\sum_i W_j^{(2)} f(c_{ij})}
\end{equation} 

As well, the single (population) constraint in Equation \ref{eq:population-constraint} ensures that the the total constraint (e.g., \(\sum_j M^S_{j} = \sum_i\sum_j  M^S_{ji} = O\)) is maintained.

With these constraints, \(\frac{M_j^S}{O}\) can be interpreted as the proportion of the total population serviced by location \(j\).

Solving the numeric example, the balancing factors \(A_i\) for zones 1, 2 and 3 for the highest decay group \(f_1(c_{ij})\) is: 4.5685279, 5.9720772 and 4.2192774. For \(f_2(c_{ij})\) decay group is: 0.3896104, 0.5087045 and 0.3634895. And for the lowest decay group \(f_3(c_{ij})\) is: 0.0026944, 0.0027512 and 0.0026757. Using these these balancing constants, we calculate the singly constrained accessible population:

\begin{table}[!t]
\caption{Simple system: singly constrained accessible population} \label{tab:chp2_simple_example_singly_pop_market_tab}
\fontsize{7.5pt}{9.0pt}\selectfont
\begin{tabular*}{\linewidth}{@{\extracolsep{\fill}}>{\raggedright\arraybackslash}p{\dimexpr 36.00pt -2\tabcolsep-1.5\arrayrulewidth}|>{\centering\arraybackslash}p{\dimexpr 108.00pt -2\tabcolsep-1.5\arrayrulewidth}>{\centering\arraybackslash}p{\dimexpr 108.00pt -2\tabcolsep-1.5\arrayrulewidth}>{\centering\arraybackslash}p{\dimexpr 108.00pt -2\tabcolsep-1.5\arrayrulewidth}>{\centering\arraybackslash}p{\dimexpr 108.00pt -2\tabcolsep-1.5\arrayrulewidth}}
\toprule
 &  & \multicolumn{3}{>{\centering\arraybackslash}m{\dimexpr 324.00pt -2\tabcolsep-1.5\arrayrulewidth}}{M\textsubscript{j}\textsuperscript{S}} \\ 
\cmidrule(lr){3-5}
 &  & f\textsubscript{1} (c\textsubscript{ij}) = 1/c\textsubscript{ij}\textsuperscript{3} & f\textsubscript{2} (c\textsubscript{ij}) = 1/c\textsubscript{ij}\textsuperscript{2} & f\textsubscript{3} (c\textsubscript{ij}) = 1/c\textsubscript{ij}\textsuperscript{0.1} \\ 
\cmidrule(lr){3-3} \cmidrule(lr){4-4} \cmidrule(lr){5-5}
Dest. & Physicans & units: \emph{people in 10,000s} & units: \emph{people in 10,000s} & units: \emph{people in 10,000s} \\ 
\midrule\addlinespace[2.5pt]
1 & 160 & 4.478 & 4.949 & 6.462 \\ 
2 & 150 & 9.303 & 8.414 & 6.174 \\ 
3 & 180 & 6.219 & 6.638 & 7.364 \\ 
\midrule 
\midrule 
Sum & — & 20 & 20 & 20 \\ 
\bottomrule
\end{tabular*}
\end{table}



Similar to the total constrained case (Table \ref{tab:chp2_simple_example_total_pop_market_tab}), the singly constrained accessible population (Table \ref{tab:chp2_simple_example_singly_pop_market_tab}) shows that Zone 1 attracts the fewest people under both high travel cost (\(f_1(c_{ij})\)) and medium decay (\(f_2(c_{ij})\)) scenarios. This is largely due to its small local population (4 out of 20 units in the region), and the system's lower ability for other zones to attract away or attract population from other zones). However, under low decay (\(f_3(c_{ij})\)), where people travel more freely, Zone 1 surpasses Zone 2 in attracting population. Despite having only a medium level of physician supply, Zone 1's central location--close to both the urban core (Zone 3) and `bedroom community' Zone 2--makes it more competitive in low decay scenarios. In contrast, Zone 2 has the largest local population but the fewest physicians, and its relative isolation boosts its attractiveness only when travel is highly restricted (i.e., under \(f_1(c_{ij})\)). The difference in results between Zones 1 and 2 across scenarios highlights that in the singly constrained accessible population measure, both travel costs (cost effect) and the supply of physicians (mass effect) influence population allocation. Furthermore, these values could be expressed as the rate of accessible population per opportunity (\(m_j^S\)), aligning with the per capita logic of \(v_i^S\).

\subsection{{[}NEW{]} Multimodal extension of spatial availability and market availability}\label{new-multimodal-extension-of-spatial-availability-and-market-availability}

\(V^{S}_i\) and \(M^{S}_j\) can be extended to explicitly incorporate multiple modes (or travel groups) \(m\) like established in the total constrained accessibility case. This extension is relevant if: 1) the singly constrained measure is pertinent, i.e., one has information about both marginals and an intuition that the to-be-constrained marginal should be allocated proportionally based on the matching marginal and associated travel impedance. 2) the analyst is interested in accounting for different travel behaviour groups \(m\) within the same system. In this way, the multimodal extension of the singly constrained accessibility measure is allocates the constrained marginal based on the mass effect (of the matching marginal) and the travel cost effect along multiple \(m\) in the system. The singly constrained multimodal accessible opportunities at each \(i\) from \(j\) for each group \(m\) can be expressed as follows:

\begin{equation}
\label{eq:singly-constrained-multimodal-accessibility}
V^{mS}_{ij} = B_j^{m} D_j W_i^{m(1)} f^m(c^m_{ij})
\end{equation} 

Where the mass at the origin \(W_i^{m(1)}\) corresponds to each \(m\) group and the associated multimodal balancing factor equals \(B_j = \frac{1}{\sum_m\sum_i W_i^{m(1)} f^m(c^m_{ij})}\). \(V^{mS}_{ij}\) can be aggregated and expressed as accessible opportunities for each \(m\) at \(i\) \(V^{mS}_{i} = \sum_j B_j D_j W_i^{m(1)} f^m(c^m_{ij})\), and even for each \(i\) (across \(m\)s) by summing all \(m\)s (\(\sum_m V^{mS}_{i}\)). These aggregations remain balanced, as all aggregation of \(ij\) values ensure the opportunity-side single constraint is maintained (i.e., \(\sum_m\sum_i\sum_j V^{mS}_{ij} = \sum_m\sum_i V^{mS}_{i} = D\) as well as \(\sum_m \sum_i V^{mS}_{ij} =  D_j\)).

Like in the unimodal singly constrained accessibility formulation, \(\kappa_{ij}^{mS}\) a unitless expression of the proportion of \(D_j\) that is allocated to each zone \(i\) for each \(m\). It can also be defined for the origin zone \(i\), allowing for the expression of singly constrained multimodal accessible opportunities at a zone \(i\) to be \(V^{mS}_i = \sum_j \kappa_{ij}^{mS} D_j\), where \(\kappa_{ij}^{mS}\) is:

\begin{equation}
\label{eq:multimodal-opportunity-constrained-proportional-allocation-factor}
\kappa^{mS}_{ij} = \frac{W_i^{m(1)} f^m(c^m_{ij})}{\sum_m\sum_i W_i^{m(1)} f^m(c^m_{ij})}
\end{equation} 

Regarding the multimodal market potential variant, i.e., the \(V_i^{mS}\) transpose of \(i\) and \(j\). \(M^{mS}_{ji}\), the multimodal accessible population from opportunity types \(m\), and can also be defined for each \(j\) to \(i\) flow as:

\begin{equation}
\label{eq:singly-constrained-multimodal-market}
M^{mS}_{ji} = A_i^{m} O_i W_j^{m(2)} f^m(c^m_{ij})
\end{equation} 

And like in the unimodal formulation of market potential, the multimodal balancing factor \(A_i^{m} = \frac{1}{\sum_j W_j^{m(2)} f^m(c^m_{ij})}\) can be defined as a unitless factor \(\hat \kappa^{mS}_{ji}\) that serves to proportionally allocates the marginal populations at each origin \(O_i\) to each zone \(j\) for each \(m\). \(\hat \kappa^{mS}_{ji}\) allows for \(M^{mT}_j\) to be equal to \(\sum_i \hat \kappa_{ji}^{mS} O_i\) where:
\[
\hat \kappa_{ji}^{mS} = \frac{W_j^{m(2)} f^m(c^m_{ij})}{\sum_m\sum_j W_j^{m(2)} f^m(c^m_{ij})}
\]

Returning to the numeric example, we need two additional pieces of information: the proportion of population (i.e., origin mass) that travel by each travel impedance scenario for the destination-constrained case (singly constrained accessible opportunities) and the proportion of the opportunities (i.e., destination mass) that are reached by each travel impedance scenario for the origin-constrained case (singly constrained accessible population).

For the singly constrained multimodal accessible opportunities case. We presume that the in Zone 3, the split between \(f_1(c_{ij})\), \(f_2(c_{ij})\), and \(f_3(c_{ij})\) is 70:20:10--a zone in the urban centre with the majority of people travelling locally, with only 30\% of the population traveling further and even further distances. For Zone 1--on the edge of the urban core-- the split is 50:30:20, more people travel from further distances. For Zone 2--the more distance bedroom community--people tend to travel further distances, with a split of 20:30:50.

Beginning with the singly constrained multimodal accessible opportunities case. Each destination is associated with a calculated \(B_j^m\): 0.2138794, 0.1995429 and 0.2110702 for zones 1, 2 and 3 respectively. Note, these values are larger than the unimodal \(B_j\) for the lowest decay group (i.e., mean of 0.0665606 across the \(j\)s), but not by much compared to the unimodal \(B_j\) for \(f_1(c_{ij}\) (mean of 128.4634991) and \(f_2(c_{ij}\) (mean of 10.7630538). This is notable since the balancing factor directly reflects how potentially energetic the system is over all (i.e, the denominator of \(B_j\)) relative to what's being attracted to the zone (the numerator). We know all \(m\) are considered, and proportions of the population (mass at origins) demand for opportunities based on each one of the \(m\)s, zones with a higher proportion of `energetic' population (i.e., share of \(f_3(c_{ij})\) in our case) will be allocated a higher proportion of opportunities.

In fact, we can examine the sum of \(\kappa_{ij}^m\) values for the proportion that is allocated from each \(j\) to the \(m = f_3(c_{ij})\) populations at each \(i\). As \(f_3(c_{ij}) = 1/c_{ij}^0.1\), exceptionally less steep than \(= 1/c_{ij}^2\) and \(= 1/c_{ij}^3\), each destination allocates practically \emph{all} (99\%) of its opportunities to populations at these zones. The singly constrained opportunity accessibility for each \(m\) group is demonstrated accordingly Table \ref{tab:chp2_simple_example_singly_m_opp_access_tab}.

\begin{table}[!t]
\caption{Simple system: singly constrained multimodal accessible opportunities. Destination Zone 1, 2, and 3 have a (50:30:20), (20:30:50) and (70:20:10) split in $m$ respectively.} \label{tab:chp2_simple_example_singly_m_opp_access_tab}
\fontsize{7.2pt}{8.6pt}\selectfont
\begin{tabular*}{\linewidth}{@{\extracolsep{\fill}}>{\raggedright\arraybackslash}p{\dimexpr 45.00pt -2\tabcolsep-1.5\arrayrulewidth}|>{\centering\arraybackslash}p{\dimexpr 101.25pt -2\tabcolsep-1.5\arrayrulewidth}>{\centering\arraybackslash}p{\dimexpr 101.25pt -2\tabcolsep-1.5\arrayrulewidth}>{\centering\arraybackslash}p{\dimexpr 101.25pt -2\tabcolsep-1.5\arrayrulewidth}>{\centering\arraybackslash}p{\dimexpr 101.25pt -2\tabcolsep-1.5\arrayrulewidth}}
\toprule
 & \multicolumn{3}{>{\centering\arraybackslash}m{\dimexpr 303.75pt -2\tabcolsep-1.5\arrayrulewidth}}{$V_{i}^{mS}$} & $\sum_m V_{i}^{mS}$ \\ 
\cmidrule(lr){2-4} \cmidrule(lr){5-5}
 & $f_1(c_{ij}) = \frac{1}{c_{ij}^3}$ & $f_2(c_{ij}) = \frac{1}{c_{ij}^2}$ & $f_3(c_{ij}) = \frac{1}{c_{ij}^{0.1}}$ & For all groups \\ 
\cmidrule(lr){2-2} \cmidrule(lr){3-3} \cmidrule(lr){4-4} \cmidrule(lr){5-5}
Origin & units: physicians & units: physicians & units: physicians & units: physicians \\ 
\midrule\addlinespace[2.5pt]
1 & 0.093 & 0.653 & 61.971 & 62.717 \\ 
2 & 0.067 & 1.194 & 378.330 & 379.592 \\ 
3 & 0.210 & 0.696 & 46.785 & 47.691 \\ 
\midrule 
\midrule 
Sum & 0.3706339 & 2.543451 & 487.0859 & 490 \\ 
\bottomrule
\end{tabular*}
\end{table}



As can be observed in Table \ref{tab:chp2_simple_example_singly_m_opp_access_tab}, indeed \(f_3(c_{ij})\) group is allocated about 99\% of the total 490 opportunities. This \emph{may} be reasonable if the \(m\)-origin-mass split at each \(i\) was predominately \(f_3(c_{ij})\), but recall, it is assumed in Zone 1, 2 and 3 the split is 20\% (of the 40,000), 50\% (of the 100,000) and 10\% (of the 60,000); meaning 118,000 of the 200,000 people in the system (or 59\%) capture 99\% of the accessible opportunities. Put another way, the \(f_3(c_{ij})\) group has access to a mean 77.0347963 physicians per 10,000 capita while \(f_2(c_{ij})\) and \(f_3(c_{ij})\) only have access to 0.5074515 and 0.0434215 physicians per capita (recall, the system average (and Canada's average PPR (WHO, 2025)) is 24.5 physicians per 10,000 population). In our simple example: the cost effect appears to be significantly more influential than the impact of the population mass.

From the perspective of singly constrained multimodal market potential, or market \emph{availability} we can ask ``what is the amount of potential physician-seeking population that is accessible at a destination, considering three different travel behaviours to physicians?''. As this case is singly-constrained on the side of the population allocation, we don't know how they travel. However, we do know the split in how attractive opportunities are, hence the proportion of \(m\) that accesses them. Let's assume that the destination zone 1--on the edge of the urban core--has plenty of specialist physicans, catering to people all over the region: hence a split of 10:20:70 between \(f_1(c_{ij})\), \(f_2(c_{ij})\), and \(f_3(c_{ij})\) is assumed. Zone 2--the more distance bedroom community--contains physicans that are mostly used locally (e.g., general practitioners) that aren't necessarily attractive to the full region, so the split is opposite of Zone 1, at 70:20:10. Lastly, Zone 3--a zone in the urban center-- is presumed to contain an equal mix: a third for each \(m\).

Each origin is associated with a calculated \(A_i^m\): 0.0068463, 0.0073718 and 0.006905 for zones 1, 2 and 3 respectively. Like in the case of Table \ref{tab:chp2_simple_example_singly_m_opp_access_tab}, these balancing factors are exceptionally low: matching the lower magnitude values of \(A_i\) in the unimodal case. Using these balancing factors to calculate the multimodal market availability in Table \ref{tab:chp2_simple_example_singly_m_pop_market_tab}, similar trends as in Table \ref{tab:chp2_simple_example_singly_m_opp_access_tab} can be observed. Namely: the physicians that are assumed to be the most attractive (i.e., to overcome spatial separation by the low decay of \(f_3(c_{ij})\) group) are allocated the majority of `potential' population.

\begin{table}[!t]
\caption{Simple system: singly constrained multimodal accessible population. Destination Zone 1, 2, and 3 have a (10:20:70), (70:20:10) and (33:33:33) split in $m$ respectively.} \label{tab:chp2_simple_example_singly_m_pop_market_tab}
\fontsize{7.2pt}{8.6pt}\selectfont
\begin{tabular*}{\linewidth}{@{\extracolsep{\fill}}>{\raggedright\arraybackslash}p{\dimexpr 45.00pt -2\tabcolsep-1.5\arrayrulewidth}|>{\centering\arraybackslash}p{\dimexpr 101.25pt -2\tabcolsep-1.5\arrayrulewidth}>{\centering\arraybackslash}p{\dimexpr 101.25pt -2\tabcolsep-1.5\arrayrulewidth}>{\centering\arraybackslash}p{\dimexpr 101.25pt -2\tabcolsep-1.5\arrayrulewidth}>{\centering\arraybackslash}p{\dimexpr 101.25pt -2\tabcolsep-1.5\arrayrulewidth}}
\toprule
 & \multicolumn{3}{>{\centering\arraybackslash}m{\dimexpr 303.75pt -2\tabcolsep-1.5\arrayrulewidth}}{$M_{j}^{mS}$} & $\sum_m M_{j}^{mS}$ \\ 
\cmidrule(lr){2-4} \cmidrule(lr){5-5}
 & $f_1(c_{ij}) = \frac{1}{c_{ij}^3}$ & $f_2(c_{ij}) = \frac{1}{c_{ij}^2}$ & $f_3(c_{ij}) = \frac{1}{c_{ij}^{0.1}}$ & For all groups \\ 
\cmidrule(lr){2-2} \cmidrule(lr){3-3} \cmidrule(lr){4-4} \cmidrule(lr){5-5}
Dest. & units: people in 10,000s & units: people in 10,000s & units: people in 10,000s & units: people in 10,000s \\ 
\midrule\addlinespace[2.5pt]
1 & 0.001 & 0.017 & 11.852 & 11.870 \\ 
2 & 0.008 & 0.025 & 1.621 & 1.654 \\ 
3 & 0.003 & 0.039 & 6.434 & 6.476 \\ 
\midrule 
\midrule 
Sum & 0.01205932 & 0.08153133 & 19.90641 & 20 \\ 
\bottomrule
\end{tabular*}
\end{table}



This case (Table \ref{tab:chp2_simple_example_singly_m_pop_market_tab}) also makes the role of the single constraint more conceptually clear: though the majority of the population is allocated to destination zones that contain \(f_3(c_{ij})\)-attractive physicians, those zones may not be able to satisfy this demand. For instance, physicians in the \(f_3(c_{ij})\)-group attract, on average, 1070.3984037 people per physician --- significantly more than the regional benchmark of \(\frac{10,000}{24.5} \approx 408\) people per physician. In contrast, \(f_2(c_{ij})\) and \(f_1(c_{ij})\) physicians attract only 6.759189 and 0.5801247 people per physician, respectively. This difference illustrates how population is allocated based on the availability of physicians and the impedance-weighted attractiveness of those opportunities, as defined by the travel cost function associated with each \(m\) subset of opportunities.

However, this allocation captures \emph{potential}, as information to adjust/reflect the actual capacity or realized demand is not used. For example, while 187 of the 490 physicians (about 38\%) are assumed to be \(f_3(c_{ij})\)-attractive physicians, this may not mean those physicians \emph{will} or \emph{do} serve a similar share of the population. We are only estimating which opportunities people could access, not which they actually do. Because this measure applies a single constraint (of the marginal population), the total `demand' is fixed at the origin, i.e., \(M_{ij}^m\) values always sum to the population at each origin \(i\). Take Zone 1, located at the edge of the urban core --- while it has a residential population of 60,000, its potential accessible population is \ensuremath{1.1869629\times 10^{5}} people. This value is higher than the actual population: mostly because the majority of population from other origins are allocated to this zone due to teh concentration of opportunities (i.e., 70\% of physicians in the zone are \(f_3(c_{ij})\)-attractive). The high market availability of Zone 1 reflects the attractiveness of this zone -- not actual utilization.

In summary, the singly constrained cases describes where populations may go (or where opportunities are allocated, in the opportunity-constrained case), given the spatial distribution of the other marginal and travel impedance. It does not also allocate based on the \emph{other} marginal. To answer this motivation, both population (demand) and opportunity (supply) marginals must be imposed simultaneously. This logic is the foundation of the doubly constrained measure, which matches demand to supply and reveals \emph{realized} accessibility flows or simply \emph{access}.

\section{Doubly constrained accessibility}\label{doubly-constrained-accessibility}

This accessibility case requires zonal populations and opportunities to match one-to-one, like the doubly constrained spatial interaction model. In this way, doubly constrained accessibility can be thought as ``access'' or simply spatial interaction (no potential).

To contextualize this point, the total and singly constrained accessibility measures discussed thus far have used either \(O_i\), \(D_j\), or the regional sums of either, but never both simultaneously. For example, when opportunities \(D_j\) are used to constrain Equation \ref{eq:opportunity-constrained-accessibility} in the \emph{singly constrained accessible opportunities measure}, the specific mass of the population at origin \(i\) demanding only those opportunities at that \(j\) is unknown. Instead, only the population at \(i\) demanding opportunities in the region is known, and this is more generally represented as \(W_i^{(1)}\) (as in Equation \ref{eq:opportunity-constrained-proportionality-constants}). Similarly, when the population \(O_i\) is used as a constraint in Equation \ref{eq:population-constrained-accessibility} in the \emph{singly constrained accessible population measure}, the mass at the destination is given by \(W_j^{(2)}\) (also in Equation \ref{eq:population-constrained-proportionality-constants}) since only the mass of opportunities at each \(j\) is known and information on what opportunities are allocated to what zone is unknown.

By contrast, the double-constrained accessibility case requires that both populations and opportunities match. Meaning, opportunities at each destination can be accessed by all populations, while at the same time, the population at each origin can be accessed by all opportunities. However, this requirement is often unintuitive in traditional accessibility analysis. Namely, the distinction between population (origin masses) and opportunities (destination masses) typically represent different entities without a shared unit of measurement. On the population side, we usually count people; on the opportunity side, we may be referring to physicians, clinics, grocery stores, schools, parks, or libraries. In a few cases, a one-to-one correspondence or their own capacities at which they interact and are interacted with may exist e.g., one person with one job. Another similar opportunity type example is healthcare, e.g., one person and one unit of capacity, e.g.~a vaccine shot.

A doubly constrained approach to accessibility calculation needs a one-to-one relationships between population and opportunities to be present. Mathematically, this model requires the simultaneous imposition of both the population- and opportunity- constraints in the preceding singly constrained variants (Equation \ref{eq:opportunity-constraint} and Equation \ref{eq:population-constraint}), namely the sum of population in all origins should match the sum of opportunities in all destinations (Equation \ref{eq:opportunity-population-equality}):

\begin{equation}
\label{eq:opportunity-population-equality}
\sum_i O_i = \sum_j D_j
\end{equation} 

As before, the simultaneous imposition of both constraints ensures the total system constraint is maintained i.e., the sum of all doubly constrained accessibility values \(\sum_i V^D_{i} = \sum_i\sum_j  V^D_{ij} = D\) remains equal to the total number of opportunities in the region \(O\) as shown in Equation \ref{eq:total-constraint-accessibility}.

As the doubly constrained accessibility measure \(V_{ij}^D\) takes the form of the production-attraction (doubly constrained) spatial interaction model, as shown in Equation \ref{eq:doubly-constrained-gravitymodel}, \(V_{i}^D\) is as follows:

\begin{equation}
\label{eq:doubly-constrained-accessibility}
V_{ij}^D = A_i B_j O_i D_j f(c_{ij})
\end{equation} 

\noindent where the corresponding balancing factors \(A_i\) and \(B_j\), as per Wilson, are:

\[
\begin{array}{l}
A_i = \frac{1}{\sum_j B_j D_j f(c_{ij})}\\
B_j = \frac{1}{\sum_i A_i O_i f(c_{ij})}
\end{array}
\]

Calibration of the two sets of proportionality constants is accomplished by means of iterative proportional fitting, whereby the values of \(A_i\) are initialized as one for all i to obtain an initial estimate of \(B_j\). The values of \(B_j\) are used to update the underlying \(V_{ij}^D\) matrix, before calibrating \(A_i\). This process continues to update \(A_i\) and \(B_j\) until a convergence criterion is met (see Ortúzar \& Willumsen, 2011, pp. 193--195). The proportional allocation factor \(\kappa_{ij}^D\) would then be:

\[
\kappa_{ij}^D = \sum_j \frac{1}{\sum_j B_j D_j f(c_{ij})} \frac{1}{\sum_i A_i O_i f(c_{ij})} O_i f(c_{ij})
\]

One could rewrite Equation \ref{eq:doubly-constrained-accessibility} as an origin summary expression of proportionally allocated opportunities:

\begin{equation}
\label{eq:doubly-constrained-accessibility-w-paf}
V^D_{i} = \sum_j \kappa^D_{ij}  D_j
\end{equation} 

However, \(V^D_{i}\) is not wholly helpful, as it will simply equal the origin-mass marginal. In this way, only \(V^D{ij}\) values are interpretable. Furthermore, unlike in the total and singly constrained cases, the doubly-constrained case does not have an interpretable per capita version. For instance, representing \(V_{ij}^D\) per capita is not meaningful, as the value \emph{already} matches population to opportunities. Following this logic, the market potential form \(M^D_{ij}\) is effectively equivalent to \(V_{ij}^D\), but can be read with a different interpretation: i.e., the opportunities accessed from \(j\) at an \(i\) vs.~the population accessed from \(i\) at a \(j\). The inputs of `opportunities accessed' and `accessed population' can already be interpreted as inherently being sensitive to both opportunities and population.

To calculate doubly constrained accessibility, the interpretation of the population data and the counts of the opportunity data in the numeric example must be modified. Namely, a count of physician \emph{capacity} per destination is needed instead of just the number of physicians, as used to calculated total and singly constrained cases. We also must be able to clearly state that the population is a count of people seeking opportunities at the new capacities, i.e., the population must reflect the \emph{capacity} of the population to interact with opportunities.

So, this adjusted simple example is summarised in Table \ref{tab:chp2_small_system_land_use_doubly_cons_tab}: with the population (in units of 10,000s of people seeking physicians) and the opportunities (in units of 10,00s of physician-capacity) per zone. For the population, we leave this unchanged numerically but theoretically know that each person interacts with one physician capacity (i.e., our opportunities with a \emph{capacity}). Hence, the number of opportunities per destination is new: the example is modified such that the physician-capacity at each zone is an approximately scaled version of the number of destination-side physicians at each zone from the unmodified example (Table \ref{tab:chp2_small_system_land_use_tab}). To emphasis the new definition of `provider' as physician capacity, the new system PPR is simply 1, this is compared to the unmodified example which yields system PPR of24.5. We keep the same zonal cost matrix, and travel impedance functions for three types of travel behaviour as before (Table \ref {tab:chp2_small_system_land_use_cost_tab} and Equation \ref{eq:travel-behaviour-scenarios}).

\begin{table}[!t]
\caption{Modified simple system with three zones reflecting matched population and opportunities. Population is in 10,000 persons and opportunities in 10,000 of physician-capacity.} \label{tab:chp2_small_system_land_use_doubly_cons_tab}
\fontsize{7.5pt}{9.0pt}\selectfont
\begin{tabular*}{\linewidth}{@{\extracolsep{\fill}}rcc}
\toprule
ID (i or j) & Population & Opportunities \\ 
\midrule\addlinespace[2.5pt]
1 & 4 & 7 \\ 
2 & 10 & 5 \\ 
3 & 6 & 8 \\ 
\bottomrule
\end{tabular*}
\end{table}



However, despite the modifications to the example, our objective remains the same as in previous cases: to measure accessibility under different travel behavior scenarios. Specifically, we aim to quantify the number of \emph{potential} spatial interactions between the physician-seeking population in each zone \(i\) to the physician capacity in a zone \(j\) i.e., \(V_{ij}^D\). The highest decay travel behaviour scenario (\(f_1(c_ij)\)) is presented in Table \ref{tab:chp2_adjusted_small_system_land_use_doubly_cons_f1cij_access_tab}.

\begin{table}[!t]
\caption{Doubly constrained opportunity accessibility for the travel behaviour group with the highest travel decay in the modified simple system.} \label{tab:chp2_adjusted_small_system_land_use_doubly_cons_f1cij_access_tab}
\fontsize{7.5pt}{9.0pt}\selectfont
\begin{tabular*}{\linewidth}{@{\extracolsep{\fill}}l|rcccc}
\toprule
 &  & \multicolumn{3}{c}{Destination ID} &  \\ 
\cmidrule(lr){3-5}
 & Origin ID & 1 & 2 & 3 & sum \\ 
\midrule\addlinespace[2.5pt]
 & 1 & 3.235859 & 0.01032226 & 0.7556568 & 4 \\ 
 & 2 & 2.132602 & 4.95932483 & 2.9044391 & 10 \\ 
 & 3 & 1.631539 & 0.03035291 & 4.3399040 & 6 \\ 
\midrule 
\midrule 
Sum & — & 7 & 5 & 8 & — \\ 
\bottomrule
\end{tabular*}
\end{table}



As mentioned, accessibility values are typically interpreted as a summary of the proportionally allocated opportunities at each \(i\). Hence, in interpreting the doubly constrained accessibility from Table \ref{tab:chp2_adjusted_small_system_land_use_doubly_cons_f1cij_access_tab}, \(V_i^D\) values (i.e., the sum of values at all three \(j\) destinations for each origin \(i\)) would be 4.0018381, 9.9963656, and 6.0017963 physician-capacity accessible for Zones 1, 2 and 3 respectively. This approximately equal to the number of population at each of these zones. Conversely, the market potential \(M_j^D\) interpretation of these values would be 7, 5, and 8 people accessible from Zones 1, 2 and 3 respectively, equal to the number of \emph{opportunities} (physician-capacities accessible) at each of these zones. Notice, the mass weight at the origin equals the mass weight of the destination: this is precisely the function of the double constraint. In other words, \(V_i^D\) is the number of accessed opportunities and \(M_j^D\) is the number of population accessed. For the other two travel behaviour, identical \(V_i^D\) and \(M_j^D\) values are calculated, following the same logic. Hence, the usefulness of the doubly constrained measure lies in the interpretation as \(V_{ij}^D\) values.

For instance, differences in \(V_{ij}^D\) values between travel behaviour scenarios are notable. These values can be directly compared to discuss mass and distance decay impacts. Examining Zone 2 (the bedroom community), Table \ref{tab:chp2_adjusted_small_system_land_use_doubly_cons_allfs_zone2_access_tab} demonstrates these \(i\) to \(j\) access values for this more relatively remote, higher-populated and lower-opportunity rich zone. It can be observed that the number of intrazonal opportunities proportionally allocated decreases as the assumed distance decay decreases e.g., from 4.9593248 to 2.6672837 out of the \textasciitilde10 opportunities allocated to Zone 2 (a population of 10). Following the intuition discussed in the singly constrained opportunity case, as decay decreases, the mass effect of the population (at origin) and opportunity (at destination) is more evident: zonal opportunities are supplied and zonal populations demand at the weights assigned to these zones, with minimal decay adjustment, reflected by the proportional allocation factors \(\kappa_{ij}^D\).

\begin{table}[!t]

\caption{Doubly constrained opportunity accessibility for all travel behaviour groups at Zone 2 in the modified simple system.} \label{tab:chp2_adjusted_small_system_land_use_doubly_cons_allfs_zone2_access_tab}
\fontsize{7.5pt}{9.0pt}\selectfont
\begin{tabular*}{\linewidth}{@{\extracolsep{\fill}}>{\raggedright\arraybackslash}p{\dimexpr 22.50pt -2\tabcolsep-1.5\arrayrulewidth}|>{\centering\arraybackslash}p{\dimexpr 67.50pt -2\tabcolsep-1.5\arrayrulewidth}>{\centering\arraybackslash}p{\dimexpr 67.50pt -2\tabcolsep-1.5\arrayrulewidth}>{\centering\arraybackslash}p{\dimexpr 67.50pt -2\tabcolsep-1.5\arrayrulewidth}>{\centering\arraybackslash}p{\dimexpr 67.50pt -2\tabcolsep-1.5\arrayrulewidth}>{\centering\arraybackslash}p{\dimexpr 67.50pt -2\tabcolsep-1.5\arrayrulewidth}}
\toprule
 &  &  & \multicolumn{3}{>{\centering\arraybackslash}m{\dimexpr 202.50pt -2\tabcolsep-1.5\arrayrulewidth}}{V\textsubscript{\{ij\}}\textsuperscript{D}} \\ 
\cmidrule(lr){4-6}
 &  &  & f\textsubscript{1} (c\textsubscript{ij}) = 1/c\textsubscript{ij}\textsuperscript{3} & f\textsubscript{2} (c\textsubscript{ij}) = 1/c\textsubscript{ij}\textsuperscript{2} & f\textsubscript{3} (c\textsubscript{ij}) = 1/c\textsubscript{ij}\textsuperscript{0.1} \\ 
\cmidrule(lr){4-4} \cmidrule(lr){5-5} \cmidrule(lr){6-6}
Dest. & Population at 2 (units: \emph{people in 10,000s}) & Opportunities (units: \emph{capacity in 10,000s}) & units: \emph{physician-capacity in 10,000s} & units: \emph{physician-capacity in 10,000s} & units: \emph{physician-capacity in 10,000s} \\ 
\midrule\addlinespace[2.5pt]
1 & 10.000 & 7.000 & 2.133 & 2.272 & 3.411 \\ 
2 & 10.000 & 5.000 & 4.959 & 4.766 & 2.667 \\ 
3 & 10.000 & 8.000 & 2.904 & 2.958 & 3.919 \\ 
\bottomrule
\end{tabular*}
\end{table}



Recall, accessibility is defined as ``the potential for interaction'' and is traditionally presented as a summary zonal measure. In the doubly constrained case, we force the zonal population and zonal opportunities to match one-to-one, hence providing a zonal summary is no longer relevant: the sum of \(V_{ij}^D\) for all \(i\)s ends up being equal to the population at the zone. However, if what interests readers is the ``potential for interaction'' in the case population and opportunities to match one-to-one, reframing the investigation may be needed. In this sense, it would be examining ``interaction'' (much less room for potential) through the values of \(V_{ij}^D\) e.g., how many opportunities are being allocated to an origin from a destination (or in a transposed sense for \(M_{ji}^D\)). In this sense, the doubly constrained case can be thought of as an estimate of \emph{realized} accessibility or access: reflecting spatial interaction. It is also formulaically identical to the doubly constrained spatial interaction model, but with specific interpretations of the origin and destination weights as `population' and `opportunity', respectively.

As Wilson explicitly noted, origin and destination weights defined in the spatial interaction model \emph{can} be defined using any unit. Accessibility, however, is often presented and understood as a zonal summary of \emph{potential} for interaction between origins and destinations that contain inherently different units. Arriving at the doubly constrained case through the unconstrained, total, and singly constrained cases make the connection between the \emph{potential} for spatial interaction (accessibility) and \emph{realized} potential for spatial interaction more interpretable. Namely, by demonstrating that these members of the accessibility family all derive from the same root and can be derived from Wilson's original formulation, this more clearly demonstrates what \emph{potential} is, within the context of spatial interaction modelling.

Namely, \emph{potential} depends on the framing of the masses at the origin and destination, and how similar they are in their units. The appropriate constraint should be decided based on the the input data and their similarity. In increasing unit similarity from the perspective of opportunity-constraint: the total-constraint can be used if population (origin mass) is not known, the single constraint can be used if population is known and matters to \emph{potential} but it does not match the capacity of opportunities, and lastly, the double constraint is appropriate if population is known, matters and population matches the capacity of opportunities. These constraint measures can reflect the opportunities accessible at each zone, but reflects the assumptions embedded in the input data about the potential to spatially interact.

A multimodal version could be defined by solving the multimodal singly-constrained balancing factors \(A_i^m\) and \(B_j^m\) simultaneously. However, we do not pursue this approach here in this work. As discussed earlier, the doubly-constrained accessibility measure is equivalent to the doubly-constrained spatial interaction model--one that models realized spatial interaction flows rather than `potential'. Since accessibility is fundamentally about potential access, not realized flows, this extension falls outside the scope at this moment.

\section{Chapter conclusions}\label{chapter-conclusions}

In this chapter, balancing factors are introduced akin to those used in Wilson's family of spatial interaction models. They are formulated to incorporate system-wide or zonal constraints (i.e., knowns) to accessibility. Four cases of the family are outlined: unconstrained, total constrained, singly constrained and doubly constrained. Variants of each case (i.e., either accessible opportunities, or accessible populations), along with their multimodal extensions, are mathematically formulated and solved assuming a simple toy example.

The constraining constants, depending on the case (i.e., total, singly- or doubly- constrained), restrict the degree of \emph{potential}, linking accessibility (the potential for spatial interaction) with access (spatial interaction) on the same continuum based on the constraint used. This chapter also discussed how popular measures such as the one used in Hansen (1959) and the 2SFCA link into the family of accessibility measures. The family of accessibility measures, as follows.

We first place the popular Hansen-type accessibility measure (Hansen, 1959) within this family of measures as an ``unconstrained'' case, demonstrating that resulting values cannot be directly compared across different travel scenarios without ad-hoc adjustments. We then show how applying a total constraint balances the units and produces a statistically averaged solution that converges to the regional average for each zone as the decay effect decreases. In other words, the total-constraint model could be a more interpretable alternative for the unconstrained case if population-competition is not relevant and one is interested in capturing the maximum \emph{potential}; specifically, if there is a fixed number of opportunities in the region, and if it makes sense to assume that people accessing proximate opportunities leave fewer for others, \emph{without} considering the population size at the origins.

We then introduce the singly constrained case, which \emph{does} takes into account the population size at the origin in the allocation of opportunities (unlike the total constraint). It is also mathematically equivalent to the spatial availability introduced in Soukhov et al. (2023). In this case, all accessibility values are fixed to sum to a known zonal opportunity-size value (implicitly, the regional total of opportunities), but they are not required to sum to any population-based values at the zone or regional level. The singly constrained model could be useful if regional competition is a factor and if the acknowledgment that only a finite number of opportunities can be allocated from each destination (with those allocations distributed based on origin population size) is suitable. We also introduce an `accessible' PPR (e.g., opportunities per capita), calculated by dividing each accessibility value by the zonal population. To clarify, this per capita expression of the singly constrained case is equivalent to the 2SFCA (Luo \& Wang, 2003a; Q. Shen, 1998b), hence linking this literature back to spatial interaction principles.

Lastly, the doubly constrained case is introduced. In this case, the sums must equal both the regional total and ensure that no zone allocates more opportunities than it has available. Specifically, accessibility values for each \(i\)-\(j\) pair must be a proportion of he zonal opportunity and population values simultaneously. For example, the accessibility at zone 1 must equal the sum of opportunities from zones 1, 2, and 3, as well as the sum of the population at zone 1. Satisfying the double constraint means the opportunities and population data must match one-to-one, so working with the accessibility \(i\)-\(j\) pair values should be of interest. In this sense, the research question should be concerned with `access' (how many opportunities accessed from \(j\) at \(i\) based on given zonal opportunities and populations) instead of potential spatial interaction (e.g., typically expressed as a zonal summary measure of how many opportunities one could reach (out of a regional total and/or zonal-allocation).

In summary, building on Wilson (1971)'s foundational work, this paper proposed a unified framework for accessibility that is able to account for competition. By reintroducing Wilson's proportionality constant, the proposed family of constrained accessibility measures restores measurement units to accessibility estimates. This enhancement provides a more interpretable, consistent, and theoretically grounded basis for accessibility analysis, which could help advance the adoption of accessibility-oriented planning.

With the aim of demonstrating how the family of accessibility measures may improve interpretability planning for accessibility-oriented planning, the next chapter outlines an empirical example of the population and parkland in the City of Toronto. This data will be used in later chapters of this dissertation to calculate cases of the family of accessibility measure and demonstrate their interpretability advantages.

\chapter{CHP 3 - Describing the data: an empirical case study of Toronto's Parklands}\label{chp-3---describing-the-data-an-empirical-case-study-of-torontos-parklands}

\section{Introduction}\label{introduction-1}

Urban greenspace has many benefits for residents and the environment: it is linked to wellbeing {[}{]}, higher rates of physical activity {[}{]}, and contributes to reducing urban island heating as well as mitigating carbon pollution in combating climate change {[}{]}. Parkland, or greenspace that is owned and operated by a locality, can also be framed as a public service. In this way, the spatial accessibility to parkland should be concerned with ensuring it is distributed in an equitable way.

In this chapter, the empirical example of parkland city of Toronto will be detailed, including (1) the spatial resolution of the zoning system and the residing population, (2) the assumptions associated with calculating parkland entrance points, and the (3) assumed interaction with parkland. (4) Following the data, the totally-constrained accessibility measure is briefly summarised. The calculated results are presented and interpreted in a policy-relevant lens: namely, the amount of parkland that is accessible per zone, identification of zones with low and high accessibility.

Notably, this process of methods, data and results presentation was done in consultation with staff from the City of Toronto. They are actively working on developing the Transportation Equity Policy Framework and are partners of the Mobilizing Justice Partnership. All data is openly available, but is edited by the city staff (as attributed) or using assumptions the city staff were agreement with.

\section{Methods and Data}\label{methods-and-data}

\subsection{Origins: the dissemination block and associated census data}\label{origins-the-dissemination-block-and-associated-census-data}

The most disaggregated level that census variables (e.g., household income, proportion commute mode) is available is the level of the dissemination area (DA). Census zoning systems are designed by Statistics Canada to represent the population's socioeconomic characteristics as homogeneously, with disaggregated systems (like the DA) requiring to nest within more aggregated systems while remaining a relatively compact geographic size (Statistics Canada, 2021a). Each DA in the Toronto metropolitan area represents population between 441 (Q1) to 836(Q3)) and each representing between 0.0732 sq km (Q1) to 0.2322 sq km (Q3)) in area.

However, population data is available at an even more disaggregated level, the dissemination blocks (DB). The DB is an artifact of the road network (Statistics Canada, 2021b), with between 2 to 5 DBs being typically nested within one DA . The more accurately the origin point represents the known residing population, the more accurately the routed travel time and hence the accessibility results along with its associated assumptions, can be understood. Moreover, the highest level of spatial disaggregation available, is typically the preferred appropriate: to reduce issues associated with the scale effect of the modifiable areal unit problem (MAUP) {[}{]}.

For these reasons, points that are representative of the population within each DB was calculated using the methodology described by Statistics Canada (Statistics Canada, 2021c). These representative points are available through Statistics Canada at the DA level, however, DB level points are not. hence, the calculated DB points, or the DB weighted centroids, were generated by the staff at the City of Toronto, and sent to the author on May 2, 2025. The methodology to prepare the points is summarised as follows: (1) if the DB contained no address points, then the geometric centroid of the DB was used, (2) if the DB contains 1 or more address points, then the central point between the dwelling weighted address points was calculated. If this point was not with the DB boundaries, it was manually moved into the DB.

To contextualise the Toronto metropolitan area, the population (per DB) and the associated DA boundaries plus the cities that compose the Toronto metropolitan area are visualised in Figure \ref{fig:chp3-toronto_CMA_plot}.

\begin{figure}

{\centering \includegraphics[width=6in]{./data/figures/chp3-toronto_CMA_plot} 

}

\caption{\label{fig:chp3-toronto_CMA_plot}Map of Toronto CMA with population per DB from the 2021 Census}\label{fig:unnamed-chunk-42}
\end{figure}

To contextualise the city of Toronto, the focus of this analysis, Figure \ref{fig:chp3-toronto_popden_NIAs_plot} demonstrates the 158 neighbourhoods in Toronto according to improvement classification: either a ``Neighbourhood Improvement Area'' (NIA), ``Emerging Neighbourhood'' (EA), or neither, based on an analysis conducted by the City. The methodology was based on the development of a composite indicator including dimensions of relative marginalization of the population who resides in these neighbourhood and of the neighbourhood infrastructure itself, such as variables reflecting economic opportunities (e.g., unemployment), social development (e.g., highschool graduation), participation in decision-making (e.g., voting), neighbourhood infrastructure (e.g., walkability, community places for meetings), and health (e.g., premature mortality) (City of Toronto, 2014, 2024). Overall, higher-population neighbourhoods are concentrated near the lake in the downtown core. Most NIAs are located in the east and northwest of the city and tend to have lower population densities, though two high-density NIAs are located in downtown.

\begin{figure}

{\centering \includegraphics[width=6in]{./data/figures/chp3-toronto_popden_NIAs_plot} 

}

\caption{\label{fig:chp3-toronto_popden_NIAs_plot}Map of the 158 Toronto neighbourhoods, with 'Improvement' classification and population density from the 2021 Census}\label{fig:unnamed-chunk-43}
\end{figure}

And lastly, the Figure \ref{fig:chp3-toronto_popden_DBCent_plot} displays the calculated DB weighted centroids, used as the origins in the accessibility analysis. There are 13322 points within the City, more spatially clustered in areas with higher population density.

\begin{figure}

{\centering \includegraphics[width=6in]{./data/figures/chp3-toronto_popden_DBCent_plot} 

}

\caption{\label{fig:chp3-toronto_popden_DBCent_plot}Map of the City of Toronto's DB weighted centroids atop the population density from the 2021 Census}\label{fig:unnamed-chunk-44}
\end{figure}

\subsection{Toronto parkland destinations and normative travel behaviour}\label{toronto-parkland-destinations-and-normative-travel-behaviour}

Parkland is defined as city operated and/or owned `parks' identified in the greenspaces shapefile available through the city's official Open Data portal (City of Toronto, 2025). These are the same park assets that are identified as part of the Parkland Strategy report commissioned by the City of Toronto (City of Toronto, 2019, p. 20). This report serves as a guideline for assumptions made regarding park classification and interaction catchments based on park classification. Toronto's parkland is visualized in Figure \ref{fig:chp3-parkland_paths_plot}. Of note, other Open Spaces include in the greenspaces shapefile are federally or provincially owned/operated spaces, school yards, cemeteries, and hydro corridors. These spaces are not assumed to be Toronto operated and/or owned parkland, hence are not included in this analysis. These greenspaces are reflected as `no population' DBs.

\begin{figure}

{\centering \includegraphics[width=6in]{./data/figures/chp3-parkland_paths_plot} 

}

\caption{\label{fig:chp3-parkland_paths_plot}Map of the City of Toronto's Parkland (with paths or no paths) atop the population density from the 2021 Census}\label{fig:unnamed-chunk-45}
\end{figure}

As retrieved from City of Toronto (2019, p. pg.15), the parkland can be categorized by size: `Parkette' (\textless0.5 ha) (40\% of all parks), `Small Park' (0.5--1.5 ha) (20\%), `Medium Park' (1.5--3.0 ha) 16\%, `Large Park' (3.0--5.0 ha) 9\%, `City Park' (5.0--8.0 ha) 5\%, and `Legacy Park' (\textgreater8.0 ha) 10\%. Interaction with parks can be assumed in a variety of ways, but in this analysis, it is normatively (A. Paez et al., 2012) assume interaction based on the park classification catchments described in City of Toronto (2019). Parkettes are assumed to only be accessible within a 10-minute travel window, small parks within 20 minutes, and medium parks within 30 minutes. In contrast, large, city, and legendary parks are considered to have citywide appeal, attracting people throughout the city regardless of travel time.

How parks can potentially be accessed is assumed based on their entrances. Entrances are not explicitly available, so they are assumed. They are calculated based on the parkland edge intersection with each path within the park itself or intersecting the park edge. 4 of the the 1607 parks have paths, each with anywhere between 1 to 84 (median of 2) entrances. The remaining 1603 parks do not have paths, and hence their entrances cannot be precisely assumed. Parks without an entrance are significantly smaller in area (i.e., median of 0.24 ha as opposed to the median of area parks with paths which is 1.43 ha). Upon visual inspection using Google Maps streetview, parks with no paths often contain a playground, some sport amenity or gardens within the center. Some of them are unfurnished -containing only mowed grass. For these parks with no internal paths, it is assumed then that these spaces can be entered from any direction and that the geometry centroid is the point of interest calculated and assumed as the entrance point. This point is snapped onto the transportation network based on an origin's shortest path, as will be later described in the routing subsection.

As a visualisation, Figure \ref{fig:chp3-park_entrance_example_plot} contains a panel of three DAs in Toronto, each displaying the roads, parks, and assumed park entrances. Readers should note that in the first two panels, these parks contain multiple entrance points, corresponding to where the edge of the paths within the parks intersect with the edge of the park boundary. These plots showcase different DAs that represent the diversity in DA size and park composition across the city. The first plot showcases a relatively small DA, near the downtown core of the city featuring high density of population, other amenities and multimodal transportation systems. This DA is unique in the area, as it features a large planned `Legacy' park by the name of Christie Pits. Planned parks are typically square or rectangular and can be accessed from most of the sides. They are also contained within the city in all areal sizes. The second plot contains a larger DA, north of downtown, that contains a few parks and near higher density suburban built form. One is the Roycroft Park Lands, a large `City' (smaller than Legacy), part of the Don River wetlands with maintained trails, enjoyed as nature reserve. Note it's long shape and minimal entrances. Natural parklands like Roycroft Park Lands are common along wetlands and other preserved natural spaces; they are typically larger in size (Medium, Large, or City parks) hence offer a lot of parkland space to those near their entrances. In the third plot, the DA is also large but with lower density suburban built form, and mostly residential. It contains only one small park: Pleasantview Park. This park contains no internal pathways, a playground in the middle of the park with no other amenities.

\begin{figure}

{\centering \includegraphics[width=6in]{./data/figures/chp3-park_entrance_example_plot} 

}

\caption{\label{fig:chp3-park_entrance_example_plot} Three DAs featuring parkland, road network by OSM tag, and park entrances. From top to bottom: planned Legacy park Christie Pits near the downtown core, parks north of the downtown core preserving natural space, and planned park with no paths in a more suburban DA.}\label{fig:unnamed-chunk-46}
\end{figure}

\subsection{Multimodal origin to destination routing and trip lengths}\label{multimodal-origin-to-destination-routing-and-trip-lengths}

Routing of travel times was done using the travel\_time\_matrix() function in \{r5r\}, an R package that provides an R-interface for the Java-based R5 Routing engine (Pereira, Saraiva, Herszenhut, Braga, \& Conway, 2021). The function was run four time, one for bicycle, car, transit and walking modes, producing four separate travel time matrices. Each origin destination pair has an associated shortest travel time, selected by the function based on all possible origin destination routes given the input road network (and transit schedule, for the transit mode). The road network is an edited version of OpenStreetMaps (OSM) street network (December 9, 2022) and edited General Transit Feed Specification (GTFS) files (February 4, 2024) for transit operating within the City i.e., GO (regional commuter train and bus service), TTC (local subway, lightrail, and bus network), and the UP Express (regional commuter train line). The files were edited by city staff to more accurately reflect access into TTC subway stations and reflect the transit schedule for the week of February 4 2025.

Concerning the inputs for the `travel\_time\_matrix()' function for all modal travel time calculations: the origins are the DB weighted centroids (13322 locations), the destinations are points representing the 1607 parks and the OSM road network were used. Some parks have multiple pieces, separated by the road network: in total, there are 1958 park pieces, each with between 1 and 68 path entrances (median 3 entrances) or 1 park piece centroid. In total, there are 6274 possible destination points (5724 path entrances and 550 centroids) for each of the 13322 origins.

For the motorized modes, a maximum travel time of 120 minutes was selected. For transit, the GTFS files and a departure time of between 11:00-11:15am (i.e., a 15 minute departure window) on February 8th 2025 was used, to reflect a Saturday afternoon transit schedule. Travel speeds reflect the posted speeds and intersections as gleaned from the OSM road network, and for transit, the scheduled transit arrival times at stops according to the GTFS file. For non-motorized modes, a maximum travel time of 30 minutes was selected, and the default travel speeds of 3.6 km/h for walk and 12 km/h for cycling was assumed. The travel time thresholds of 120 minutes and 30 minutes was set to normatively reflect the likelihood to travel to parks, by mode.

It is worthwhile summarising the multimodal travel time matrices. Notably, within a 120 minutes trip by motorized modes, the majority of DBs can reach all parks by car (with exception to the 6 on the Toronto Islands, inaccessible by car), i.e., the median DB can reach 1601 out of the 1607 parks, while the most isolated can still reach 184 parks. By contrast, within a 120 minute trip or less by transit, a median DB can only reach 1116 parks, with the most central DB reaching 1539 parks meaning 4\% of parks are feasibly unreachable by transit for DBs in the City of Toronto. These parks are located at the edges of the city in that are transit poor, as will be demonstrated and discussed in the findings. Comparing this reach to the lower range non-motorized modes, the number of parks reachable by foot or by cycle within 30 minutes from a DB is much lower: a median DB can only reach 15 parks (max. 61 parks for the most central DB) by walking and 86 parks (max. 295 parks for the most central DB) by cycling.

As a final note on routing assumptions, it's useful to compare travel times to park centroids versus park path entrances. For the 1204 parks with known path entrances, centroids were also computed and travel times from all DBs to all centroids were calculated. This comparison highlights the impact of the R5 routing algorithm, namely, how the routed travel times can differ depending on how the destination points are `snapped' to the nearest road segment. If a point is not already on the road network, R5 snaps it to the nearest network segment, adding a walking time penalty based on distance. This snapping algorithm prioritizes minimizing overall travel time from the origin. Since edge entrances are typically already on the network, their snapping penalty is minimal. In contrast, centroid points--especially in large and irregular natural parks---may snap to parts of the road network that are unreachable by certain roads (e.g., far enough away from bus stops, or bike lanes), inflating travel times. Hence: when paths within the parks are available from the OSM network, they are used, as it more accurately reflects the points at each the parks can actually be entered from. But when the park has no entrances, it is assumed that the centroid (or the middle of the park itself) is the destination point, and the associated snapping penalty is folded into the calculated travel time. The following Figure \ref{fig:chp3-ent_vs_cent_tt_car_transit_scatter} and Figure \ref{fig:chp3-ent_vs_cent_tt_cycle_walk_scatter} demonstrate the relationship between the minimum travel time used in the analysis for each parks with path entrances and the travel time if its centroid point was used, along with a 45 degree dashed line representing a perfect linear relationship.

Regarding the motorized modes in Figure \ref{fig:chp3-ent_vs_cent_tt_car_transit_scatter}), the relationship appears to be roughly, with centroid times being consistently lower than path entrance times--except for a few parks that fall to the left of the dashed line. Parks with entrance times that are larger than their centroid times have entrances that are not in opportune positions on the network relative to the snapped centroid point, and vice versa for entrances that have \emph{lower} travel times than their centroid points which is often the case. Transit shows a similar trend but with more noise. A few parks exhibit exceptionally high transit times relative to their centroid times. These are typically larger parks where the centroid snaps to a location near access points to the transit system, but the \emph{actual} path entrances are not in proximity to those opportune system access point. Since the road network for cars is more continuous and the system is more evenly accessible (e.g., the majority of roads can be driven on, whereas the transit system can only be entered in specific points), this discrepancy is not observed in the car mode. This comparison underscores the importance of using realistic entrance points especially for larger parks with few entrance points and for modes like transit, which do not provide uniform access into the system.

\begin{figure}

{\centering \includegraphics[width=6in]{./data/figures/chp3-ent_vs_cent_tt_transit_car_scatter} 

}

\caption{\label{fig:chp3-ent_vs_cent_tt_car_transit_scatter} Scatter plot of car and transit travel times from origins to destinations by destination type for each park (either mininum median travel time to park centroid, or minimum median travel time to park entrance). }\label{fig:unnamed-chunk-47}
\end{figure}

Figure \ref{fig:chp3-ent_vs_cent_tt_cycle_walk_scatter} captures the non-motorized modes, which exemplifies a similar pattern as the motorized modes, with centroid travel times beign typically longer than path entrance points in a linear relationship. However, like transit and car, travel by bike and walk demonstrate different levels of noise. As the travel time threshold is 30 minutes, distance can be traversed at a faster rate by bike than by foot, hence differences in distances between the path entrance point and the snapped centroid is less impact on the estimated travel time for bike mode.

\begin{figure}

{\centering \includegraphics[width=6in]{./data/figures/chp3-ent_vs_cent_tt_walk_cycle_scatter} 

}

\caption{\label{fig:chp3-ent_vs_cent_tt_cycle_walk_scatter}  Scatter plot of cycle and walk travel times from origins to destinations by destination type for each park (either min mean travel time to park centroid, or to min mean travel time to park entrance). }\label{fig:unnamed-chunk-48}
\end{figure}

\subsubsection{Normative park trip length}\label{normative-park-trip-length}

A key component of accessibility is how the cost of overcoming spatial separation to reach a destination is considered. In this analysis, access to parkland is modeled using distance catchments based on park size classifications from the City of Toronto (2019) report, and modal options informed by discussions with city staff.

It is assumed that the smaller parks, i.e., parkette (\textless0.5ha), small parks (0.5-1.0 ha), and medium parks (1.5-3 ha) offer a service coverage of only 0.5km, 1.0km and 1.5km, respectively. Meaning, people are only likely to interact with the park within this travel distance. This assumption has to do with the qualities these parks possess, and the availability of similar parks in the area: namely, these smaller parks are often located in residential neighbourhoods, evenly spatially distributed, and often offer no exceptional amenity that can't be substituted by another similarly sized park nearby as inferred from City of Toronto (2019). For larger parks, such as large parks (3-5 ha), city parks, and legacy parks, their service catchments are larger as they are more unique, attractive, and harder to substitute parks, as inferred from the City of Toronto (2019) report. Large parks are assumed a service coverage of 3km and city and legacy parks are assumed to be attractive, and hence cover, the whole city.

\begin{verbatim}
[1] 17.12376
\end{verbatim}

Trip lengths to parks vary by both travel mode and park classification. Based on assumptions drawn from City of Toronto (2019) and discussions with city staff, travel impedance functions are created based on trip lengths that are \emph{seen} as acceptable. It is assumed that (1) non-motorized travelers can access all parks based on their service coverage catchment (i.e., a 0.5 km catchment for parkettes that translate to 8 mins of walking or 3 mins of cycling based on median travel speeds), however, they are constrained by their maximum travel time length of 30 minutes. (2) transit users can access all types of parks, but based on a negative exponential distance decay function (\(f(c_{ij}) = e^{-0.02c_{ij}}\)). This function has a median travel time of approximately 30 minutes, meaning that half of the total weight of the travel impedance function--representing the spatial separation between people and parks--comes from trips between 0 to 30 minutes, and the other half comes from trips between 30 to 120 minutes. The function captures the idea that the deterrent effect of distance increases gradually, assigning progressively less `spatial interaction' weight to more distant parks, even though they remain reachable. (3) Car users are assumed \emph{not} to travel to smaller parks at all, and as such, these parks are excluded from accessibiltiy results for the car mode. For larger parks, car travel impedance is assumed also based on a negative exponential distance decay function, but with parameter \(-0.04\) (\(f(c_{ij}) = e^{-0.04c_{ij}}\)), twice as steep as the transit impedance function. This reflects a stronger deterrent effect with increasing travel time. Namely, with a median travel time of approximately 17 minutes, trips that are between 0 to 17 minutes are assigned half the weight of the travel impedance function and the remaining half is defused acrossed trips that are between 17 minutes to 60 minutes. Destinations 60 minutes are considered effectively unreachable, with the function modeled as 0 beyond this point.

These assumptions on interaction with parks and trip lengths by mode have equity implications:
- Smaller parks are assumed to only be attractive to, hence accessible for, non-motorized and transit users
- For larger parks which can be reached by all modes, car mode has the largest range and has a steep decay, meaning it may provide favourable spatial seperation weight, relative to other modes.
- Travel impedance for transit users is penalized more gradually, assuming broader reach to all park types
- Non-motorized users reach is constrained practically: they are assumed not to travel beyond 30 minutes
- For parks in less densely populated areas, i.e., with travel times by sustainable modes beyond 30 minutes, accessibility will only be present for car users. In sum, individuals in these areas who do not have access to a vehicle are effectively excluded.

These assumptions on interaction with parks and trip lengths by mode have equity implications:
- Smaller parks are assumed to be attractive only to non-motorized and transit users, and are therefore not included in the accessibility profile for car users.
- For larger parks, which are accessible by all modes, the car mode has the broadest assumed range and steepest decay, potentially resulting in more opportune spatial separation weighting relative to other modes.
- Transit users face a more gradual decay in travel impedance, reflecting broader assumed reach across all park types.
- Non-motorized users are assumed to be practically constrained in their reach, with travel limited to within 30 minutes on foot or by bike.
- Parks in lower-density areas requiring more than 30 minutes of travel by sustainable modes will contribute to high accessibility for car users exclusively. Individuals in these areas without access to a private vehicle are, effectively, excluded, and these zones will reflect very low levels of accessibility by sustainable modes.

To summarise the modal \(m\) travel behaviour based on parkland classification type \(y\) as the resulting travel impedance functions \(f^m(c^m_{ij})\) in Table \ref{tab:chp3-travel-impedance-by-park-mode}:

\begin{table}[ht]
\centering
\small
\begin{tabular}{|l|c|c|c|c|}
\hline
\textbf{Park Type} & \textbf{Car} & \textbf{Transit} & \textbf{Cycling} & \textbf{Walking} \\
\hline
Parkette &
0 &
$e^{-0.02} \cdot c_{ij}^{\text{transit}}$ &
$1$ if $c \leq 3$ min, else 0 &
$1$ if $c \leq 8$ min, else 0 \\
\hline
Small Park &
0 &
$e^{-0.02} \cdot c_{ij}^{\text{transit}}$ &
$1$ if $c \leq 5$ min, else 0 &
$1$ if $c \leq 17$ min, else 0 \\
\hline
Medium Park &
0 &
$e^{-0.02} \cdot c_{ij}^{\text{transit}}$ &
$1$ if $c \leq 8$ min, else 0 &
$1$ if $c \leq 25$ min, else 0 \\
\hline
Large Park &
$e^{-0.04} \cdot c_{ij}^{\text{car}}$ &
$e^{-0.02} \cdot c_{ij}^{\text{transit}}$ &
$1$ if $c \leq 15$ min, else 0 &
$1$ if $c \leq 30$ min, else 0 \\
\hline
City Park &
$e^{-0.04} \cdot c_{ij}^{\text{car}}$ &
$e^{-0.02} \cdot c_{ij}^{\text{transit}}$ &
$1$ if $c \leq 30$ min, else 0 &
$1$ if $c \leq 30$ min, else 0 \\
\hline
Legacy Park &
$e^{-0.04} \cdot c_{ij}^{\text{car}}$ &
$e^{-0.02} \cdot c_{ij}^{\text{transit}}$ &
$1$ if $c \leq 30$ min, else 0 &
$1$ if $c \leq 30$ min, else 0 \\
\hline
\end{tabular}
\caption{Normative travel impedance functions by parkland classification $y$ and mode $m$}
\label{tab:chp3-travel-impedance-by-park-mode}
\end{table}

To reiterate, the trip length based travel time behaviour summarised in Table \ref{tab:chp3-travel-impedance-by-park-mode} are \emph{normative}--they represent a statement about what should be considered a reasonable travel time that defines parkland accessibility (A. Paez et al., 2012). In practice, however, travel behaviour empirically may diverge from these normative statements. For example, empirical data from the 2023 Transportation Tomorrow Survey of trips made for `leisure' purposes by different modes in the Greater Toronto Area region, reveals different patterns (Data Management Group, 2023). Figure \ref{fig:chp3-norm_pos_impedance_mode_parktype_plot} compares this normative travel behaviour (green lines) from Table \ref{tab:chp3-travel-impedance-by-park-mode} with the TTS empirically derived curves (dashed lines). For this analysis, the normative curves (i.e., Table \ref{tab:chp3-travel-impedance-by-park-mode}) are used to define accessibility. This will enable the interpretation of results in terms of what \emph{should} be accessible via each mode, according to planning goals, rather than what currently is based on observed travel behaviour.

\begin{figure}

{\centering \includegraphics[width=6in]{./data/figures/chp3-norm_pos_impedance_mode_parktype_plot} 

}

\caption{\label{fig:chp3-norm_pos_impedance_mode_parktype_plot}  Scatter plot of cycle and walk travel times from origins to destinations by destination type for each park (either min mean travel time to park centroid, or to min mean travel time to park entrance). }\label{fig:unnamed-chunk-50}
\end{figure}

\subsection{Totally-constrained accessible parkland: all people demand it equally}\label{totally-constrained-accessible-parkland-all-people-demand-it-equally}

\subsubsection{A measure of parkland accessibility and population accessibility}\label{a-measure-of-parkland-accessibility-and-population-accessibility}

Totally-constrained accessibility \(V^T_{ij}\), a measure of parkland accessibility, is defined as follows:

\[
V^T_{ij} = K^T \cdot W_j^{(2)} \cdot f(c_{ij})
\] \{\#eq-total-constrained-access\}

Where:
- \(V^T_{ij}\) is the number of opportunities that can be accessed at origin zone \(i\) from destination zone \(j\),
- \(f(c_{ij})\) is the cost of travel \(c_{ij}\) from \(i\) to \(j\),
- the destination zone attraction mass \(W_j^{(2)}\) is the number of opportunities (i.e., the parkland in hectares at a park \(j\)), and
- \(K^T\) is the total constraint \(\frac{D}{\sum_i\sum_j W^{(2)}_jf(c_{ij})}\) that serves to proportionally allocate the opportunities \(D\) in the region and ensures units remain balanced.

Equation \ref{eq:total-constrained-access} represents totally-constrained access at \(i\) from \(j\), but it can be summarised as \(V^T_i\) by summing all \(V^T_{ij}\) for a specific \(i\) (i.e., \(V^T_i = \sum_j V^T_{ij}\)). \(V^T_i\) is linearly proportional to the Hansen-type accessibility measure \(S_i = W^{(2)}_jf(c_{ij})\). Furthermore, it is worth reiterating that the sum of \(V^T_{ij}\) across the region is equal to \(D\) i.e., \(\sum_i\sum_j V^T_{ij} = \sum_i V^T_{i} = D\).

A measure of population accessibility \(M^T_{ji}\) can also be defined using the totally-constrained formulation:
\[
M^T_{ji} = \hat K^T \cdot W_i^{(1)} f(c_{ji})
\] \{\#eq-total-constrained-market\}

Where:
- \(M^T_{ji}\) is the number of population that can be accessed from origin zone \(i\) by destination zone \(j\),
- \(f(c_{ji})\) is the cost of travel \(c_{ji}\) from \(j\) to \(i\),
- the destination zone attraction mass \(W_j^{(2)}\) is the number of opportunities (i.e., the parkland in hectares at a park \(j\)), and
- \(\hat K^T\) is the total constraint \(\frac{O}{\sum_i\sum_j W^{(2)}_jf(c_{ji})}\) that serves to proportionally allocate the population \(O\) in the region and ensures units remain balanced.

Equation \ref{eq:total-constrained-market} can also be summarised as \(M^T_j\) by summing all \(M^T_{ji}\) for a specific \(i\) (i.e., \(M^T_j = \sum_i M^T_{ji}\)). \(M^T_j\) is linearly proportional to the Hansen-type accessibility measure of market potential, hence, it is worth reiterating that the sum of \(M^T_{ji}\) across the region is equal to \(O\) i.e., \(\sum_i\sum_j M^T_{ji} = \sum_i M^T_{j} = O\).

\subsubsection{Multimodal extension}\label{multimodal-extension}

This measure can also be extended to reflect multiple modes. In this analysis, four modes are considered: the motorized car and transit options, and the non-motorized cycling and walking options. From origins to destinations, each mode has a different travel impedance function \(f^m(\cdot)\) and travel time cost \(c^m_{ij}\) (note: \(c^m_{ij}\) and \(c^m_{ji}\) are assumed to be equal). The totally-constrained formula is modified as follows include a sub-index \(m\):

\[
V^{mT}_{ij} = K^{mT} \cdot W_j^{(2)} \cdot f^m(c^m_{ij})
\]\{\#eq-total-constrained-multimodal-access\}

Where:
- \(V^{mT}_{ij}\) is the number of opportunities that can be accessed at origin zone \(i\) from destination zone \(j\) by mode \(m\),
- \(f^m(c^m_{ij})\) is the cost of travel \(c^m_{ij}\) by mode \(m\) from \(i\) to \(j\),
- the destination zone attraction mass \(W_j^{(2)}\) is the number of opportunities (i.e., the parkland in hectares at a park \(j\)), and
- \(K^{mT}\) is the modal total constraint \(\frac{D}{\sum_m\sum_i\sum_j W^{(2)}_jf^m(c^m_{ij})}\) that serves to proportionally allocate the opportunities \(D\) in the region and ensures units remain balanced.

Summarising equation \ref{eq:total-constrained-multimodal-access} as a measure of modal totally-constrained accessibility \(V^{mT}_i\) by summing all \(V^{mT}_{ij}\) for a specific \(i\) and \(m\) (i.e., \(V^{mT}_i = \sum_j V^{mT}_{ij}\)). \(V^{mT}_i\) can also be summed by mode to equal \(V^{T}_i\) (i.e., \(\sum_m V^{mT}_i = V^{T}_i\)) and summed across the region to equal \(D\) (i.e., \(\sum_m\sum_i\sum_j V^{mT}_{ij} = \sum_m\sum_i V^{mT}_{i} = D\)).

Transposing \(i\) and \(j\) and expressing \(M^{mT}_{ji}\), the multimodal `market potential' or the population accessible by mode can also be defined using the totally-constrained formulation, with similar parameters as previously defined.
\[
M^{mT}_{ji} = \hat K^{mT} \cdot W_i^{(1)} f^m(c^m_{ji})
\] \{\#eq-total-constrained-multimodal-market\}

\subsubsection{Multimodal and multi-opportunity type extension}\label{multimodal-and-multi-opportunity-type-extension}

And lastly, the measure can also be extended to reflect multiple opportunities types (sub-index \(y\)), in addition to multiple modes.

\[
V^{ymT}_{ij} = K^{mT} \cdot W_j^{y} \cdot f^m(c^m_{ij})
\]\{\#eq-total-constrained-multimodal-multiopp-access\}

If there was a multiple population groups considered, then a \(M^{ymT}_{ji}\) could be specified. But in this analysis, this data will not be incorporated. However, market potential per mode per park can be specified:

\[
M^{ymT}_{ji} = \hat K^{mT} \cdot W_i^{(1)} f^m(c^m_{ji})
\] \{\#eq-total-constrained-multimodal-multiopp-market\}

\subsubsection{Representing constrained accessibility as a ratio}\label{representing-constrained-accessibility-as-a-ratio}

parkland accessible per capita:
\[
v^{T}_{i} = V^{T}_{i} /P_{i}^{m}
\]\{\#eq-total-constrained-access-per-capita\}

\[
v^{mT}_{i} = V^{mT}_{i} /P_{i}^{m}
\]\{\#eq-total-constrained-multimodal-access-per-capita\}

\[
v^{ymT}_{i} = V^{ymT}_{i} /P_{i}^{ym}
\]\{\#eq-total-constrained-multimodal-multiopp-access-per-capita\}

Per population \(P_i\) per zone \(i\) overall or a subset, as in population of mode user \(m\) or population of mode user for a specific park classification \(y\).

This can likewise be applied to market potential, representing population accessible per parkland area:
\[
m^{T}_{j} = M^{T}_{j} /O_{j}^{m}
\]\{\#eq-total-constrajned-access-per-capjta\}

\[
m^{mT}_{j} = M^{mT}_{j} /O_{j}^{m}
\]\{\#eq-total-constrajned-multjmodal-access-per-capjta\}

Per parkland \(O_j\) per zone \(i\) overall or a subset, as in opportunities reached by mode \(m\) or opportunities reached by mode of a specific park classification \(y\).

\section{Chapter conclusions}\label{chapter-conclusions-1}

\begin{itemize}
\tightlist
\item
\end{itemize}

\chapter{CHP 4 - Comparing total-, singly and unconstrained spatial access to Toronto's parklands}\label{chp-4---comparing-total--singly-and-unconstrained-spatial-access-to-torontos-parklands}

THIS CHAPTER WILL frames opportunities as hectares of parkland for one type, Large park, as all modes are present in the normative impedances. We will assume everyone walks, and hence the population of walkers is equal to the population at the origin. To keep it simple, just walking as a mode will be analysised, either 1 or 0 binary function at different thresholds.

These assumptions will yield values of \(V^0_i\), \(V^T_i\) and \(V^S_i\) as the `access to parkland'. As well as \(V^0_i\), \(V^T_i\) and \(V^S_i\)

\begin{longtable}[]{@{}
  >{\raggedleft\arraybackslash}p{(\columnwidth - 0\tabcolsep) * \real{0.0417}}@{}}
\toprule\noalign{}
\endhead
\bottomrule\noalign{}
\endlastfoot
In this analysis, four modes are considered: the motorized car and transit options, and the non-motorized cycling and walking options. From origins to destinations, each mode has a different travel impedance function \(f^m(\cdot)\) and travel time cost \(c^m_{ij}\) (note: \(c^m_{ij}\) and \(c^m_{ji}\) are assumed to be equal). The totally-constrained formula is modified as follows include a sub-index \(m\): \\
\(
V^{mT}_{ij} = K^{mT} \cdot W_j^{(2)} \cdot f^m(c^m_{ij})
\)\{\#eq-total-constrained-multimodal-access\} \\
Where:
- \(V^{mT}_{ij}\) is the number of opportunities that can be accessed at origin zone \(i\) from destination zone \(j\) by mode \(m\),
- \(f^m(c^m_{ij})\) is the cost of travel \(c^m_{ij}\) by mode \(m\) from \(i\) to \(j\),
- the destination zone attraction mass \(W_j^{(2)}\) is the number of opportunities (i.e., the parkland in hectares at a park \(j\)), and
- \(K^{mT}\) is the modal total constraint \(\frac{D}{\sum_m\sum_i\sum_j W^{(2)}_jf^m(c^m_{ij})}\) that serves to proportionally allocate the opportunities \(D\) in the region and ensures units remain balanced. \\
Summarising equation \ref{eq:total-constrained-multimodal-access} as a measure of modal totally-constrained accessibility \(V^{mT}_i\) by summing all \(V^{mT}_{ij}\) for a specific \(i\) and \(m\) (i.e., \(V^{mT}_i = \sum_j V^{mT}_{ij}\)). \(V^{mT}_i\) can also be summed by mode to equal \(V^{T}_i\) (i.e., \(\sum_m V^{mT}_i = V^{T}_i\)) and summed across the region to equal \(D\) (i.e., \(\sum_m\sum_i\sum_j V^{mT}_{ij} = \sum_m\sum_i V^{mT}_{i} = D\)). \\
Transposing \(i\) and \(j\) and expressing \(M^{mT}_{ji}\), the multimodal `market potential' or the population accessible by mode can also be defined using the totally-constrained formulation, with similar parameters as previously defined.
\(
M^{mT}_{ji} = \hat K^{mT} \cdot W_i^{(1)} f^m(c^m_{ji})
\) \{\#eq-total-constrained-multimodal-market\} \\
\end{longtable}

\section{Results}\label{results}

The figure demonstrates the accessible to all parks for each DB (recall, out of the total 1607 parks), as an overview of the city.
{[}1 Toronto DB plot - featuring accessibility to all parks considering `average' modal impedance{]}

If interested in the coverage of the parks, the following figure demonstrates the accessibility to people for each DB, in the context of their travel to parks.
{[}1 Toronto DB plot - featuring accessibility to all people considering `average' modal impedance{]}

\subsection{Potential access to parks}\label{potential-access-to-parks}

Focusing on the accessible number of parks: the following plot demonstrates a disaggregated version of figure above. Each plot demonstrates the number of parks that are accessible, by mode. Again, the embedded assumption is the number of parkland in the city is allocated to each DB based on the travel impedance of that mode's zone relative to all the travel impedance of all modes in the region.
{[}4 Toronto DB plots - featuring accessibility to all parks by mode{]}

{[}4 Toronto DB plots, summarised by neighbourhood - featuring accessibility to ONLY large parks by mode{]}

{[}4 Toronto DB plots, summarised by neighbourhood - featuring accessibility to ONLY small parks by mode{]}

{[}4 Toronto DB plots - featuring accessibility to all parks per capita by mode{]}

{[}summary table of top 5, middle 5, and bottom 5 neighbourhoods{]}

\subsection{Potential access to peopulation}\label{potential-access-to-peopulation}

If interested in using totally-constrained accessibility as an indicator of service provision, conceptualing the `market pontential' variant is useful to yield values of `number of people that can access parks' for the zonal unit of question.

The following is the access to people, by mode. Notably, it is not correlated with opportunities, as there is a mis-match in population and parkland in certain neighbourhood (see fig..X)
{[}4 Toronto DB plots - featuring accessibility to all people by mode{]}

The following is the access to people, by mode:
{[}4 Toronto DB plots, summarised by neighbourhood - featuring accessibility to people from ONLY large parks by mode{]}

{[}4 Toronto DB plots, summarised by neighbourhood - featuring accessibility to people from ONLY small parks by mode{]}

{[}4 Toronto DB plots - featuring accessibility to all people per parkland capita by mode{]}

{[}summary table of top 5, middle 5, and bottom 5 neighbourhoods{]}

\chapter{CHP 5 - Comparing multimodal total-, singly and unconstrained spatial access to Toronto's parklands}\label{chp-5---comparing-multimodal-total--singly-and-unconstrained-spatial-access-to-torontos-parklands}

THIS CHAPTER WILL \ldots{}

\chapter*{Conclusion}\label{conclusion}
\addcontentsline{toc}{chapter}{Conclusion}

Concluding\ldots{}

\begin{itemize}
\tightlist
\item
  open and reproducibility of methods learned are discussed (text from \emph{``TTS2016R: A data set to study population and employment patterns from the 2016 Transportation Tomorrow Survey in the Greater Golden Horseshoe area, Ontario, Canada''} is included)
\item
  how constrained measures can be interpreted,
\item
  how this interpretation can aid policy makers-- per capita, per any other zonal property. comparisions between regions, between times.
\item
  how policy makers can ultimately use this to plan for inequities (some text pulled from \emph{``Searching for fairness standards in the transportation literature''}).
\item
  Future research directions (?)
\end{itemize}

DESTINATIONS -- in flux in their competition.
- hyper competitive (limited opps + required qualitifcations)-- like employment, and post-secondary education
- competitive and institional (imited opps creates competitive experiences, open to and essential for whole population/subgroup) -- like primary, and sceondary education, child care centre, healthcare facilities

\backmatter

\chapter*{References}\label{references}
\addcontentsline{toc}{chapter}{References}

\markboth{References}{References}

\noindent

\setlength{\parindent}{-0.20in}
\setlength{\leftskip}{0.20in}
\setlength{\parskip}{8pt}

\phantomsection\label{refs}
\begin{CSLReferences}{1}{0}
\bibitem[\citeproctext]{ref-albacete2017measuring}
Albacete, X., Olaru, D., Paül, V., \& Biermann, S. (2017). Measuring the accessibility of public transport: A critical comparison between methods in helsinki. \emph{Applied Spatial Analysis and Policy}, \emph{10}, 161--188.

\bibitem[\citeproctext]{ref-allenMeasureCompetitiveAccess2020}
Allen, J., \& Farber, S. (2020). paez2019 to destinations for comparing across multiple study regions. \emph{Geographical Analysis}, \emph{52}(1), 69--86. http://doi.org/\href{https://doi.org/10.1111/gean.12188}{10.1111/gean.12188}

\bibitem[\citeproctext]{ref-alonso2014labour}
Alonso, M. P., Beamonte, M., Gargallo, P., \& Salvador, M. J. (2014). Labour and residential accessibility: A bayesian analysis based on poisson gravity models with spatial effects. \emph{Journal of Geographical Systems}, \emph{16}, 409--439.

\bibitem[\citeproctext]{ref-ashiru2003space}
Ashiru, O., Polak, J. W., \& Noland, R. B. (2003). Space-time user benefit and utility accessibility measures for individual activity schedules. \emph{Transportation Research Record}, \emph{1854}(1), 62--73.

\bibitem[\citeproctext]{ref-battyChronicleScientificPlanning1994}
Batty, Michael. (1994). A {Chronicle} of {Scientific} {Planning}: {The} {Anglo}-{American} {Modeling} {Experience}. \emph{Journal of the American Planning Association}, \emph{60}(1), 7--16. http://doi.org/\href{https://doi.org/10.1080/01944369408975546}{10.1080/01944369408975546}

\bibitem[\citeproctext]{ref-battyMethodResiduesUrban1976}
Batty, M., \& March, L. (1976). The method of residues in urban modelling. \emph{Environment and Planning A: Economy and Space}, \emph{8}(2), 189--214. http://doi.org/\href{https://doi.org/10.1068/a080189}{10.1068/a080189}

\bibitem[\citeproctext]{ref-beaumontLocationallocationProblemsPlane1981}
Beaumont, J. R. (1981). Location-allocation problems in a plane a review of some models. \emph{Socio-Economic Planning Sciences}, \emph{15}(5), 217--229. http://doi.org/\href{https://doi.org/10.1016/0038-0121(81)90042-2}{10.1016/0038-0121(81)90042-2}

\bibitem[\citeproctext]{ref-beckers2022incorporating}
Beckers, J., Birkin, M., Clarke, G., Hood, N., Newing, A., \& Urquhart, R. (2022). Incorporating e-commerce into retail location models. \emph{Geographical Analysis}, \emph{54}(2), 274--293.

\bibitem[\citeproctext]{ref-boisjoly2017informality}
Boisjoly, G., Moreno-Monroy, A. I., \& El-Geneidy, A. (2017). Informality and accessibility to health by public transit: Evidence from the são paulo metropolitan region. \emph{Journal of Transport Geography}, \emph{64}, 89--96. Journal Article. http://doi.org/\href{https://doi.org/10.1016/j.jtrangeo.2017.08.005}{10.1016/j.jtrangeo.2017.08.005}

\bibitem[\citeproctext]{ref-campbell2019accessibility}
Campbell, K. B., Rising, J. A., Klopp, J. M., \& Mbilo, J. M. (2019). Accessibility across transport modes and residential developments in nairobi. \emph{Journal of Transport Geography}, \emph{74}, 77--90.

\bibitem[\citeproctext]{ref-careyPrinciplesSocialScience1858}
Carey, H. C. (1858). \emph{Principles of social science}. University of Michigan Library Digital Collections: In the digital collection Making of America Books. Retrieved from \url{https://name.umdl.umich.edu/AFR1829.0001.001.}

\bibitem[\citeproctext]{ref-caschili2015accessibility}
Caschili, S., De Montis, A., \& Trogu, D. (2015). Accessibility and rurality indicators for regional development. \emph{Computers, Environment and Urban Systems}, \emph{49}, 98--114.

\bibitem[\citeproctext]{ref-cavendish_xxi_1798}
Cavendish, H. (1798). {XXI}. {Experiments} to determine the density of the earth. \emph{Philosophical Transactions of the Royal Society}, \emph{88}, 469--526. http://doi.org/\href{https://doi.org/10.1098/rstl.1798.0022}{10.1098/rstl.1798.0022}

\bibitem[\citeproctext]{ref-chen2013regional}
Chen, G., \& Silva, J. de A. e. (2013). Regional impacts of high-speed rail: A review of methods and models. \emph{Transportation Letters}, \emph{5}(3), 131--143.

\bibitem[\citeproctext]{ref-chia2020extending}
Chia, J., \& Lee, J. B. (2020). Extending public transit accessibility models to recognise transfer location. \emph{Journal of Transport Geography}, \emph{82}, 102618.

\bibitem[\citeproctext]{ref-toronto_NEI_2014}
City of Toronto. (2014). TSNS 2020 neighbourhood equity index methodological documentation. City of Toronto - Social Policy Analysis; Research. Retrieved from \url{https://www.toronto.ca/wp-content/uploads/2017/11/97eb-TSNS-2020-NEI-equity-index-methodology-research-report-backgroundfile-67350.pdf}

\bibitem[\citeproctext]{ref-toronto_parkland2019}
City of Toronto. (2019). Parkland strategy: Growing toronto parkland. Toronto, Ontario: City of Toronto - Parks, Forestry; Recreation. Retrieved from \url{https://www.toronto.ca/wp-content/uploads/2019/11/97fb-parkland-strategy-full-report-final.pdf}

\bibitem[\citeproctext]{ref-toronto_neighbourhoods2024}
City of Toronto. (2024). Neighbourhoods (Version Oct 15, 2024). City of Toronto. Retrieved from \url{https://open.toronto.ca/dataset/neighbourhoods/}

\bibitem[\citeproctext]{ref-toronto_greenspaces2025}
City of Toronto. (2025). Green spaces in toronto (Version Apr 29, 2025). City of Toronto. Retrieved from \url{https://open.toronto.ca/dataset/green-spaces/}

\bibitem[\citeproctext]{ref-clarke2002deriving}
Clarke, G., Eyre, H., \& Guy, C. (2002). Deriving indicators of access to food retail provision in british cities: Studies of cardiff, leeds and bradford. \emph{Urban Studies}, \emph{39}(11), 2041--2060.

\bibitem[\citeproctext]{ref-cliff_evaluating_1974}
Cliff, A. D., Martin, R. L., \& Ord, J. K. (1974). Evaluating the friction of distance parameter in gravity models. \emph{Regional Studies}, \emph{8}(3-4), 281--286. http://doi.org/\href{https://doi.org/10.1080/09595237400185281}{10.1080/09595237400185281}

\bibitem[\citeproctext]{ref-condecco2018road}
Condeço-Melhorado, A., \& Christidis, P. (2018). Road accessibility in border regions: A joint approach. \emph{Networks and Spatial Economics}, \emph{18}(2), 363--383.

\bibitem[\citeproctext]{ref-cuiSpatialAccessPublic2020}
Cui, B., Boisjoly, G., Wasfi, R., Orpana, H., Manaugh, K., Buliung, R., \ldots{} El-Geneidy, A. (2020). Spatial access by public transport and likelihood of healthcare consultations at hospitals. \emph{{TRANSPORTATION} {RESEARCH} {RECORD}}, \emph{2674}(12), 188--198. http://doi.org/\href{https://doi.org/10.1177/0361198120952793}{10.1177/0361198120952793}

\bibitem[\citeproctext]{ref-curtis2010planning}
Curtis, C., \& Scheurer, J. (2010). Planning for sustainable accessibility: Developing tools to aid discussion and decision-making. \emph{Progress in Planning}, \emph{74}(2), 53--106.

\bibitem[\citeproctext]{ref-dai2017visualization}
Dai, L., Wan, L., \& Gai, S. (2017). A visualization framework for synthesizing spatial impacts from multiple site factors. \emph{Journal of Asian Architecture and Building Engineering}, \emph{16}(2), 311--315.

\bibitem[\citeproctext]{ref-data_management_group_tts_2023}
Data Management Group. (2023). {TTS} - {Transportation} {Tomorrow} {Survey} 2023. Retrieved from \url{https://tts2023.ca/en/index.php}

\bibitem[\citeproctext]{ref-delamater2013spatial}
Delamater, P. L. (2013). Spatial accessibility in suboptimally configured health care systems: A modified two-step floating catchment area (M2SFCA) metric. \emph{Health \& Place}, \emph{24}, 30--43. Journal Article. http://doi.org/\href{https://doi.org/10.1016/j.healthplace.2013.07.012}{10.1016/j.healthplace.2013.07.012}

\bibitem[\citeproctext]{ref-el2011place}
El-Geneidy, A., \& Levinson, D. (2011). Place rank: Valuing spatial interactions. \emph{Networks and Spatial Economics}, \emph{11}, 643--659.

\bibitem[\citeproctext]{ref-elgeneidyMakingAccessibilityWork2022}
El-Geneidy, A., \& Levinson, D. (2022). Making accessibility work in practice. \emph{Transport Reviews}, \emph{42}(2), 129--133. http://doi.org/\href{https://doi.org/10.1080/01441647.2021.1975954}{10.1080/01441647.2021.1975954}

\bibitem[\citeproctext]{ref-farber_2013_social}
Farber, S., Neutens, T., Miller, H. J., \& Li, X. (2013). The social interaction potential of metropolitan regions: A time-geographic measurement approach using joint accessibility. \emph{Annals of the Association of American Geographers}, \emph{103}(3), 483--504. Journal Article. http://doi.org/\href{https://doi.org/10.1080/00045608.2012.689238}{10.1080/00045608.2012.689238}

\bibitem[\citeproctext]{ref-farber_running_2011}
Farber, Steven, \& Páez, A. (2011). Running to stay in place: The time-use implications of automobile oriented land-use and travel. \emph{Journal of Transport Geography}, \emph{19}(4), 782--793. http://doi.org/\href{https://doi.org/10.1016/j.jtrangeo.2010.09.008}{10.1016/j.jtrangeo.2010.09.008}

\bibitem[\citeproctext]{ref-farberActivitySpacesMeasurement2012}
Farber, S., Páez, A., \& Morency, C. (2012). Activity spaces and the measurement of clustering and exposure: A case study of linguistic groups in {Montreal}. \emph{Environment and Planning A}, \emph{44}(2), 315--332.

\bibitem[\citeproctext]{ref-ferreiraReenactingMobilityAccessibility2020}
Ferreira, A., \& Papa, E. (2020). Re-enacting the mobility versus accessibility debate: Moving towards collaborative synergies among experts. \emph{Case Studies on Transport Policy}, \emph{8}(3), 1002--1009. http://doi.org/\href{https://doi.org/10.1016/j.cstp.2020.04.006}{10.1016/j.cstp.2020.04.006}

\bibitem[\citeproctext]{ref-fotheringhamSPATIALSTRUCTUREDISTANCE1981}
Fotheringham, A. Stewart. (1981). {SPATIAL} {STRUCTURE} {AND} {DISTANCE}‐{DECAY} {PARAMETERS}. \emph{Annals of the Association of American Geographers}, \emph{71}(3), 425--436. http://doi.org/\href{https://doi.org/10.1111/j.1467-8306.1981.tb01367.x}{10.1111/j.1467-8306.1981.tb01367.x}

\bibitem[\citeproctext]{ref-fotheringham_spatial_1984}
Fotheringham, A. S. (1984). Spatial {Flows} and {Spatial} {Patterns}. \emph{Environment and Planning A: Economy and Space}, \emph{16}(4), 529--543. http://doi.org/\href{https://doi.org/10.1068/a160529}{10.1068/a160529}

\bibitem[\citeproctext]{ref-fotheringhamSpatialCompetitionAgglomeration1985}
Fotheringham, A. S. (1985). Spatial competition and agglomeration in urban modelling. \emph{Environment and Planning A: Economy and Space}, \emph{17}(2), 213--230. http://doi.org/\href{https://doi.org/10.1068/a170213}{10.1068/a170213}

\bibitem[\citeproctext]{ref-geurs2004}
Geurs, Karst T., \& van Wee, B. (2004a). Accessibility evaluation of land-use and transport strategies: review and research directions. \emph{Journal of Transport Geography}, \emph{12}(2), 127--140. http://doi.org/\href{https://doi.org/10.1016/j.jtrangeo.2003.10.005}{10.1016/j.jtrangeo.2003.10.005}

\bibitem[\citeproctext]{ref-geursAccessibilityEvaluationLanduse2004}
Geurs, Karst T., \& van Wee, B. (2004b). Accessibility evaluation of land-use and transport strategies: Review and research directions. \emph{Journal of Transport Geography}, \emph{12}(2), 127--140. http://doi.org/\href{https://doi.org/10.1016/j.jtrangeo.2003.10.005}{10.1016/j.jtrangeo.2003.10.005}

\bibitem[\citeproctext]{ref-geurs2006accessibility}
Geurs, Karst T., van Wee, B., \& Rietveld, P. (2006). Accessibility appraisal of integrated land-use---transport strategies: Methodology and case study for the netherlands randstad area. \emph{Environment and Planning B: Planning and Design}, \emph{33}(5), 639--660.

\bibitem[\citeproctext]{ref-giuliano2010accessibility}
Giuliano, G., Gordon, P., Pan, Q., \& Park, J. (2010). Accessibility and residential land values: Some tests with new measures. \emph{Urban Studies}, \emph{47}(14), 3103--3130.

\bibitem[\citeproctext]{ref-grengs2004measuring}
Grengs, J. (2004). Measuring change in small-scale transit accessibility with geographic information systems: Buffalo and rochester, new york. \emph{Transportation Research Record}, \emph{1887}(1), 10--17.

\bibitem[\citeproctext]{ref-grengs2010job}
Grengs, J. (2010a). Job accessibility and the modal mismatch in detroit. \emph{Journal of Transport Geography}, \emph{18}(1), 42--54.

\bibitem[\citeproctext]{ref-grengsJobAccessibilityModal2010}
Grengs, J. (2010b). Job accessibility and the modal mismatch in detroit. \emph{Journal of Transport Geography}, \emph{18}(1), 42--54. http://doi.org/\href{https://doi.org/10.1016/j.jtrangeo.2009.01.012}{10.1016/j.jtrangeo.2009.01.012}

\bibitem[\citeproctext]{ref-grengs2012equity}
Grengs, J. (2012). Equity and the social distribution of job accessibility in detroit. \emph{Environment and Planning B: Planning and Design}, \emph{39}(5), 785--800.

\bibitem[\citeproctext]{ref-grengs2015nonwork}
Grengs, J. (2015). Nonwork accessibility as a social equity indicator. \emph{International Journal of Sustainable Transportation}, \emph{9}(1), 1--14.

\bibitem[\citeproctext]{ref-grengs2010intermetropolitan}
Grengs, J., Levine, J., Shen, Q., \& Shen, Q. (2010). Intermetropolitan comparison of transportation accessibility: Sorting out mobility and proximity in san francisco and washington, DC. \emph{Journal of Planning Education and Research}, \emph{29}(4), 427--443.

\bibitem[\citeproctext]{ref-gutierrez_evaluating_2011}
Gutierrez, J., Condeco-Melhorado, A., Lopez, E., \& Monzon, A. (2011). Evaluating the {European} added value of {TEN}-{T} projects: A methodological proposal based on spatial spillovers, accessibility and {GIS}. \emph{Journal of Transport Geography}, \emph{19}(4), 840--850. http://doi.org/\href{https://doi.org/10.1016/j.jtrangeo.2010.10.011}{10.1016/j.jtrangeo.2010.10.011}

\bibitem[\citeproctext]{ref-gutierrezLocationEconomicPotential2001}
Gutiérrez, J. (2001). Location, economic potential and daily accessibility: An analysis of the accessibility impact of the high-speed line madrid--barcelona--french border. \emph{Journal of Transport Geography}, \emph{9}(4), 229--242. http://doi.org/\href{https://doi.org/10.1016/S0966-6923(01)00017-5}{10.1016/S0966-6923(01)00017-5}

\bibitem[\citeproctext]{ref-handyACCESSIBILITYVSMOBILITYENHANCING2002}
Handy, S. (2002). \emph{{ACCESSIBILITY}- {VS}. {MOBILITY}-{ENHANCING} {STRATEGIES} {FOR} {ADDRESSING} {AUTOMOBILE} {DEPENDENCE} {IN} {THE} u.s.} Retrieved from \url{https://escholarship.org/content/qt5kn4s4pb/qt5kn4s4pb_noSplash_18f73162ff86f04dcb255331d63eeba8.pdf}

\bibitem[\citeproctext]{ref-handy2020}
Handy, S. (2020). Is accessibility an idea whose time has finally come? \emph{Transportation Research Part D: Transport and Environment}, \emph{83}, 102319. http://doi.org/\href{https://doi.org/10.1016/j.trd.2020.102319}{10.1016/j.trd.2020.102319}

\bibitem[\citeproctext]{ref-handyMeasuringAccessibilityExploration1997}
Handy, S. L., \& Niemeier, D. A. (1997). Measuring accessibility: An exploration of issues and alternatives. \emph{Environment and Planning A: Economy and Space}, \emph{29}(7), 1175--1194. http://doi.org/\href{https://doi.org/10.1068/a291175}{10.1068/a291175}

\bibitem[\citeproctext]{ref-hansen1959}
Hansen, W. G. (1959). How Accessibility Shapes Land Use. \emph{Journal of the American Institute of Planners}, \emph{25}(2), 73--76. http://doi.org/\href{https://doi.org/10.1080/01944365908978307}{10.1080/01944365908978307}

\bibitem[\citeproctext]{ref-harrisEquilibriumValuesDynamics1978}
Harris, B., \& Wilson, A. G. (1978). Equilibrium values and dynamics of attractiveness terms in production-constrained spatial-interaction models. \emph{Environment and Planning A: Economy and Space}, \emph{10}(4), 371--388. http://doi.org/\href{https://doi.org/10.1068/a100371}{10.1068/a100371}

\bibitem[\citeproctext]{ref-harris_market_1954}
Harris, C. D. (1954). The {Market} as a {Factor} in the {Localization} of {Industry} in the {United} {States}. \emph{Annals of the Association of American Geographers}, \emph{44}(4), 315--348. Retrieved from \url{https://www.jstor.org/stable/2561395}

\bibitem[\citeproctext]{ref-he2017measuring}
He, J., Li, C., Yu, Y., Liu, Y., \& Huang, J. (2017). Measuring urban spatial interaction in wuhan urban agglomeration, central china: A spatially explicit approach. \emph{Sustainable Cities and Society}, \emph{32}, 569--583.

\bibitem[\citeproctext]{ref-higgsModellingSpatialAccess2017}
Higgs, G., Zahnow, R., Corcoran, J., Langford, M., \& Fry, R. (2017). Modelling spatial access to general practitioner surgeries: Does public transport availability matter? \emph{{JOURNAL} {OF} {TRANSPORT} \& {HEALTH}}, \emph{6}, 143--154. http://doi.org/\href{https://doi.org/10.1016/j.jth.2017.05.361}{10.1016/j.jth.2017.05.361}

\bibitem[\citeproctext]{ref-holl2007twenty}
Holl, A. (2007). Twenty years of accessibility improvements. The case of the spanish motorway building programme. \emph{Journal of Transport Geography}, \emph{15}(4), 286--297.

\bibitem[\citeproctext]{ref-hutton_xxxiii_1778}
Hutton, C. (1778). {XXXIII}. {An} account of the calculations made from the survey and measures taken at {Schehallien}, in order to ascertain the mean density of the {Earth}. \emph{Philosophical Transactions of the Royal Society}, \emph{68}, 689--788. http://doi.org/\href{https://doi.org/10.1098/rstl.1778.0034}{10.1098/rstl.1778.0034}

\bibitem[\citeproctext]{ref-josephMeasuringPotentialPhysical1982}
Joseph, A. E., \& Bantock, P. R. (1982). Measuring potential physical accessibility to general practitioners in rural areas: A method and case study. \emph{Social Science \& Medicine}, \emph{16}(1), 85--90. http://doi.org/\href{https://doi.org/10.1016/0277-9536(82)90428-2}{10.1016/0277-9536(82)90428-2}

\bibitem[\citeproctext]{ref-kapatsila_resolving_2023}
Kapatsila, B., Palacios, M. S., Grisé, E., \& El-Geneidy, A. (2023). Resolving the accessibility dilemma: {Comparing} cumulative and gravity-based measures of accessibility in eight {Canadian} cities. \emph{Journal of Transport Geography}, \emph{107}, 103530. http://doi.org/\href{https://doi.org/10.1016/j.jtrangeo.2023.103530}{10.1016/j.jtrangeo.2023.103530}

\bibitem[\citeproctext]{ref-karstEvaluationAccessibilityImpacts2003}
Karst, T., \& Van Eck, J. R. R. (2003). Evaluation of accessibility impacts of land-use scenarios: The implications of job competition, land-use, and infrastructure developments for the netherlands. \emph{Environment and Planning B: Planning and Design}, \emph{30}(1), 69--87. http://doi.org/\href{https://doi.org/10.1068/b12940}{10.1068/b12940}

\bibitem[\citeproctext]{ref-kawabataJobAccessibilityIndicator2006}
Kawabata, M., \& Shen, Q. (2006). Job accessibility as an indicator of auto-oriented urban structure: A comparison of boston and los angeles with tokyo. \emph{Environment and Planning B: Planning and Design}, \emph{33}(1), 115--130. http://doi.org/\href{https://doi.org/10.1068/b31144}{10.1068/b31144}

\bibitem[\citeproctext]{ref-kelobonye2020measuring}
Kelobonye, K., Zhou, H., McCarney, G., \& Xia, J. (2020). Measuring the accessibility and spatial equity of urban services under competition using the cumulative opportunities measure. \emph{Journal of Transport Geography}, \emph{85}, 102706. Journal Article. http://doi.org/\url{https://doi.org/10.1016/j.jtrangeo.2020.102706}

\bibitem[\citeproctext]{ref-kharel2024examining}
Kharel, S., Sharifiasl, S., \& Pan, Q. (2024). Examining food access equity by integrating grocery store pricing into spatial accessibility measures. \emph{Transportation Research Record}, 03611981241254382.

\bibitem[\citeproctext]{ref-kirbyNormalizingFactorsGravity1970}
Kirby, H. R. (1970). Normalizing factors of the gravity model---an interpretation. \emph{Transportation Research}, \emph{4}(1), 37--50. http://doi.org/\href{https://doi.org/10.1016/0041-1647(70)90073-0}{10.1016/0041-1647(70)90073-0}

\bibitem[\citeproctext]{ref-kovatch1971modeling}
Kovatch, G., Zames, G., et al. (1971). Modeling transportation systems: An overview. Retrieved from \url{https://rosap.ntl.bts.gov/view/dot/11813}

\bibitem[\citeproctext]{ref-kwan1998space}
Kwan, M.-P. (1998). Space-time and integral measures of individual accessibility: A comparative analysis using a point-based framework. \emph{Geographical Analysis}, \emph{30}(3), 191--216.

\bibitem[\citeproctext]{ref-kwokUseModalAccessibility2004}
Kwok, R. C. W., \& Yeh, A. G. O. (2004). The use of modal accessibility gap as an indicator for sustainable transport development. \emph{Environment and Planning A: Economy and Space}, \emph{36}(5), 921--936. http://doi.org/\href{https://doi.org/10.1068/a3673}{10.1068/a3673}

\bibitem[\citeproctext]{ref-laraSpace2015}
Lara-Valencia, F., \& García-Pérez, H. (2015). Space for equity: Socioeconomic variations in the provision of public parks in hermosillo, mexico. \emph{Local Environment}, \emph{20}(3), 350--368. Journal Article. http://doi.org/\href{https://doi.org/10.1080/13549839.2013.857647}{10.1080/13549839.2013.857647}

\bibitem[\citeproctext]{ref-lavery_driving_2013}
Lavery, T. A., Páez, A., \& Kanaroglou, P. S. (2013). Driving out of choices: {An} investigation of transport modality in a university sample. \emph{Transportation Research Part A: Policy and Practice}, \emph{57}, 37--46. http://doi.org/\href{https://doi.org/10.1016/j.tra.2013.09.010}{10.1016/j.tra.2013.09.010}

\bibitem[\citeproctext]{ref-leonardiOptimumFacilityLocation1978}
Leonardi, G. (1978). Optimum facility location by accessibility maximizing. \emph{Environment and Planning A: Economy and Space}, \emph{10}(11), 1287--1305. http://doi.org/\href{https://doi.org/10.1068/a101287}{10.1068/a101287}

\bibitem[\citeproctext]{ref-leonardiRandomUtilityDemand1984}
Leonardi, Giorgio, \& Tadei, R. (1984). Random utility demand models and service location. \emph{Regional Science and Urban Economics}, \emph{14}(3), 399--431. http://doi.org/\href{https://doi.org/10.1016/0166-0462(84)90009-7}{10.1016/0166-0462(84)90009-7}

\bibitem[\citeproctext]{ref-levine2012does}
Levine, J., Grengs, J., Shen, Q., \& Shen, Q. (2012). Does accessibility require density or speed? A comparison of fast versus close in getting where you want to go in US metropolitan regions. \emph{Journal of the American Planning Association}, \emph{78}(2), 157--172.

\bibitem[\citeproctext]{ref-levinson2012positive}
Levinson, D., \& Huang, A. (2012). A positive theory of network connectivity. \emph{Environment and Planning B: Planning and Design}, \emph{39}(2), 308--325.

\bibitem[\citeproctext]{ref-levinsonGeneralTheoryAccess2020}
Levinson, D., \& Wu, H. (2020). Towards a general theory of access. \emph{Journal of Transport and Land Use}, \emph{13}(1), 129--158. Retrieved from \url{https://www.jstor.org/stable/26967239}

\bibitem[\citeproctext]{ref-liMeasuringMultiactivitiesAccessibility2024a}
Li, C., \& Wang, J. (2024). Measuring multi-activities accessibility and equity with accessibility-oriented development strategies. \emph{Transportation Research Part D: Transport and Environment}, \emph{126}, 104035. http://doi.org/\href{https://doi.org/10.1016/j.trd.2023.104035}{10.1016/j.trd.2023.104035}

\bibitem[\citeproctext]{ref-liang_novel_2024}
Liang, H., Yan, Q., \& Yan, Y. (2024). A novel spatiotemporal framework for accessing green space accessibility change in adequacy and equity: {Evidence} from a rapidly urbanizing {Chinese} {City} in 2012--2021. \emph{Cities}, \emph{151}, 105112. http://doi.org/\href{https://doi.org/10.1016/j.cities.2024.105112}{10.1016/j.cities.2024.105112}

\bibitem[\citeproctext]{ref-liottaPlanning2020}
Liotta, C., Kervinio, Y., Levrel, H., \& Tardieu, L. (2020). Planning for environmental justice - reducing well-being inequalities through urban greening. \emph{Environmental Science \& Policy}, \emph{112}, 47--60. Journal Article. http://doi.org/\url{https://doi.org/10.1016/j.envsci.2020.03.017}

\bibitem[\citeproctext]{ref-liu2004accessibility}
Liu, S., \& Zhu, X. (2004). Accessibility analyst: An integrated GIS tool for accessibility analysis in urban transportation planning. \emph{Environment and Planning B: Planning and Design}, \emph{31}(1), 105--124.

\bibitem[\citeproctext]{ref-liu2015spatial}
Liu, X., \& Zhou, J. (2015). Spatial pattern of land use and its implications for mode-based accessibility: Case study of nanjing, china. \emph{Journal of Urban Planning and Development}, \emph{141}(2), 05014012.

\bibitem[\citeproctext]{ref-lopezMeasuring2008}
Lopez, E., Gutierrez, J., \& Gomez, G. (2008). Measuring regional cohesion effects of large-scale transport infrastructure investments: {An} accessibility approach. \emph{European Planning Studies}, \emph{16}(2), 277--301.

\bibitem[\citeproctext]{ref-lunkeModalAccessibilityDisparities2022}
Lunke, E. B. (2022). Modal accessibility disparities and transport poverty in the oslo region. \emph{Transportation Research Part D: Transport and Environment}, \emph{103}, 103171. http://doi.org/\href{https://doi.org/10.1016/j.trd.2022.103171}{10.1016/j.trd.2022.103171}

\bibitem[\citeproctext]{ref-luoEnhanced2009}
Luo, W., \& Qi, Y. (2009). An enhanced two-step floating catchment area ({E2SFCA}) method for measuring spatial accessibility to primary care physicians. \emph{Health \& Place}, \emph{15}(4), 1100--1107.

\bibitem[\citeproctext]{ref-luo2003}
Luo, W., \& Wang, F. (2003a). Measures of Spatial Accessibility to Health Care in a GIS Environment: Synthesis and a Case Study in the Chicago Region. \emph{Environment and Planning B: Planning and Design}, \emph{30}(6), 865--884. http://doi.org/\href{https://doi.org/10.1068/b29120}{10.1068/b29120}

\bibitem[\citeproctext]{ref-luoMeasuresSpatialAccessibility2003}
Luo, W., \& Wang, F. (2003b). Measures of spatial accessibility to health care in a {GIS} environment: Synthesis and a case study in the chicago region. \emph{Environment and Planning B: Planning and Design}, \emph{30}(6), 865--884. http://doi.org/\href{https://doi.org/10.1068/b29120}{10.1068/b29120}

\bibitem[\citeproctext]{ref-maharjanSpatialEquityModal2022}
Maharjan, S., Tilahun, N., \& Ermagun, A. (2022). Spatial equity of modal access gap to multiple destination types across chicago. \emph{Journal of Transport Geography}, \emph{104}, 103437. http://doi.org/\href{https://doi.org/10.1016/j.jtrangeo.2022.103437}{10.1016/j.jtrangeo.2022.103437}

\bibitem[\citeproctext]{ref-manaugh2012makes}
Manaugh, K., \& El-Geneidy, A. M. (2012). What makes travel'local' defining and understanding local travel behavior. \emph{Journal of Transport and Land Use}, \emph{5}(3), 15--27.

\bibitem[\citeproctext]{ref-maoMeasuringSpatialAccessibility2013}
Mao, L., \& Nekorchuk, D. (2013). Measuring spatial accessibility to healthcare for populations with multiple transportation modes. \emph{Health \& Place}, \emph{24}, 115--122. http://doi.org/\href{https://doi.org/10.1016/j.healthplace.2013.08.008}{10.1016/j.healthplace.2013.08.008}

\bibitem[\citeproctext]{ref-margaridacondecomelhoradoImpactMeasuringInternal2016}
Margarida Condeço Melhorado, A., Demirel, H., Kompil, M., Navajas, E., \& Christidis, P. (2016). The impact of measuring internal travel distances on selfpotentials and accessibility. \emph{European Journal of Transport and Infrastructure Research}. http://doi.org/\href{https://doi.org/10.18757/EJTIR.2016.16.2.3139}{10.18757/EJTIR.2016.16.2.3139}

\bibitem[\citeproctext]{ref-marques_accessibility_2021}
Marques, J. L., Wolf, J., \& Feitosa, F. (2021). Accessibility to primary schools in {Portugal}: A case of spatial inequity? \emph{Regional Science Policy \& Practice}, \emph{13}(3), 693--708. http://doi.org/\href{https://doi.org/10.1111/rsp3.12303}{10.1111/rsp3.12303}

\bibitem[\citeproctext]{ref-martensFair2021}
Martens, K., \& Golub, A. (2021). A fair distribution of accessibility: Interpreting civil rights regulations for regional transportation plans. \emph{Journal of Planning Education and Research}, \emph{41}(4), 425--444. Journal Article. http://doi.org/\href{https://doi.org/10.1177/0739456x18791014}{10.1177/0739456x18791014}

\bibitem[\citeproctext]{ref-marwal2022literature}
Marwal, A., \& Silva, E. (2022). Literature review of accessibility measures and models used in land use and transportation planning in last 5 years. \emph{Journal of Geographical Sciences}, \emph{32}(3), 560--584.

\bibitem[\citeproctext]{ref-mayaud2019future}
Mayaud, J. R., Tran, M., Pereira, R. H., \& Nuttall, R. (2019). Future access to essential services in a growing smart city: The case of surrey, british columbia. \emph{Computers, Environment and Urban Systems}, \emph{73}, 1--15.

\bibitem[\citeproctext]{ref-mckeanManual1883}
McKean, K. (1883). \emph{Manual of {Social} {Science} being a {Condensation} of the {Principles} of {Social} {Science} of {H}.{C}. {Carey}}. Philadelphia: Henry Carey Baird; Co. Industrial Publishers.

\bibitem[\citeproctext]{ref-mdotMnDOTJoins2007}
MDOT. (2007, December 4). Mn/{DOT} joins interstate highway system's 50th anniversary celebration. Retrieved February 3, 2025, from \url{https://web.archive.org/web/20071204072603/http://www.dot.state.mn.us/interstate50/50facts.html}

\bibitem[\citeproctext]{ref-merlin2017competition}
Merlin, L. A., \& Hu, L. (2017). Does competition matter in measures of job accessibility? Explaining employment in los angeles. \emph{Journal of Transport Geography}, \emph{64}, 77--88. Journal Article. http://doi.org/\href{https://doi.org/10.1016/j.jtrangeo.2017.08.009}{10.1016/j.jtrangeo.2017.08.009}

\bibitem[\citeproctext]{ref-millerAccessibilityMeasurementApplication2018}
Miller, E. J. (2018). Accessibility: Measurement and application in transportation planning. \emph{Transport Reviews}, \emph{38}(5), 551--555. http://doi.org/\href{https://doi.org/10.1080/01441647.2018.1492778}{10.1080/01441647.2018.1492778}

\bibitem[\citeproctext]{ref-millerMeasuringSpaceTimeAccessibility1999}
Miller, H. J. (1999). Measuring space‐time accessibility benefits within transportation networks: Basic theory and computational procedures. \emph{Geographical Analysis}, \emph{31}(1), 187--212. http://doi.org/\href{https://doi.org/10.1111/gean.1999.31.1.187}{10.1111/gean.1999.31.1.187}

\bibitem[\citeproctext]{ref-miller_collaborative_2011}
Miller, H. J. (2011). Collaborative mobility: Using geographic information science to cultivate cooperative transportation systems. \emph{Procedia - Social and Behavioral Sciences}, \emph{21}(0), 24--28. http://doi.org/\url{http://dx.doi.org/10.1016/j.sbspro.2011.07.005}

\bibitem[\citeproctext]{ref-morrisAccessibilityIndicatorsTransport1979}
Morris, J. M., Dumble, P. L., \& Wigan, M. R. (1979). Accessibility indicators for transport planning. \emph{Transportation Research Part A: General}, \emph{13}(2), 91--109. http://doi.org/\href{https://doi.org/10.1016/0191-2607(79)90012-8}{10.1016/0191-2607(79)90012-8}

\bibitem[\citeproctext]{ref-naqavi2023mobility}
Naqavi, F., Sundberg, M., Västberg, O. B., Karlström, A., \& Hugosson, M. B. (2023). Mobility constraints and accessibility to work: Application to stockholm. \emph{Transportation Research Part A: Policy and Practice}, \emph{175}, 103790.

\bibitem[\citeproctext]{ref-naturalenglandNatureNearbyAccessible2010}
Natural England. (2010). \emph{Nature nearby: Accessible natural greenspace guidance}. http://www.naturalengland.org.uk/. Retrieved from \url{https://redfrogforum.org/wp-content/uploads/2019/11/67-Nature-Nearby\%E2\%80\%99-Accessible-Natural-Greenspace-Guidance.pdf}

\bibitem[\citeproctext]{ref-neutens_human_2007}
Neutens, T., Witlox, F., Van de Weghe, N., \& De Maeyer, P. (2007). Human interaction spaces under uncertainty. \emph{Transportation Research Record}, \emph{2021}(1), 28--35.

\bibitem[\citeproctext]{ref-ng2022reflection}
Ng, M. K. M., Roper, J., Lee, C. L., \& Pettit, C. (2022). The reflection of income segregation and accessibility cleavages in sydney's house prices. \emph{ISPRS International Journal of Geo-Information}, \emph{11}(7), 413.

\bibitem[\citeproctext]{ref-OECDFrameworks2013}
OECD. (2013). Frameworks and sector policies for urban development in chile. In \emph{OECD urban policy reviews, chile 2013}. Book Section. http://doi.org/\url{http://dx.doi.org/10.1787/9789264191808-en}

\bibitem[\citeproctext]{ref-ortuzar_2011_modelling}
Ortúzar, J. D., \& Willumsen, L. G. (2011). \emph{Modelling transport}. Book, New York: Wiley.

\bibitem[\citeproctext]{ref-paezDemandLevelService2019}
Paez, Antonio, Higgins, C. D., \& Vivona, S. F. (2019). Demand and level of service inflation in floating catchment area ({FCA}) methods. \emph{{PLOS} {ONE}}, \emph{14}(6), e0218773. http://doi.org/\href{https://doi.org/10.1371/journal.pone.0218773}{10.1371/journal.pone.0218773}

\bibitem[\citeproctext]{ref-paez_developing_2013}
Paez, A., Moniruzzaman, M., Bourbonnais, P. L., \& Morency, C. (2013). Developing a web-based accessibility calculator prototype for the {Greater} {Montreal} {Area}. \emph{Transportation Research Part A-Policy and Practice}, \emph{58}, 103--115. http://doi.org/\href{https://doi.org/10.1016/j.tra.2013.10.020}{10.1016/j.tra.2013.10.020}

\bibitem[\citeproctext]{ref-paez2012measuring}
Paez, A., Scott, D. M., \& Morency, C. (2012). Measuring accessibility: Positive and normative implementations of various accessibility indicators. \emph{Journal of Transport Geography}, \emph{25}, 141--153. Journal Article. http://doi.org/\href{https://doi.org/10.1016/j.jtrangeo.2012.03.016}{10.1016/j.jtrangeo.2012.03.016}

\bibitem[\citeproctext]{ref-paez_jobs_2013}
Páez, Antonio, Farber, S., Mercado, R., Roorda, M., \& Morency, C. (2013). Jobs and the {Single} {Parent}: {An} {Analysis} of {Accessibility} to {Employment} in {Toronto}. \emph{Urban Geography}, \emph{34}(6), 815--842. http://doi.org/\href{https://doi.org/10.1080/02723638.2013.778600}{10.1080/02723638.2013.778600}

\bibitem[\citeproctext]{ref-paezAccessibilityHealthCare2010}
Páez, Antonio, Mercado, R. G., Farber, S., Morency, C., \& Roorda, M. (2010). Accessibility to health care facilities in montreal island: An application of relative accessibility indicators from the perspective of senior and non-senior residents. \emph{International Journal of Health Geographics}, \emph{9}(1), 52. http://doi.org/\href{https://doi.org/10.1186/1476-072X-9-52}{10.1186/1476-072X-9-52}

\bibitem[\citeproctext]{ref-paezRelative2010}
Páez, A., Mercado, R. G., Farber, S., Morency, C., \& Roorda, M. (2010). Relative accessibility deprivation indicators for urban settings: Definitions and application to food deserts in montreal. \emph{Urban Studies}, \emph{47}(7), 1415--1438. Journal Article. http://doi.org/\href{https://doi.org/10.1177/0042098009353626}{10.1177/0042098009353626}

\bibitem[\citeproctext]{ref-pan2013impacts}
Pan, Q. (2013). The impacts of an urban light rail system on residential property values: A case study of the houston METRORail transit line. \emph{Transportation Planning and Technology}, \emph{36}(2), 145--169.

\bibitem[\citeproctext]{ref-pan2020measuring}
Pan, Q., Jin, Z., \& Liu, X. (2020). Measuring the effects of job competition and matching on employment accessibility. \emph{Transportation Research Part D: Transport and Environment}, \emph{87}, 102535.

\bibitem[\citeproctext]{ref-pereira_2021_geographic}
Pereira, R. H. M., Braga, C. K. V., Servo, L. M., Serra, B., Amaral, P., Gouveia, N., \& Paez, A. (2021). Geographic access to COVID-19 healthcare in brazil using a balanced float catchment area approach. \emph{Social Science \& Medicine}, \emph{273}, 113773. Journal Article. http://doi.org/\url{https://doi.org/10.1016/j.socscimed.2021.113773}

\bibitem[\citeproctext]{ref-pereiraR5rRapidRealistic2021}
Pereira, R. H. M., Saraiva, M., Herszenhut, D., Braga, C. K. V., \& Conway, M. W. (2021). r5r: Rapid realistic routing on multimodal transport networks with r \(^{\textrm{5}}\) in r. \emph{Findings}. http://doi.org/\href{https://doi.org/10.32866/001c.21262}{10.32866/001c.21262}

\bibitem[\citeproctext]{ref-pirie_measuring_1979}
Pirie, G. H. (1979). Measuring {Accessibility}: {A} {Review} and {Proposal}. \emph{Environment and Planning A: Economy and Space}, \emph{11}(3), 299--312. http://doi.org/\href{https://doi.org/10.1068/a110299}{10.1068/a110299}

\bibitem[\citeproctext]{ref-rau2012spatial}
Rau, H., \& Vega, A. (2012). Spatial (im) mobility and accessibility in i reland: Implications for transport policy. \emph{Growth and Change}, \emph{43}(4), 667--696.

\bibitem[\citeproctext]{ref-ravensteinLawsMigration1885}
Ravenstein, E. G. (1885). The laws of migration paper 1. \emph{Journal of the Royal Statistical Society}, \emph{48}(2), 167--227.

\bibitem[\citeproctext]{ref-ravensteinLawsMigration1889}
Ravenstein, E. G. (1889). The laws of migration paper 2. \emph{Journal of the Royal Statistical Society}, \emph{52}(2), 241--305. http://doi.org/\href{https://doi.org/10.2307/2979333}{10.2307/2979333}

\bibitem[\citeproctext]{ref-reggianiGuestEditorialNew2011}
Reggiani, A., \& Martín, J. C. (2011). Guest editorial: New frontiers in accessibility modelling: An introduction. \emph{Networks and Spatial Economics}, \emph{11}(4), 577--580. http://doi.org/\href{https://doi.org/10.1007/s11067-011-9155-x}{10.1007/s11067-011-9155-x}

\bibitem[\citeproctext]{ref-reilly1929methods}
Reilly, W. J. (1929). \emph{Methods for the study of retail relationships} (No. 2944).

\bibitem[\citeproctext]{ref-reyesAccessibility2014}
Reyes, M., Paez, A., \& Morency, C. (2014). Walking accessibility to urban parks by children: {A} case study of {Montreal}. \emph{Landscape and Urban Planning}, \emph{125}, 38--47. http://doi.org/\href{https://doi.org/10.1016/j.landurbplan.2014.02.002}{10.1016/j.landurbplan.2014.02.002}

\bibitem[\citeproctext]{ref-ribeiro_road_2010}
Ribeiro, A., Antunes, A. P., \& Páez, A. (2010). Road accessibility and cohesion in lagging regions: {Empirical} evidence from {Portugal} based on spatial econometric models. \emph{Journal of Transport Geography}, \emph{18}(1), 125--132.

\bibitem[\citeproctext]{ref-roblot2021participation}
Roblot, M., Boisjoly, G., Francesco, C., \& Martin, T. (2021). Participation in shared mobility: An analysis of the influence of walking and public transport accessibility to vehicles on carsharing membership in montreal, canada. \emph{Transportation Research Record}, \emph{2675}(12), 1160--1171.

\bibitem[\citeproctext]{ref-rojas_accessibility_2016}
Rojas, C., Paez, A., Barbosa, O., \& Carrasco, J. (2016). Accessibility to urban green spaces in {Chilean} cities using adaptive thresholds. \emph{Journal of Transport Geography}, \emph{57}, 227--240. http://doi.org/\href{https://doi.org/10.1016/j.jtrangeo.2016.10.012}{10.1016/j.jtrangeo.2016.10.012}

\bibitem[\citeproctext]{ref-romanillosAccessibilitySchoolsSpatial2018}
Romanillos, G., \& Garcia-Palomares, J. C. (2018). Accessibility to {Schools}: {Spatial} and {Social} {Imbalances} and the {Impact} of {Population} {Density} in {Four} {European} {Cities}. \emph{Journal of Urban Planning and Development}, \emph{144}(4). http://doi.org/\href{https://doi.org/10.1061/(asce)up.1943-5444.0000491}{10.1061/(asce)up.1943-5444.0000491}

\bibitem[\citeproctext]{ref-sahebgharani2019computing}
Sahebgharani, A., Mohammadi, M., \& Haghshenas, H. (2019). Computing spatiotemporal accessibility to urban opportunities: A reliable space-time prism approach in uncertain urban networks. \emph{Computation}, \emph{7}(3), 51.

\bibitem[\citeproctext]{ref-santanapalacios2022}
Santana Palacios, M., \& El-geneidy, A. (2022). Cumulative versus Gravity-based Accessibility Measures: Which One to Use? \emph{Findings}. http://doi.org/\href{https://doi.org/10.32866/001c.32444}{10.32866/001c.32444}

\bibitem[\citeproctext]{ref-SchuurmanMeasuring2010}
Schuurman, N., Berube, M., \& Crooks, V. A. (2010). Measuring potential spatial access to primary health care physicians using a modified gravity model. \emph{Canadian Geographer-Geographe Canadien}, \emph{54}(1), 29--45.

\bibitem[\citeproctext]{ref-seniorGravityModellingEntropy1979}
Senior, M. L. (1979). From gravity modelling to entropy maximizing: A pedagogic guide. \emph{Progress in Human Geography}, \emph{3}(2), 175--210. http://doi.org/\href{https://doi.org/10.1177/030913257900300218}{10.1177/030913257900300218}

\bibitem[\citeproctext]{ref-sharifiasl2023incorporating}
Sharifiasl, S., Kharel, S., \& Pan, Q. (2023). Incorporating job competition and matching to an indicator-based transportation equity analysis for auto and transit in dallas-fort worth area. \emph{Transportation Research Record}, \emph{2677}(12), 240--254.

\bibitem[\citeproctext]{ref-shen1998}
Shen, Q. (1998b). Location characteristics of inner-city neighborhoods and employment accessibility of low-wage workers. \emph{Environment and Planning B: Planning and Design}, \emph{25}(3), 345--365. http://doi.org/\href{https://doi.org/10.1068/b250345}{10.1068/b250345}

\bibitem[\citeproctext]{ref-shenLocationCharacteristicsInnercity1998}
Shen, Q. (1998a). Location characteristics of inner-city neighborhoods and employment accessibility of low-wage workers. \emph{Environment and Planning B: Planning and Design}, \emph{25}(3), 345--365. http://doi.org/\href{https://doi.org/10.1068/b250345}{10.1068/b250345}

\bibitem[\citeproctext]{ref-shen2019segregation}
Shen, Y. (2019). Segregation through space: A scope of the flow-based spatial interaction model. \emph{Journal of Transport Geography}, \emph{76}, 10--23.

\bibitem[\citeproctext]{ref-silvaAccessibilityInstrumentsPlanning2017}
Silva, C., Bertolini, L., Te Brömmelstroet, M., Milakis, D., \& Papa, E. (2017). Accessibility instruments in planning practice: Bridging the implementation gap. \emph{Transport Policy}, \emph{53}, 135--145. http://doi.org/\href{https://doi.org/10.1016/j.tranpol.2016.09.006}{10.1016/j.tranpol.2016.09.006}

\bibitem[\citeproctext]{ref-soukhovSearchingFairnessStandards2025}
Soukhov, A., Aitken, I. T., Palm, M., Farber, S., \& Paez, A. (2025). Searching for fairness standards in the transportation literature. (Forthcoming). http://doi.org/\href{https://doi.org/10.17605/OSF.IO/RSB92}{10.17605/OSF.IO/RSB92}

\bibitem[\citeproctext]{ref-soukhovElectricMobilityEmission2022}
Soukhov, A., Foda, A., \& Mohamed, M. (2022). Electric mobility emission reduction policies: A multi-objective optimization assessment approach. \emph{Energies}, \emph{15}(19), 6905. http://doi.org/\href{https://doi.org/10.3390/en15196905}{10.3390/en15196905}

\bibitem[\citeproctext]{ref-soukhovTenYearsSchool2025}
Soukhov, A., Higgins, C. D., Páez, A., \& Mohamed, M. (2025). Ten years of school closures and consolidations in hamilton, canada and the impact on multimodal accessibility. \emph{Networks and Spatial Economics}. http://doi.org/\href{https://doi.org/10.1007/s11067-025-09677-z}{10.1007/s11067-025-09677-z}

\bibitem[\citeproctext]{ref-soukhovOccupancyGHGEmissions2022}
Soukhov, A., \& Mohamed, M. (2022). Occupancy and {GHG} emissions: Thresholds for disruptive transportation modes and emerging technologies. \emph{Transportation Research Part D: Transport and Environment}, \emph{102}, 103127. http://doi.org/\href{https://doi.org/10.1016/j.trd.2021.103127}{10.1016/j.trd.2021.103127}

\bibitem[\citeproctext]{ref-soukhovIntroducingSpatialAvailability2023}
Soukhov, A., Paez, A., Higgins, C. D., \& Mohamed, M. (2023). Introducing spatial availability, a singly-constrained measure of competitive accessibility {\textbar} {PLOS} {ONE}. \emph{{PLOS} {ONE}}, 1--30. http://doi.org/\href{https://\%20doi.org/10.1371/journal.pone.0278468}{https:// doi.org/10.1371/journal.pone.0278468}

\bibitem[\citeproctext]{ref-soukhovTTS2016RDataSet2023}
Soukhov, A., \& Páez, A. (2023). {TTS}2016R: A data set to study population and employment patterns from the 2016 transportation tomorrow survey in the greater golden horseshoe area, ontario, canada. \emph{Environment and Planning B: Urban Analytics and City Science}, 23998083221146781. http://doi.org/\href{https://doi.org/10.1177/23998083221146781}{10.1177/23998083221146781}

\bibitem[\citeproctext]{ref-soukhovfamilyofaccessibility2025}
Soukhov, A., Pereira, Rafael H M, Higgins, Christopher H., \& Paez, A. (2025). A family of accessibility measures derived from spatial interaction principles. (Forthcoming).

\bibitem[\citeproctext]{ref-soukhovMultimodalSpatialAvailability2024}
Soukhov, A., Tarriño-Ortiz, J., Soria-Lara, J. A., \& Páez, A. (2024). Multimodal spatial availability: A singly-constrained measure of accessibility considering multiple modes. \emph{{PLOS} {ONE}}, \emph{19}(2), e0299077. http://doi.org/\href{https://doi.org/10.1371/journal.pone.0299077}{10.1371/journal.pone.0299077}

\bibitem[\citeproctext]{ref-statcan_DAdef_2021}
Statistics Canada. (2021a). Dissemination area (DA) (Version 2023-07-07). Statistics Canada. Retrieved from \url{https://www12.statcan.gc.ca/census-recensement/2021/ref/dict/az/Definition-eng.cfm?ID=geo021}

\bibitem[\citeproctext]{ref-statcan_DBdef_2021}
Statistics Canada. (2021b). Dissemination area (DA) (Version 2023-07-07). Statistics Canada. Retrieved from \url{https://www12.statcan.gc.ca/census-recensement/2021/ref/dict/az/Definition-eng.cfm?ID=geo014}

\bibitem[\citeproctext]{ref-statcan_reppoint_2021}
Statistics Canada. (2021c). Representative point (Version November 17, 2021). Statistics Canada. Retrieved from \url{https://www12.statcan.gc.ca/census-recensement/2021/ref/dict/az/definition-eng.cfm?ID=geo040}

\bibitem[\citeproctext]{ref-stewartPrinciples1947}
Stewart, J. Q. (1947). Suggested {Principles} of ''{Social} {Physics}''. \emph{Science}, \emph{106}(2748), 179--180.

\bibitem[\citeproctext]{ref-stewartDemographicGravitationEvidence1948}
Stewart, J. Q. (1948). Demographic gravitation: Evidence and applications. \emph{Sociometry}, \emph{11}(1), 31--58. http://doi.org/\href{https://doi.org/10.2307/2785468}{10.2307/2785468}

\bibitem[\citeproctext]{ref-su2023untangling}
Su, R., \& Goulias, K. (2023). Untangling the relationships among residential environment, destination choice, and daily walk accessibility. \emph{Journal of Transport Geography}, \emph{109}, 103595.

\bibitem[\citeproctext]{ref-suel2024measuring}
Suel, E., Lynch, C., Wood, M., Murat, T., Casey, G., \& Dennett, A. (2024). Measuring transport-associated urban inequalities: Where are we and where do we go from here? \emph{Transport Reviews}, \emph{44}(6), 1235--1257.

\bibitem[\citeproctext]{ref-tahmasbiMultimodalAccessibilitybasedEquity2019}
Tahmasbi, B., Mansourianfar, M. H., Haghshenas, H., \& Kim, I. (2019). Multimodal accessibility-based equity assessment of urban public facilities distribution. \emph{Sustainable Cities and Society}, \emph{49}, 101633. http://doi.org/\href{https://doi.org/10.1016/j.scs.2019.101633}{10.1016/j.scs.2019.101633}

\bibitem[\citeproctext]{ref-tao_investigating_2020}
Tao, Z., Zhou, J., Lin, X., Chao, H., \& Li, G. (2020). Investigating the impacts of public transport on job accessibility in {Shenzhen}, {China}: A multi-modal approach. \emph{LAND USE POLICY}, \emph{99}. http://doi.org/\href{https://doi.org/10.1016/j.landusepol.2020.105025}{10.1016/j.landusepol.2020.105025}

\bibitem[\citeproctext]{ref-tong2015transportation}
Tong, L., Zhou, X., \& Miller, H. J. (2015). Transportation network design for maximizing space--time accessibility. \emph{Transportation Research Part B: Methodological}, \emph{81}, 555--576.

\bibitem[\citeproctext]{ref-turk2019socio}
Türk, U. (2019). Socio-economic determinants of student mobility and inequality of access to higher education in italy. \emph{Networks and Spatial Economics}, \emph{19}(1), 125--148.

\bibitem[\citeproctext]{ref-vanweeAccessible2016}
van Wee, B. (2016). Accessible accessibility research challenges. \emph{Journal of Transport Geography}, \emph{51}, 9--16. http://doi.org/\url{https://doi.org/10.1016/j.jtrangeo.2015.10.018}

\bibitem[\citeproctext]{ref-vanweeAccessibilityMeasuresCompetition2001}
Van Wee, B., Hagoort, M., \& Annema, J. A. (2001). Accessibility measures with competition. \emph{Journal of Transport Geography}, \emph{9}(3), 199--208. http://doi.org/\href{https://doi.org/10.1016/S0966-6923(01)00010-2}{10.1016/S0966-6923(01)00010-2}

\bibitem[\citeproctext]{ref-vickermanAccessibilityAttractionPotential1974}
Vickerman, R. W. (1974). Accessibility, attraction, and potential: A review of some concepts and their use in determining mobility. \emph{Environment and Planning A}, \emph{6}, 675--691. http://doi.org/\href{https://doi.org/10.1068/a060675}{10.1068/a060675}

\bibitem[\citeproctext]{ref-vickermanAccessibility1999}
Vickerman, R., Spiekermann, K., \& Wegener, M. (1999). Accessibility and economic development in {Europe}. \emph{Regional Studies}, \emph{33}(1), 1--15.

\bibitem[\citeproctext]{ref-wachs_physical_1973}
Wachs, M., \& Kumagai, T. G. (1973). Physical accessibility as a social indicator. \emph{Socio-Economic Planning Sciences}, \emph{7}(5), 437--456. http://doi.org/\href{https://doi.org/10.1016/0038-0121(73)90041-4}{10.1016/0038-0121(73)90041-4}

\bibitem[\citeproctext]{ref-wan2012three}
Wan, N., Zou, B., \& Sternberg, T. (2012). A three-step floating catchment area method for analyzing spatial access to health services. \emph{International Journal of Geographical Information Science}, \emph{26}(6), 1073--1089. Journal Article. http://doi.org/\href{https://doi.org/10.1080/13658816.2011.624987}{10.1080/13658816.2011.624987}

\bibitem[\citeproctext]{ref-wang_2sfca_2021}
Wang, F. (2021). From {2SFCA} to {i2SFCA}: Integration, derivation and validation. \emph{International Journal of Geographical Information Science}, \emph{35}(3), 628--638. http://doi.org/\href{https://doi.org/10.1080/13658816.2020.1811868}{10.1080/13658816.2020.1811868}

\bibitem[\citeproctext]{ref-wang_inverted_2018}
Wang, F. H. (2018). Inverted {Two}-{Step} {Floating} {Catchment} {Area} {Method} for {Measuring} {Facility} {Crowdedness}. \emph{Professional Geographer}, \emph{70}(2), 251--260. http://doi.org/\href{https://doi.org/10.1080/00330124.2017.1365308}{10.1080/00330124.2017.1365308}

\bibitem[\citeproctext]{ref-weibull_axiomatic_1976}
Weibull, Jörgen W. (1976). An axiomatic approach to the measurement of accessibility. \emph{Regional Science and Urban Economics}, \emph{6}(4), 357--379. http://doi.org/\href{https://doi.org/10.1016/0166-0462(76)90031-4}{10.1016/0166-0462(76)90031-4}

\bibitem[\citeproctext]{ref-weibullNumericalMeasurementAccessibility1980}
Weibull, J. W. (1980). On the numerical measurement of accessibility. \emph{Environment and Planning A: Economy and Space}, \emph{12}(1), 53--67. http://doi.org/\href{https://doi.org/10.1068/a120053}{10.1068/a120053}

\bibitem[\citeproctext]{ref-weinerUrbanTransportationPlanning2016}
Weiner, E. (2016). \emph{Urban transportation planning in the united states}. Springer Cham: Springer International Publishing. http://doi.org/\href{https://doi.org/10.1007/978-3-319-39975-1}{10.1007/978-3-319-39975-1}

\bibitem[\citeproctext]{ref-whoMedicalDoctors102025}
WHO. (2025). Medical doctors (per 10 000 population). Retrieved February 19, 2025, from \url{https://www.who.int/data/gho/data/indicators/indicator-details/GHO/medical-doctors-(per-10-000-population)}

\bibitem[\citeproctext]{ref-williams_disparities_2014}
Williams, S., \& Wang, F. H. (2014). Disparities in accessibility of public high schools, in metropolitan {Baton} {Rouge}, {Louisiana} 1990-2010. \emph{Urban Geography}, \emph{35}(7), 1066--1083. http://doi.org/\href{https://doi.org/10.1080/02723638.2014.936668}{10.1080/02723638.2014.936668}

\bibitem[\citeproctext]{ref-willigers2007accessibility}
Willigers, J., Floor, H., \& van Wee, B. (2007). Accessibility indicators for location choices of offices: An application to the intraregional distributive effects of high-speed rail in the netherlands. \emph{Environment and Planning A}, \emph{39}(9), 2086--2898.

\bibitem[\citeproctext]{ref-wilsonSTATISTICALTHEORYSPATIAL1967}
Wilson, A. G. (1967). A {STATISTICAL} {THEORY} {OF} {SPATIAL} {DISTRIBUTION} {MODELS}. \emph{Transportation Research}, \emph{1}, 253--269. Retrieved from \url{https://journals-scholarsportal-info.libaccess.lib.mcmaster.ca/pdf/00411647/v01i0003/253_astosdm.xml_en}

\bibitem[\citeproctext]{ref-wilson1971}
Wilson, A. G. (1971). A Family of Spatial Interaction Models, and Associated Developments. \emph{Environment and Planning A: Economy and Space}, \emph{3}(1), 1--32. http://doi.org/\href{https://doi.org/10.1068/a030001}{10.1068/a030001}

\bibitem[\citeproctext]{ref-wuUnifyingAccess2020}
Wu, H., \& Levinson, D. (2020). Unifying access. \emph{Transportation Research Part D: Transport and Environment}, \emph{83}, 102355. http://doi.org/\href{https://doi.org/10.1016/j.trd.2020.102355}{10.1016/j.trd.2020.102355}

\bibitem[\citeproctext]{ref-ye_spatial_2018}
Ye, C., Zhu, Y., Yang, J., \& Fu, Q. (2018). Spatial equity in accessing secondary education: {Evidence} from a gravity-based model: {Spatial} equity in accessing secondary education. \emph{The Canadian Geographer / Le Géographe Canadien}, \emph{62}(4), 452--469. http://doi.org/\href{https://doi.org/10.1111/cag.12482}{10.1111/cag.12482}

\bibitem[\citeproctext]{ref-zipfDeterminantsCirculationInformation1946}
Zipf, G. K. (1946a). Some determinants of the circulation of information. \emph{The American Journal of Psychology}, \emph{59}(3), 401--421. http://doi.org/\href{https://doi.org/10.2307/1417611}{10.2307/1417611}

\bibitem[\citeproctext]{ref-zipfHypothesisIntercityMovement1946}
Zipf, G. K. (1946b). The \emph{p} \(_{\textrm{1}}\) \emph{p} \(_{\textrm{2}}\) / \emph{d} hypothesis: On the intercity movement of persons, \emph{11}(6), 677--686.

\bibitem[\citeproctext]{ref-zipfHypothesisCaseRailway1946}
Zipf, G. K. (1946c). The \emph{p} \(_{\textrm{1}}\) \emph{p} \(_{\textrm{2}}\) / \emph{d} hypothesis: The case of railway express. \emph{The Journal of Psychology}, \emph{22}(1), 3--8. http://doi.org/\href{https://doi.org/10.1080/00223980.1946.9917292}{10.1080/00223980.1946.9917292}

\end{CSLReferences}

\end{document}
